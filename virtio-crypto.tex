\section{Crypto Device}\label{sec:Device Types / Crypto Device}

The virtio crypto device is a virtual cryptography device as well as a
virtual cryptographic accelerator. The virtio crypto device provides the
following crypto services: CIPHER, MAC, HASH, AEAD and AKCIPHER. Virtio crypto
devices have a single control queue and at least one data queue. Crypto
operation requests are placed into a data queue, and serviced by the
device. Some crypto operation requests are only valid in the context of a
session. The role of the control queue is facilitating control operation
requests. Sessions management is realized with control operation
requests.

\subsection{Device ID}\label{sec:Device Types / Crypto Device / Device ID}

20

\subsection{Virtqueues}\label{sec:Device Types / Crypto Device / Virtqueues}

\begin{description}
\item[0] dataq1
\item[\ldots]
\item[N-1] dataqN
\item[N] controlq
\end{description}

N is set by \field{max_dataqueues}.

\subsection{Feature bits}\label{sec:Device Types / Crypto Device / Feature bits}

\begin{description}
\item VIRTIO_CRYPTO_F_REVISION_1 (0) revision 1. Revision 1 has a specific
    request format and other enhancements (which result in some additional
    requirements).
\item VIRTIO_CRYPTO_F_CIPHER_STATELESS_MODE (1) stateless mode requests are
    supported by the CIPHER service.
\item VIRTIO_CRYPTO_F_HASH_STATELESS_MODE (2) stateless mode requests are
    supported by the HASH service.
\item VIRTIO_CRYPTO_F_MAC_STATELESS_MODE (3) stateless mode requests are
    supported by the MAC service.
\item VIRTIO_CRYPTO_F_AEAD_STATELESS_MODE (4) stateless mode requests are
    supported by the AEAD service.
\item VIRTIO_CRYPTO_F_AKCIPHER_STATELESS_MODE (5) stateless mode requests are
    supported by the AKCIPHER service.
\end{description}


\subsubsection{Feature bit requirements}\label{sec:Device Types / Crypto Device / Feature bit requirements}

Some crypto feature bits require other crypto feature bits
(see \ref{drivernormative:Basic Facilities of a Virtio Device / Feature Bits}):

\begin{description}
\item[VIRTIO_CRYPTO_F_CIPHER_STATELESS_MODE] Requires VIRTIO_CRYPTO_F_REVISION_1.
\item[VIRTIO_CRYPTO_F_HASH_STATELESS_MODE] Requires VIRTIO_CRYPTO_F_REVISION_1.
\item[VIRTIO_CRYPTO_F_MAC_STATELESS_MODE] Requires VIRTIO_CRYPTO_F_REVISION_1.
\item[VIRTIO_CRYPTO_F_AEAD_STATELESS_MODE] Requires VIRTIO_CRYPTO_F_REVISION_1.
\item[VIRTIO_CRYPTO_F_AKCIPHER_STATELESS_MODE] Requires VIRTIO_CRYPTO_F_REVISION_1.
\end{description}

\subsection{Supported crypto services}\label{sec:Device Types / Crypto Device / Supported crypto services}

The following crypto services are defined:

\begin{lstlisting}
/* CIPHER (Symmetric Key Cipher) service */
#define VIRTIO_CRYPTO_SERVICE_CIPHER 0
/* HASH service */
#define VIRTIO_CRYPTO_SERVICE_HASH   1
/* MAC (Message Authentication Codes) service */
#define VIRTIO_CRYPTO_SERVICE_MAC    2
/* AEAD (Authenticated Encryption with Associated Data) service */
#define VIRTIO_CRYPTO_SERVICE_AEAD   3
/* AKCIPHER (Asymmetric Key Cipher) service */
#define VIRTIO_CRYPTO_SERVICE_AKCIPHER 4
\end{lstlisting}

The above constants designate bits used to indicate the which of crypto services are
offered by the device as described in, see \ref{sec:Device Types / Crypto Device / Device configuration layout}.

\subsubsection{CIPHER services}\label{sec:Device Types / Crypto Device / Supported crypto services / CIPHER services}

The following CIPHER algorithms are defined:

\begin{lstlisting}
#define VIRTIO_CRYPTO_NO_CIPHER                 0
#define VIRTIO_CRYPTO_CIPHER_ARC4               1
#define VIRTIO_CRYPTO_CIPHER_AES_ECB            2
#define VIRTIO_CRYPTO_CIPHER_AES_CBC            3
#define VIRTIO_CRYPTO_CIPHER_AES_CTR            4
#define VIRTIO_CRYPTO_CIPHER_DES_ECB            5
#define VIRTIO_CRYPTO_CIPHER_DES_CBC            6
#define VIRTIO_CRYPTO_CIPHER_3DES_ECB           7
#define VIRTIO_CRYPTO_CIPHER_3DES_CBC           8
#define VIRTIO_CRYPTO_CIPHER_3DES_CTR           9
#define VIRTIO_CRYPTO_CIPHER_KASUMI_F8          10
#define VIRTIO_CRYPTO_CIPHER_SNOW3G_UEA2        11
#define VIRTIO_CRYPTO_CIPHER_AES_F8             12
#define VIRTIO_CRYPTO_CIPHER_AES_XTS            13
#define VIRTIO_CRYPTO_CIPHER_ZUC_EEA3           14
\end{lstlisting}

The above constants have two usages:
\begin{enumerate}
\item As bit numbers, used to tell the driver which CIPHER algorithms
are supported by the device, see \ref{sec:Device Types / Crypto Device / Device configuration layout}.
\item As values, used to designate the algorithm in (CIPHER type) crypto
operation requests, see \ref{sec:Device Types / Crypto Device / Device Operation / Control Virtqueue / Session operation}.
\end{enumerate}

\subsubsection{HASH services}\label{sec:Device Types / Crypto Device / Supported crypto services / HASH services}

The following HASH algorithms are defined:

\begin{lstlisting}
#define VIRTIO_CRYPTO_NO_HASH            0
#define VIRTIO_CRYPTO_HASH_MD5           1
#define VIRTIO_CRYPTO_HASH_SHA1          2
#define VIRTIO_CRYPTO_HASH_SHA_224       3
#define VIRTIO_CRYPTO_HASH_SHA_256       4
#define VIRTIO_CRYPTO_HASH_SHA_384       5
#define VIRTIO_CRYPTO_HASH_SHA_512       6
#define VIRTIO_CRYPTO_HASH_SHA3_224      7
#define VIRTIO_CRYPTO_HASH_SHA3_256      8
#define VIRTIO_CRYPTO_HASH_SHA3_384      9
#define VIRTIO_CRYPTO_HASH_SHA3_512      10
#define VIRTIO_CRYPTO_HASH_SHA3_SHAKE128      11
#define VIRTIO_CRYPTO_HASH_SHA3_SHAKE256      12
\end{lstlisting}

The above constants have two usages:
\begin{enumerate}
\item As bit numbers, used to tell the driver which HASH algorithms
are supported by the device, see \ref{sec:Device Types / Crypto Device / Device configuration layout}.
\item As values, used to designate the algorithm in (HASH type) crypto
operation requires, see \ref{sec:Device Types / Crypto Device / Device Operation / Control Virtqueue / Session operation}.
\end{enumerate}

\subsubsection{MAC services}\label{sec:Device Types / Crypto Device / Supported crypto services / MAC services}

The following MAC algorithms are defined:

\begin{lstlisting}
#define VIRTIO_CRYPTO_NO_MAC                       0
#define VIRTIO_CRYPTO_MAC_HMAC_MD5                 1
#define VIRTIO_CRYPTO_MAC_HMAC_SHA1                2
#define VIRTIO_CRYPTO_MAC_HMAC_SHA_224             3
#define VIRTIO_CRYPTO_MAC_HMAC_SHA_256             4
#define VIRTIO_CRYPTO_MAC_HMAC_SHA_384             5
#define VIRTIO_CRYPTO_MAC_HMAC_SHA_512             6
#define VIRTIO_CRYPTO_MAC_CMAC_3DES                25
#define VIRTIO_CRYPTO_MAC_CMAC_AES                 26
#define VIRTIO_CRYPTO_MAC_KASUMI_F9                27
#define VIRTIO_CRYPTO_MAC_SNOW3G_UIA2              28
#define VIRTIO_CRYPTO_MAC_GMAC_AES                 41
#define VIRTIO_CRYPTO_MAC_GMAC_TWOFISH             42
#define VIRTIO_CRYPTO_MAC_CBCMAC_AES               49
#define VIRTIO_CRYPTO_MAC_CBCMAC_KASUMI_F9         50
#define VIRTIO_CRYPTO_MAC_XCBC_AES                 53
#define VIRTIO_CRYPTO_MAC_ZUC_EIA3                 54
\end{lstlisting}

The above constants have two usages:
\begin{enumerate}
\item As bit numbers, used to tell the driver which MAC algorithms
are supported by the device, see \ref{sec:Device Types / Crypto Device / Device configuration layout}.
\item As values, used to designate the algorithm in (MAC type) crypto
operation requests, see \ref{sec:Device Types / Crypto Device / Device Operation / Control Virtqueue / Session operation}.
\end{enumerate}

\subsubsection{AEAD services}\label{sec:Device Types / Crypto Device / Supported crypto services / AEAD services}

The following AEAD algorithms are defined:

\begin{lstlisting}
#define VIRTIO_CRYPTO_NO_AEAD     0
#define VIRTIO_CRYPTO_AEAD_GCM    1
#define VIRTIO_CRYPTO_AEAD_CCM    2
#define VIRTIO_CRYPTO_AEAD_CHACHA20_POLY1305  3
\end{lstlisting}

The above constants have two usages:
\begin{enumerate}
\item As bit numbers, used to tell the driver which AEAD algorithms
are supported by the device, see \ref{sec:Device Types / Crypto Device / Device configuration layout}.
\item As values, used to designate the algorithm in (DEAD type) crypto
operation requests, see \ref{sec:Device Types / Crypto Device / Device Operation / Control Virtqueue / Session operation}.
\end{enumerate}

\subsubsection{AKCIPHER services}\label{sec: Device Types / Crypto Device / Supported crypto services / AKCIPHER services}

The following AKCIPHER algorithms are defined:
\begin{lstlisting}
#define VIRTIO_CRYPTO_NO_AKCIPHER 0
#define VIRTIO_CRYPTO_AKCIPHER_RSA   1
#define VIRTIO_CRYPTO_AKCIPHER_ECDSA 2
\end{lstlisting}

The above constants have two usages:
\begin{enumerate}
\item As bit numbers, used to tell the driver which AKCIPHER algorithms
are supported by the device, see \ref{sec:Device Types / Crypto Device / Device configuration layout}.
\item As values, used to designate the algorithm in asymmetric crypto operation requests,
see \ref{sec:Device Types / Crypto Device / Device Operation / Control Virtqueue / Session operation}.
\end{enumerate}


\subsection{Device configuration layout}\label{sec:Device Types / Crypto Device / Device configuration layout}

Crypto device configuration uses the following layout structure:

\begin{lstlisting}
struct virtio_crypto_config {
    le32 status;
    le32 max_dataqueues;
    le32 crypto_services;
    /* Detailed algorithms mask */
    le32 cipher_algo_l;
    le32 cipher_algo_h;
    le32 hash_algo;
    le32 mac_algo_l;
    le32 mac_algo_h;
    le32 aead_algo;
    /* Maximum length of cipher key in bytes */
    le32 max_cipher_key_len;
    /* Maximum length of authenticated key in bytes */
    le32 max_auth_key_len;
    le32 akcipher_algo;
    /* Maximum size of each crypto request's content in bytes */
    le64 max_size;
};
\end{lstlisting}

\begin{description}
\item Currently, only one \field{status} bit is defined: VIRTIO_CRYPTO_S_HW_READY
    set indicates that the device is ready to process requests, this bit is read-only
    for the driver
\begin{lstlisting}
#define VIRTIO_CRYPTO_S_HW_READY  (1 << 0)
\end{lstlisting}

\item [\field{max_dataqueues}] is the maximum number of data virtqueues that can
    be configured by the device. The driver MAY use only one data queue, or it
    can use more to achieve better performance.

\item [\field{crypto_services}] crypto service offered, see \ref{sec:Device Types / Crypto Device / Supported crypto services}.

\item [\field{cipher_algo_l}] CIPHER algorithms bits 0-31, see \ref{sec:Device Types / Crypto Device / Supported crypto services  / CIPHER services}.

\item [\field{cipher_algo_h}] CIPHER algorithms bits 32-63, see \ref{sec:Device Types / Crypto Device / Supported crypto services  / CIPHER services}.

\item [\field{hash_algo}] HASH algorithms bits, see \ref{sec:Device Types / Crypto Device / Supported crypto services  / HASH services}.

\item [\field{mac_algo_l}] MAC algorithms bits 0-31, see \ref{sec:Device Types / Crypto Device / Supported crypto services  / MAC services}.

\item [\field{mac_algo_h}] MAC algorithms bits 32-63, see \ref{sec:Device Types / Crypto Device / Supported crypto services  / MAC services}.

\item [\field{aead_algo}] AEAD algorithms bits, see \ref{sec:Device Types / Crypto Device / Supported crypto services  / AEAD services}.

\item [\field{max_cipher_key_len}] is the maximum length of cipher key supported by the device.

\item [\field{max_auth_key_len}] is the maximum length of authenticated key supported by the device.

\item [\field{akcipher_algo}] AKCIPHER algorithms bit 0-31, see \ref{sec: Device Types / Crypto Device / Supported crypto services / AKCIPHER services}.

\item [\field{max_size}] is the maximum size of the variable-length parameters of
    data operation of each crypto request's content supported by the device.
\end{description}

\begin{note}
Unless explicitly stated otherwise all lengths and sizes are in bytes.
\end{note}

\devicenormative{\subsubsection}{Device configuration layout}{Device Types / Crypto Device / Device configuration layout}

\begin{itemize*}
\item The device MUST set \field{max_dataqueues} to between 1 and 65535 inclusive.
\item The device MUST set the \field{status} with valid flags, undefined flags MUST NOT be set.
\item The device MUST accept and handle requests after \field{status} is set to VIRTIO_CRYPTO_S_HW_READY.
\item The device MUST set \field{crypto_services} based on the crypto services the device offers.
\item The device MUST set detailed algorithms masks for each service advertised by \field{crypto_services}.
    The device MUST NOT set the not defined algorithms bits.
\item The device MUST set \field{max_size} to show the maximum size of crypto request the device supports.
\item The device MUST set \field{max_cipher_key_len} to show the maximum length of cipher key if the
    device supports CIPHER service.
\item The device MUST set \field{max_auth_key_len} to show the maximum length of authenticated key if
    the device supports MAC service.
\end{itemize*}

\drivernormative{\subsubsection}{Device configuration layout}{Device Types / Crypto Device / Device configuration layout}

\begin{itemize*}
\item The driver MUST read the \field{status} from the bottom bit of status to check whether the
    VIRTIO_CRYPTO_S_HW_READY is set, and the driver MUST reread it after device reset.
\item The driver MUST NOT transmit any requests to the device if the VIRTIO_CRYPTO_S_HW_READY is not set.
\item The driver MUST read \field{max_dataqueues} field to discover the number of data queues the device supports.
\item The driver MUST read \field{crypto_services} field to discover which services the device is able to offer.
\item The driver SHOULD ignore the not defined algorithms bits.
\item The driver MUST read the detailed algorithms fields based on \field{crypto_services} field.
\item The driver SHOULD read \field{max_size} to discover the maximum size of the variable-length
    parameters of data operation of the crypto request's content the device supports and MUST
    guarantee that the size of each crypto request's content is within the \field{max_size}, otherwise
    the request will fail and the driver MUST reset the device.
\item The driver SHOULD read \field{max_cipher_key_len} to discover the maximum length of cipher key
    the device supports and MUST guarantee that the \field{key_len} (CIPHER service or AEAD service) is within
    the \field{max_cipher_key_len} of the device configuration, otherwise the request will fail.
\item The driver SHOULD read \field{max_auth_key_len} to discover the maximum length of authenticated
    key the device supports and MUST guarantee that the \field{auth_key_len} (MAC service) is within the
    \field{max_auth_key_len} of the device configuration, otherwise the request will fail.
\end{itemize*}

\subsection{Device Initialization}\label{sec:Device Types / Crypto Device / Device Initialization}

\drivernormative{\subsubsection}{Device Initialization}{Device Types / Crypto Device / Device Initialization}

\begin{itemize*}
\item The driver MUST configure and initialize all virtqueues.
\item The driver MUST read the supported crypto services from bits of \field{crypto_services}.
\item The driver MUST read the supported algorithms based on \field{crypto_services} field.
\end{itemize*}

\subsection{Device Operation}\label{sec:Device Types / Crypto Device / Device Operation}

The operation of a virtio crypto device is driven by requests placed on the virtqueues.
Requests consist of a queue-type specific header (specifying among others the operation)
and an operation specific payload.

If VIRTIO_CRYPTO_F_REVISION_1 is negotiated the device may support both session mode
(See \ref{sec:Device Types / Crypto Device / Device Operation / Control Virtqueue / Session operation})
and stateless mode operation requests.
In stateless mode all operation parameters are supplied as a part of each request,
while in session mode, some or all operation parameters are managed within the
session. Stateless mode is guarded by feature bits 0-4 on a service level. If
stateless mode is negotiated for a service, the service accepts both session
mode and stateless requests; otherwise stateless mode requests are rejected
(via operation status).

\subsubsection{Operation Status}\label{sec:Device Types / Crypto Device / Device Operation / Operation status}
The device MUST return a status code as part of the operation (both session
operation and service operation) result. The valid operation status as follows:

\begin{lstlisting}
enum VIRTIO_CRYPTO_STATUS {
    VIRTIO_CRYPTO_OK = 0,
    VIRTIO_CRYPTO_ERR = 1,
    VIRTIO_CRYPTO_BADMSG = 2,
    VIRTIO_CRYPTO_NOTSUPP = 3,
    VIRTIO_CRYPTO_INVSESS = 4,
    VIRTIO_CRYPTO_NOSPC = 5,
    VIRTIO_CRYPTO_KEY_REJECTED = 6,
    VIRTIO_CRYPTO_MAX
};
\end{lstlisting}

\begin{itemize*}
\item VIRTIO_CRYPTO_OK: success.
\item VIRTIO_CRYPTO_BADMSG: authentication failed (only when AEAD decryption).
\item VIRTIO_CRYPTO_NOTSUPP: operation or algorithm is unsupported.
\item VIRTIO_CRYPTO_INVSESS: invalid session ID when executing crypto operations.
\item VIRTIO_CRYPTO_NOSPC: no free session ID (only when the VIRTIO_CRYPTO_F_REVISION_1
    feature bit is negotiated).
\item VIRTIO_CRYPTO_KEY_REJECTED: signature verification failed (only when AKCIPHER verification).
\item VIRTIO_CRYPTO_ERR: any failure not mentioned above occurs.
\end{itemize*}

\subsubsection{Control Virtqueue}\label{sec:Device Types / Crypto Device / Device Operation / Control Virtqueue}

The driver uses the control virtqueue to send control commands to the
device, such as session operations (See \ref{sec:Device Types / Crypto Device / Device
Operation / Control Virtqueue / Session operation}).

The header for controlq is of the following form:
\begin{lstlisting}
#define VIRTIO_CRYPTO_OPCODE(service, op)   (((service) << 8) | (op))

struct virtio_crypto_ctrl_header {
#define VIRTIO_CRYPTO_CIPHER_CREATE_SESSION \
       VIRTIO_CRYPTO_OPCODE(VIRTIO_CRYPTO_SERVICE_CIPHER, 0x02)
#define VIRTIO_CRYPTO_CIPHER_DESTROY_SESSION \
       VIRTIO_CRYPTO_OPCODE(VIRTIO_CRYPTO_SERVICE_CIPHER, 0x03)
#define VIRTIO_CRYPTO_HASH_CREATE_SESSION \
       VIRTIO_CRYPTO_OPCODE(VIRTIO_CRYPTO_SERVICE_HASH, 0x02)
#define VIRTIO_CRYPTO_HASH_DESTROY_SESSION \
       VIRTIO_CRYPTO_OPCODE(VIRTIO_CRYPTO_SERVICE_HASH, 0x03)
#define VIRTIO_CRYPTO_MAC_CREATE_SESSION \
       VIRTIO_CRYPTO_OPCODE(VIRTIO_CRYPTO_SERVICE_MAC, 0x02)
#define VIRTIO_CRYPTO_MAC_DESTROY_SESSION \
       VIRTIO_CRYPTO_OPCODE(VIRTIO_CRYPTO_SERVICE_MAC, 0x03)
#define VIRTIO_CRYPTO_AEAD_CREATE_SESSION \
       VIRTIO_CRYPTO_OPCODE(VIRTIO_CRYPTO_SERVICE_AEAD, 0x02)
#define VIRTIO_CRYPTO_AEAD_DESTROY_SESSION \
       VIRTIO_CRYPTO_OPCODE(VIRTIO_CRYPTO_SERVICE_AEAD, 0x03)
#define VIRTIO_CRYPTO_AKCIPHER_CREATE_SESSION \
       VIRTIO_CRYPTO_OPCODE(VIRTIO_CRYPTO_SERVICE_AKCIPHER, 0x04)
#define VIRTIO_CRYPTO_AKCIPHER_DESTROY_SESSION \
       VIRTIO_CRYPTO_OPCDE(VIRTIO_CRYPTO_SERVICE_AKCIPHER, 0x05)
    le32 opcode;
    /* algo should be service-specific algorithms */
    le32 algo;
    le32 flag;
    le32 reserved;
};
\end{lstlisting}

The controlq request is composed of four parts:
\begin{lstlisting}
struct virtio_crypto_op_ctrl_req {
    /* Device read only portion */

    struct virtio_crypto_ctrl_header header;

#define VIRTIO_CRYPTO_CTRLQ_OP_SPEC_HDR_LEGACY 56
    /* fixed length fields, opcode specific */
    u8 op_flf[flf_len];

    /* variable length fields, opcode specific */
    u8 op_vlf[vlf_len];

    /* Device write only portion */

    /* op result or completion status */
    u8 op_outcome[outcome_len];
};
\end{lstlisting}

\field{header} is a general header (see above).

\field{op_flf} is the opcode (in \field{header}) specific fixed-length paramenters.

\field{flf_len} depends on the VIRTIO_CRYPTO_F_REVISION_1 feature bit (see below).

\field{op_vlf} is the opcode (in \field{header}) specific variable-length paramenters.

\field{vlf_len} is the size of the specific structure used.
\begin{note}
The \field{vlf_len} of session-destroy operation and the hash-session-create
operation is ZERO.
\end{note}

\begin{itemize*}
\item If the opcode (in \field{header}) is VIRTIO_CRYPTO_CIPHER_CREATE_SESSION
    then \field{op_flf} is struct virtio_crypto_sym_create_session_flf if
    VIRTIO_CRYPTO_F_REVISION_1 is negotiated and struct virtio_crypto_sym_create_session_flf is
    padded to 56 bytes if NOT negotiated, and \field{op_vlf} is struct
    virtio_crypto_sym_create_session_vlf.
\item If the opcode (in \field{header}) is VIRTIO_CRYPTO_HASH_CREATE_SESSION
    then \field{op_flf} is struct virtio_crypto_hash_create_session_flf if
    VIRTIO_CRYPTO_F_REVISION_1 is negotiated and struct virtio_crypto_hash_create_session_flf is
    padded to 56 bytes if NOT negotiated.
\item If the opcode (in \field{header}) is VIRTIO_CRYPTO_MAC_CREATE_SESSION
    then \field{op_flf} is struct virtio_crypto_mac_create_session_flf if
    VIRTIO_CRYPTO_F_REVISION_1 is negotiated and struct virtio_crypto_mac_create_session_flf is
    padded to 56 bytes if NOT negotiated, and \field{op_vlf} is struct
    virtio_crypto_mac_create_session_vlf.
\item If the opcode (in \field{header}) is VIRTIO_CRYPTO_AEAD_CREATE_SESSION
    then \field{op_flf} is struct virtio_crypto_aead_create_session_flf if
    VIRTIO_CRYPTO_F_REVISION_1 is negotiated and struct virtio_crypto_aead_create_session_flf is
    padded to 56 bytes if NOT negotiated, and \field{op_vlf} is struct
    virtio_crypto_aead_create_session_vlf.
\item If the opcode (in \field{header}) is VIRTIO_CRYPTO_AKCIPHER_CREATE_SESSION
    then \field{op_flf} is struct virtio_crypto_akcipher_create_session_flf if
    VIRTIO_CRYPTO_F_REVISION_1 is negotiated and struct virtio_crypto_akcipher_create_session_flf is
    padded to 56 bytes if NOT negotiated, and \field{op_vlf} is struct
    virtio_crypto_akcipher_create_session_vlf.
\item If the opcode (in \field{header}) is VIRTIO_CRYPTO_CIPHER_DESTROY_SESSION
    or VIRTIO_CRYPTO_HASH_DESTROY_SESSION or VIRTIO_CRYPTO_MAC_DESTROY_SESSION or
    VIRTIO_CRYPTO_AEAD_DESTROY_SESSION then \field{op_flf} is struct
    virtio_crypto_destroy_session_flf if VIRTIO_CRYPTO_F_REVISION_1 is negotiated and
    struct virtio_crypto_destroy_session_flf is padded to 56 bytes if NOT negotiated.
\end{itemize*}

\field{op_outcome} stores the result of operation and must be struct
virtio_crypto_destroy_session_input for destroy session or
struct virtio_crypto_create_session_input for create session.

\field{outcome_len} is the size of the structure used.


\paragraph{Session operation}\label{sec:Device Types / Crypto Device / Device
Operation / Control Virtqueue / Session operation}

The session is a handle which describes the cryptographic parameters to be
applied to a number of buffers.

The following structure stores the result of session creation set by the device:

\begin{lstlisting}
struct virtio_crypto_create_session_input {
    le64 session_id;
    le32 status;
    le32 padding;
};
\end{lstlisting}

A request to destroy a session includes the following information:

\begin{lstlisting}
struct virtio_crypto_destroy_session_flf {
    /* Device read only portion */
    le64  session_id;
};

struct virtio_crypto_destroy_session_input {
    /* Device write only portion */
    u8  status;
};
\end{lstlisting}


\subparagraph{Session operation: HASH session}\label{sec:Device Types / Crypto Device / Device
Operation / Control Virtqueue / Session operation / Session operation: HASH session}

The fixed-length paramenters of HASH session requests is as follows:

\begin{lstlisting}
struct virtio_crypto_hash_create_session_flf {
    /* Device read only portion */

    /* See VIRTIO_CRYPTO_HASH_* above */
    le32 algo;
    /* hash result length */
    le32 hash_result_len;
};
\end{lstlisting}


\subparagraph{Session operation: MAC session}\label{sec:Device Types / Crypto Device / Device
Operation / Control Virtqueue / Session operation / Session operation: MAC session}

The fixed-length and the variable-length parameters of MAC session requests are as follows:

\begin{lstlisting}
struct virtio_crypto_mac_create_session_flf {
    /* Device read only portion */

    /* See VIRTIO_CRYPTO_MAC_* above */
    le32 algo;
    /* hash result length */
    le32 hash_result_len;
    /* length of authenticated key */
    le32 auth_key_len;
    le32 padding;
};

struct virtio_crypto_mac_create_session_vlf {
    /* Device read only portion */

    /* The authenticated key */
    u8 auth_key[auth_key_len];
};
\end{lstlisting}

The length of \field{auth_key} is specified in \field{auth_key_len} in the struct
virtio_crypto_mac_create_session_flf.


\subparagraph{Session operation: Symmetric algorithms session}\label{sec:Device Types / Crypto Device / Device
Operation / Control Virtqueue / Session operation / Session operation: Symmetric algorithms session}

The request of symmetric session could be the CIPHER algorithms request
or the chain algorithms (chaining CIPHER and HASH/MAC) request.

The fixed-length and the variable-length parameters of CIPHER session requests are as follows:

\begin{lstlisting}
struct virtio_crypto_cipher_session_flf {
    /* Device read only portion */

    /* See VIRTIO_CRYPTO_CIPHER* above */
    le32 algo;
    /* length of key */
    le32 key_len;
#define VIRTIO_CRYPTO_OP_ENCRYPT  1
#define VIRTIO_CRYPTO_OP_DECRYPT  2
    /* encryption or decryption */
    le32 op;
    le32 padding;
};

struct virtio_crypto_cipher_session_vlf {
    /* Device read only portion */

    /* The cipher key */
    u8 cipher_key[key_len];
};
\end{lstlisting}

The length of \field{cipher_key} is specified in \field{key_len} in the struct
virtio_crypto_cipher_session_flf.

The fixed-length and the variable-length parameters of Chain session requests are as follows:

\begin{lstlisting}
struct virtio_crypto_alg_chain_session_flf {
    /* Device read only portion */

#define VIRTIO_CRYPTO_SYM_ALG_CHAIN_ORDER_HASH_THEN_CIPHER  1
#define VIRTIO_CRYPTO_SYM_ALG_CHAIN_ORDER_CIPHER_THEN_HASH  2
    le32 alg_chain_order;
/* Plain hash */
#define VIRTIO_CRYPTO_SYM_HASH_MODE_PLAIN    1
/* Authenticated hash (mac) */
#define VIRTIO_CRYPTO_SYM_HASH_MODE_AUTH     2
/* Nested hash */
#define VIRTIO_CRYPTO_SYM_HASH_MODE_NESTED   3
    le32 hash_mode;
    struct virtio_crypto_cipher_session_flf cipher_hdr;

#define VIRTIO_CRYPTO_ALG_CHAIN_SESS_OP_SPEC_HDR_SIZE  16
    /* fixed length fields, algo specific */
    u8 algo_flf[VIRTIO_CRYPTO_ALG_CHAIN_SESS_OP_SPEC_HDR_SIZE];

    /* length of the additional authenticated data (AAD) in bytes */
    le32 aad_len;
    le32 padding;
};

struct virtio_crypto_alg_chain_session_vlf {
    /* Device read only portion */

    /* The cipher key */
    u8 cipher_key[key_len];
    /* The authenticated key */
    u8 auth_key[auth_key_len];
};
\end{lstlisting}

\field{hash_mode} decides the type used by \field{algo_flf}.

\field{algo_flf} is fixed to 16 bytes and MUST contains or be one of
the following types:
\begin{itemize*}
\item struct virtio_crypto_hash_create_session_flf
\item struct virtio_crypto_mac_create_session_flf
\end{itemize*}
The data of unused part (if has) in \field{algo_flf} will be ignored.

The length of \field{cipher_key} is specified in \field{key_len} in \field{cipher_hdr}.

The length of \field{auth_key} is specified in \field{auth_key_len} in struct
virtio_crypto_mac_create_session_flf.

The fixed-length parameters of Symmetric session requests are as follows:

\begin{lstlisting}
struct virtio_crypto_sym_create_session_flf {
    /* Device read only portion */

#define VIRTIO_CRYPTO_SYM_SESS_OP_SPEC_HDR_SIZE  48
    /* fixed length fields, opcode specific */
    u8 op_flf[VIRTIO_CRYPTO_SYM_SESS_OP_SPEC_HDR_SIZE];

/* No operation */
#define VIRTIO_CRYPTO_SYM_OP_NONE  0
/* Cipher only operation on the data */
#define VIRTIO_CRYPTO_SYM_OP_CIPHER  1
/* Chain any cipher with any hash or mac operation. The order
   depends on the value of alg_chain_order param */
#define VIRTIO_CRYPTO_SYM_OP_ALGORITHM_CHAINING  2
    le32 op_type;
    le32 padding;
};
\end{lstlisting}

\field{op_flf} is fixed to 48 bytes, MUST contains or be one of
the following types:
\begin{itemize*}
\item struct virtio_crypto_cipher_session_flf
\item struct virtio_crypto_alg_chain_session_flf
\end{itemize*}
The data of unused part (if has) in \field{op_flf} will be ignored.

\field{op_type} decides the type used by \field{op_flf}.

The variable-length parameters of Symmetric session requests are as follows:

\begin{lstlisting}
struct virtio_crypto_sym_create_session_vlf {
    /* Device read only portion */
    /* variable length fields, opcode specific */
    u8 op_vlf[vlf_len];
};
\end{lstlisting}

\field{op_vlf} MUST contains or be one of the following types:
\begin{itemize*}
\item struct virtio_crypto_cipher_session_vlf
\item struct virtio_crypto_alg_chain_session_vlf
\end{itemize*}

\field{op_type} in struct virtio_crypto_sym_create_session_flf decides the
type used by \field{op_vlf}.

\field{vlf_len} is the size of the specific structure used.


\subparagraph{Session operation: AEAD session}\label{sec:Device Types / Crypto Device / Device
Operation / Control Virtqueue / Session operation / Session operation: AEAD session}

The fixed-length and the variable-length parameters of AEAD session requests are as follows:

\begin{lstlisting}
struct virtio_crypto_aead_create_session_flf {
    /* Device read only portion */

    /* See VIRTIO_CRYPTO_AEAD_* above */
    le32 algo;
    /* length of key */
    le32 key_len;
    /* Authentication tag length */
    le32 tag_len;
    /* The length of the additional authenticated data (AAD) in bytes */
    le32 aad_len;
    /* encryption or decryption, See above VIRTIO_CRYPTO_OP_* */
    le32 op;
    le32 padding;
};

struct virtio_crypto_aead_create_session_vlf {
    /* Device read only portion */
    u8 key[key_len];
};
\end{lstlisting}

The length of \field{key} is specified in \field{key_len} in struct
virtio_crypto_aead_create_session_flf.

\subparagraph{Session operation: AKCIPHER session}\label{sec:Device Types / Crypto Device / Device
Operation / Control Virtqueue / Session operation / Session operation: AKCIPHER session}

Due to the complexity of asymmetric key algorithms, different algorithms
require different parameters. The following data structures are used as
supplementary parameters to describe the asymmetric algorithm sessions.

For the RSA algorithm, the extra parameters are as follows:
\begin{lstlisting}
struct virtio_crypto_rsa_session_para {
#define VIRTIO_CRYPTO_RSA_RAW_PADDING   0
#define VIRTIO_CRYPTO_RSA_PKCS1_PADDING 1
    le32 padding_algo;

#define VIRTIO_CRYPTO_RSA_NO_HASH   0
#define VIRTIO_CRYPTO_RSA_MD2       1
#define VIRTIO_CRYPTO_RSA_MD3       2
#define VIRTIO_CRYPTO_RSA_MD4       3
#define VIRTIO_CRYPTO_RSA_MD5       4
#define VIRTIO_CRYPTO_RSA_SHA1      5
#define VIRTIO_CRYPTO_RSA_SHA256    6
#define VIRTIO_CRYPTO_RSA_SHA384    7
#define VIRTIO_CRYPTO_RSA_SHA512    8
#define VIRTIO_CRYPTO_RSA_SHA224    9
    le32 hash_algo;
};
\end{lstlisting}

\field{padding_algo} specifies the padding method used by RSA sessions.
\begin{itemize*}
\item If VIRTIO_CRYPTO_RSA_RAW_PADDING is specified, 1) \field{hash_algo}
is ignored, 2) ciphertext and plaintext MUST be padded with leading zeros,
3) and RSA sessions with VIRTIO_CRYPTO_RSA_RAW_PADDING MUST not be used
for verification and signing operations.
\item If VIRTIO_CRYPTO_RSA_PKCS1_PADDING is specified, EMSA-PKCS1-v1_5 padding method
is used (see \hyperref[intro:rfc3447]{PKCS\#1}), \field{hash_algo} specifies how the
digest of the data passed to RSA sessions is calculated when verifying and signing.
It only affects the padding algorithm and is ignored during encryption and decryption.
\end{itemize*}

The ECC algorithms such as the ECDSA algorithm, cannot use custom curves, only the
following known curves can be used (see \hyperref[intro:NIST]{NIST-recommended curves}).

\begin{lstlisting}
#define VIRTIO_CRYPTO_CURVE_UNKNOWN   0
#define VIRTIO_CRYPTO_CURVE_NIST_P192 1
#define VIRTIO_CRYPTO_CURVE_NIST_P224 2
#define VIRTIO_CRYPTO_CURVE_NIST_P256 3
#define VIRTIO_CRYPTO_CURVE_NIST_P384 4
#define VIRTIO_CRYPTO_CURVE_NIST_P521 5
\end{lstlisting}

For the ECDSA algorithm, the extra parameters are as follows:
\begin{lstlisting}
struct virtio_crypto_ecdsa_session_para {
    /* See VIRTIO_CRYPTO_CURVE_* above */
    le32 curve_id;
};
\end{lstlisting}

The fixed-length and the variable-length parameters of AKCIPHER session requests are as follows:
\begin{lstlisting}
struct virtio_crypto_akcipher_create_session_flf {
    /* Device read only portion */

    /* See VIRTIO_CRYPTO_AKCIPHER_* above */
    le32 algo;
#define VIRTIO_CRYPTO_AKCIPHER_KEY_TYPE_PUBLIC 1
#define VIRTIO_CRYPTO_AKCIPHER_KEY_TYPE_PRIVATE 2
    le32 key_type;
    /* length of key */
    le32 key_len;

#define VIRTIO_CRYPTO_AKCIPHER_SESS_ALGO_SPEC_HDR_SIZE 44
    u8 algo_flf[VIRTIO_CRYPTO_AKCIPHER_SESS_ALGO_SPEC_HDR_SIZE];
};

struct virtio_crypto_akcipher_create_session_vlf {
    /* Device read only portion */
    u8 key[key_len];
};
\end{lstlisting}

\field{algo} decides the type used by \field{algo_flf}.
\field{algo_flf} is fixed to 44 bytes and MUST contains of be one the
following structures:
\begin{itemize*}
\item struct virtio_crypto_rsa_session_para
\item struct virtio_crypto_ecdsa_session_para
\end{itemize*}

The length of \field{key} is specified in \field{key_len} in the struct
virtio_crypto_akcipher_create_session_flf.

For the RSA algorithm, the key needs to be encoded according to
\hyperref[intro:rfc3447]{PKCS\#1}. The private key is described with the
RSAPrivateKey structure, and the public key is described with the RSAPublicKey
structure. These ASN.1 structures are encoded in DER encoding rules (see
\hyperref[intro:rfc6025]{rfc6025}).

\begin{lstlisting}
RSAPrivateKey ::= SEQUENCE {
    version          INTEGER,
    modulus          INTEGER,
    publicExponent   INTEGER,
    privateExponent  INTEGER,
    prime1           INTEGER,
    prime2           INTEGER,
    exponent1        INTEGER,
    exponent1        INTEGER,
    coefficient      INTEGER,
    otherPrimeInfos  OtherPrimeInfos OPTIONAL
}

OtherPrimeInfos ::= SEQUENCE SIZE(1...MAX) OF OtherPrimeInfo

OtherPrimeINfo ::= SEQUENCE {
    prime           INTEGER,
    exponent        INTEGER,
    coefficient     INTEGER
}

RSAPublicKey ::= SEQUENCE {
    modulus         INTEGER,
    publicExponent  INTEGER
}
\end{lstlisting}

For the ECDSA algorithm, the private key is encoded according to
\hyperref[intro:rfc5915]{RFC5915}, the private key of the ECDSA algorithm
is described by the ASN.1 structure ECPrivateKey and encoded with DER
encoding rules (see \hyperref[intro:rfc6025]{rfc6025}).

\begin{lstlisting}
ECPrivateKey ::= SEQUNCE {
    version         INTEGER,
    privateKey      OCTET STRING,
    parameters [0]  ECParameters {{ NamedCurve }} OPTIONAL,
    publicKey  [1]  BIT STRING OPTIONAL
}
\end{lstlisting}

The public key of the ECDSA algorithm is encoded according to \hyperref[intro:SEC1]{SEC1},
and the public key of ECDSA is described by the ASN.1 structure ECPoint.
When initializing a session with ECDSA public key, the ECPoint is DER encoded and the
\field{key} only contains the value part of ECPoint, that is, the header part of the
OCTET STRING will be omitted (see \hyperref[intro:rfc6025]{rfc6025}).

\begin{lstlisting}
ECPoint ::= OCTET STRING
\end{lstlisting}

The length of \field{key} is specified in \field{key_len} in
struct virtio_crypto_akcipher_create_session_flf.

\drivernormative{\subparagraph}{Session operation: create session}{Device Types / Crypto Device / Device
Operation / Control Virtqueue / Session operation / Session operation: create session}

\begin{itemize*}
\item The driver MUST set the \field{opcode} field based on service type: CIPHER, HASH, MAC, AEAD or AKCIPHER.
\item The driver MUST set the control general header, the opcode specific header,
    the opcode specific extra parameters and the opcode specific outcome buffer in turn.
    See \ref{sec:Device Types / Crypto Device / Device Operation / Control Virtqueue}.
\item The driver MUST set the \field{reversed} field to zero.
\end{itemize*}

\devicenormative{\subparagraph}{Session operation: create session}{Device Types / Crypto Device / Device
Operation / Control Virtqueue / Session operation / Session operation: create session}

\begin{itemize*}
\item The device MUST use the corresponding opcode specific structure according to the
    \field{opcode} in the control general header.
\item The device MUST extract extra parameters according to the structures used.
\item The device MUST set the \field{status} field to one of the following values of enum
    VIRTIO_CRYPTO_STATUS after finish a session creation:
\begin{itemize*}
\item VIRTIO_CRYPTO_OK if a session is created successfully.
\item VIRTIO_CRYPTO_NOTSUPP if the requested algorithm or operation is unsupported.
\item VIRTIO_CRYPTO_NOSPC if no free session ID (only when the VIRTIO_CRYPTO_F_REVISION_1
    feature bit is negotiated).
\item VIRTIO_CRYPTO_ERR if failure not mentioned above occurs.
\end{itemize*}
\item The device MUST set the \field{session_id} field to a unique session identifier only
    if the status is set to VIRTIO_CRYPTO_OK.
\end{itemize*}

\drivernormative{\subparagraph}{Session operation: destroy session}{Device Types / Crypto Device / Device
Operation / Control Virtqueue / Session operation / Session operation: destroy session}

\begin{itemize*}
\item The driver MUST set the \field{opcode} field based on service type: CIPHER, HASH, MAC, AEAD or AKCIPHER.
\item The driver MUST set the \field{session_id} to a valid value assigned by the device
    when the session was created.
\end{itemize*}

\devicenormative{\subparagraph}{Session operation: destroy session}{Device Types / Crypto Device / Device
Operation / Control Virtqueue / Session operation / Session operation: destroy session}

\begin{itemize*}
\item The device MUST set the \field{status} field to one of the following values of enum VIRTIO_CRYPTO_STATUS.
\begin{itemize*}
\item VIRTIO_CRYPTO_OK if a session is created successfully.
\item VIRTIO_CRYPTO_ERR if any failure occurs.
\end{itemize*}
\end{itemize*}


\subsubsection{Data Virtqueue}\label{sec:Device Types / Crypto Device / Device Operation / Data Virtqueue}

The driver uses the data virtqueues to transmit crypto operation requests to the device,
and completes the crypto operations.

The header for dataq is as follows:

\begin{lstlisting}
struct virtio_crypto_op_header {
#define VIRTIO_CRYPTO_CIPHER_ENCRYPT \
    VIRTIO_CRYPTO_OPCODE(VIRTIO_CRYPTO_SERVICE_CIPHER, 0x00)
#define VIRTIO_CRYPTO_CIPHER_DECRYPT \
    VIRTIO_CRYPTO_OPCODE(VIRTIO_CRYPTO_SERVICE_CIPHER, 0x01)
#define VIRTIO_CRYPTO_HASH \
    VIRTIO_CRYPTO_OPCODE(VIRTIO_CRYPTO_SERVICE_HASH, 0x00)
#define VIRTIO_CRYPTO_MAC \
    VIRTIO_CRYPTO_OPCODE(VIRTIO_CRYPTO_SERVICE_MAC, 0x00)
#define VIRTIO_CRYPTO_AEAD_ENCRYPT \
    VIRTIO_CRYPTO_OPCODE(VIRTIO_CRYPTO_SERVICE_AEAD, 0x00)
#define VIRTIO_CRYPTO_AEAD_DECRYPT \
    VIRTIO_CRYPTO_OPCODE(VIRTIO_CRYPTO_SERVICE_AEAD, 0x01)
#define VIRTIO_CRYPTO_AKCIPHER_ENCRYPT \
    VIRTIO_CRYPTO_OPCODE(VIRTIO_CRYPTO_SERVICE_AKCIPHER, 0x00)
#define VIRTIO_CRYPTO_AKCIPHER_DECRYPT \
    VIRTIO_CRYPTO_OPCODE(VIRTIO_CRYPTO_SERVICE_AKCIPHER, 0x01)
#define VIRTIO_CRYPTO_AKCIPHER_SIGN \
    VIRTIO_CRYPTO_OPCODE(VIRTIO_CRYPTO_SERVICE_AKCIPHER, 0x02)
#define VIRTIO_CRYPTO_AKCIPHER_VERIFY \
    VIRTIO_CRYPTO_OPCODE(VIRTIO_CRYPTO_SERVICE_AKCIPHER, 0x03)
    le32 opcode;
    /* algo should be service-specific algorithms */
    le32 algo;
    le64 session_id;
#define VIRTIO_CRYPTO_FLAG_SESSION_MODE 1
    /* control flag to control the request */
    le32 flag;
    le32 padding;
};
\end{lstlisting}

\begin{note}
If VIRTIO_CRYPTO_F_REVISION_1 is not negotiated the \field{flag} is ignored.

If VIRTIO_CRYPTO_F_REVISION_1 is negotiated but VIRTIO_CRYPTO_F_<SERVICE>_STATELESS_MODE
is not negotiated, then the device SHOULD reject <SERVICE> requests if
VIRTIO_CRYPTO_FLAG_SESSION_MODE is not set (in \field{flag}).
\end{note}

The dataq request is composed of four parts:
\begin{lstlisting}
struct virtio_crypto_op_data_req {
    /* Device read only portion */

    struct virtio_crypto_op_header header;

#define VIRTIO_CRYPTO_DATAQ_OP_SPEC_HDR_LEGACY 48
    /* fixed length fields, opcode specific */
    u8 op_flf[flf_len];

    /* Device read && write portion */
    /* variable length fields, opcode specific */
    u8 op_vlf[vlf_len];

    /* Device write only portion */
    struct virtio_crypto_inhdr inhdr;
};
\end{lstlisting}

\field{header} is a general header (see above).

\field{op_flf} is the opcode (in \field{header}) specific header.

\field{flf_len} depends on the VIRTIO_CRYPTO_F_REVISION_1 feature bit
(see below).

\field{op_vlf} is the opcode (in \field{header}) specific parameters.

\field{vlf_len} is the size of the specific structure used.

\begin{itemize*}
\item If the the opcode (in \field{header}) is VIRTIO_CRYPTO_CIPHER_ENCRYPT
    or VIRTIO_CRYPTO_CIPHER_DECRYPT then:
    \begin{itemize*}
    \item If VIRTIO_CRYPTO_F_CIPHER_STATELESS_MODE is negotiated, \field{op_flf} is
        struct virtio_crypto_sym_data_flf_stateless, and \field{op_vlf} is struct
        virtio_crypto_sym_data_vlf_stateless.
    \item If VIRTIO_CRYPTO_F_CIPHER_STATELESS_MODE is NOT negotiated, \field{op_flf}
        is struct virtio_crypto_sym_data_flf if VIRTIO_CRYPTO_F_REVISION_1 is negotiated
        and struct virtio_crypto_sym_data_flf is padded to 48 bytes if NOT negotiated,
        and \field{op_vlf} is struct virtio_crypto_sym_data_vlf.
    \end{itemize*}
\item If the the opcode (in \field{header}) is VIRTIO_CRYPTO_HASH:
    \begin{itemize*}
    \item If VIRTIO_CRYPTO_F_HASH_STATELESS_MODE is negotiated, \field{op_flf} is
        struct virtio_crypto_hash_data_flf_stateless, and \field{op_vlf} is struct
        virtio_crypto_hash_data_vlf_stateless.
    \item If VIRTIO_CRYPTO_F_HASH_STATELESS_MODE is NOT negotiated, \field{op_flf}
        is struct virtio_crypto_hash_data_flf if VIRTIO_CRYPTO_F_REVISION_1 is negotiated
        and struct virtio_crypto_hash_data_flf is padded to 48 bytes if NOT negotiated,
        and \field{op_vlf} is struct virtio_crypto_hash_data_vlf.
    \end{itemize*}
\item If the the opcode (in \field{header}) is VIRTIO_CRYPTO_MAC:
    \begin{itemize*}
    \item If VIRTIO_CRYPTO_F_MAC_STATELESS_MODE is negotiated, \field{op_flf} is
        struct virtio_crypto_mac_data_flf_stateless, and \field{op_vlf} is struct
        virtio_crypto_mac_data_vlf_stateless.
    \item If VIRTIO_CRYPTO_F_MAC_STATELESS_MODE is NOT negotiated, \field{op_flf}
        is struct virtio_crypto_mac_data_flf if VIRTIO_CRYPTO_F_REVISION_1 is negotiated
        and struct virtio_crypto_mac_data_flf is padded to 48 bytes if NOT negotiated,
        and \field{op_vlf} is struct virtio_crypto_mac_data_vlf.
    \end{itemize*}
\item If the the opcode (in \field{header}) is VIRTIO_CRYPTO_AEAD_ENCRYPT
    or VIRTIO_CRYPTO_AEAD_DECRYPT then:
    \begin{itemize*}
    \item If VIRTIO_CRYPTO_F_AEAD_STATELESS_MODE is negotiated, \field{op_flf} is
        struct virtio_crypto_aead_data_flf_stateless, and \field{op_vlf} is struct
        virtio_crypto_aead_data_vlf_stateless.
    \item If VIRTIO_CRYPTO_F_AEAD_STATELESS_MODE is NOT negotiated, \field{op_flf}
        is struct virtio_crypto_aead_data_flf if VIRTIO_CRYPTO_F_REVISION_1 is negotiated
        and struct virtio_crypto_aead_data_flf is padded to 48 bytes if NOT negotiated,
        and \field{op_vlf} is struct virtio_crypto_aead_data_vlf.
    \end{itemize*}
\item If the opcode (in \field{header}) is VIRTIO_CRYPTO_AKCIPHER_ENCRYPT, VIRTIO_CRYPTO_AKCIPHER_DECRYPT,
    VIRTIO_CRYPTO_AKCIPHER_SIGN or VIRTIO_CRYPTO_AKCIPHER_VERIFY then:
    \begin{itemize*}
    \item If VIRTIO_CRYPTO_F_AKCIPHER_STATELESS_MODE is negotiated, \field{op_flf} is
        struct virtio_crypto_akcipher_data_flf_statless, and \field{op_vlf} is struct
        virtio_crypto_akcipher_data_vlf_stateless.
    \item If VIRTIO_CRYPTO_F_AKCIPHER_STATELESS_MODE is NOT negotiated, \field{op_flf}
        is struct virtio_crypto_akcipher_data_flf if VIRTIO_CRYPTO_F_REVISION_1 is negotiated
        and struct virtio_crypto_akcipher_data_flf is padded to 48 bytes if NOT negotiated,
        and \field{op_vlf} is struct virtio_crypto_akcipher_data_vlf.
    \end{itemize*}
\end{itemize*}

\field{inhdr} is a unified input header that used to return the status of
the operations, is defined as follows:

\begin{lstlisting}
struct virtio_crypto_inhdr {
    u8 status;
};
\end{lstlisting}

\subsubsection{HASH Service Operation}\label{sec:Device Types / Crypto Device / Device Operation / HASH Service Operation}

Session mode HASH service requests are as follows:

\begin{lstlisting}
struct virtio_crypto_hash_data_flf {
    /* length of source data */
    le32 src_data_len;
    /* hash result length */
    le32 hash_result_len;
};

struct virtio_crypto_hash_data_vlf {
    /* Device read only portion */
    /* Source data */
    u8 src_data[src_data_len];

    /* Device write only portion */
    /* Hash result data */
    u8 hash_result[hash_result_len];
};
\end{lstlisting}

Each data request uses the virtio_crypto_hash_data_flf structure and the
virtio_crypto_hash_data_vlf structure to store information used to run the
HASH operations.

\field{src_data} is the source data that will be processed.
\field{src_data_len} is the length of source data.
\field{hash_result} is the result data and \field{hash_result_len} is the length
of it.

Stateless mode HASH service requests are as follows:

\begin{lstlisting}
struct virtio_crypto_hash_data_flf_stateless {
    struct {
        /* See VIRTIO_CRYPTO_HASH_* above */
        le32 algo;
    } sess_para;

    /* length of source data */
    le32 src_data_len;
    /* hash result length */
    le32 hash_result_len;
    le32 reserved;
};
struct virtio_crypto_hash_data_vlf_stateless {
    /* Device read only portion */
    /* Source data */
    u8 src_data[src_data_len];

    /* Device write only portion */
    /* Hash result data */
    u8 hash_result[hash_result_len];
};
\end{lstlisting}

\drivernormative{\paragraph}{HASH Service Operation}{Device Types / Crypto Device / Device Operation / HASH Service Operation}

\begin{itemize*}
\item If the driver uses the session mode, then the driver MUST set \field{session_id}
    in struct virtio_crypto_op_header to a valid value assigned by the device when the
    session was created.
\item If the VIRTIO_CRYPTO_F_HASH_STATELESS_MODE feature bit is negotiated, 1) if the
    driver uses the stateless mode, then the driver MUST set the \field{flag} field in
    struct virtio_crypto_op_header to ZERO and MUST set the fields in struct
    virtio_crypto_hash_data_flf_stateless.sess_para, 2) if the driver uses the session
    mode, then the driver MUST set the \field{flag} field in struct virtio_crypto_op_header
    to VIRTIO_CRYPTO_FLAG_SESSION_MODE.
\item The driver MUST set \field{opcode} in struct virtio_crypto_op_header to VIRTIO_CRYPTO_HASH.
\end{itemize*}

\devicenormative{\paragraph}{HASH Service Operation}{Device Types / Crypto Device / Device Operation / HASH Service Operation}

\begin{itemize*}
\item The device MUST use the corresponding structure according to the \field{opcode}
    in the data general header.
\item If the VIRTIO_CRYPTO_F_HASH_STATELESS_MODE feature bit is negotiated, the device
    MUST parse \field{flag} field in struct virtio_crypto_op_header in order to decide
    which mode the driver uses.
\item The device MUST copy the results of HASH operations in the hash_result[] if HASH
    operations success.
\item The device MUST set \field{status} in struct virtio_crypto_inhdr to one of the
    following values of enum VIRTIO_CRYPTO_STATUS:
\begin{itemize*}
\item VIRTIO_CRYPTO_OK if the operation success.
\item VIRTIO_CRYPTO_NOTSUPP if the requested algorithm or operation is unsupported.
\item VIRTIO_CRYPTO_INVSESS if the session ID invalid when in session mode.
\item VIRTIO_CRYPTO_ERR if any failure not mentioned above occurs.
\end{itemize*}
\end{itemize*}


\subsubsection{MAC Service Operation}\label{sec:Device Types / Crypto Device / Device Operation / MAC Service Operation}

Session mode MAC service requests are as follows:

\begin{lstlisting}
struct virtio_crypto_mac_data_flf {
    struct virtio_crypto_hash_data_flf hdr;
};

struct virtio_crypto_mac_data_vlf {
    /* Device read only portion */
    /* Source data */
    u8 src_data[src_data_len];

    /* Device write only portion */
    /* Hash result data */
    u8 hash_result[hash_result_len];
};
\end{lstlisting}

Each request uses the virtio_crypto_mac_data_flf structure and the
virtio_crypto_mac_data_vlf structure to store information used to run the
MAC operations.

\field{src_data} is the source data that will be processed.
\field{src_data_len} is the length of source data.
\field{hash_result} is the result data and \field{hash_result_len} is the length
of it.

Stateless mode MAC service requests are as follows:

\begin{lstlisting}
struct virtio_crypto_mac_data_flf_stateless {
    struct {
        /* See VIRTIO_CRYPTO_MAC_* above */
        le32 algo;
        /* length of authenticated key */
        le32 auth_key_len;
    } sess_para;

    /* length of source data */
    le32 src_data_len;
    /* hash result length */
    le32 hash_result_len;
};

struct virtio_crypto_mac_data_vlf_stateless {
    /* Device read only portion */
    /* The authenticated key */
    u8 auth_key[auth_key_len];
    /* Source data */
    u8 src_data[src_data_len];

    /* Device write only portion */
    /* Hash result data */
    u8 hash_result[hash_result_len];
};
\end{lstlisting}

\field{auth_key} is the authenticated key that will be used during the process.
\field{auth_key_len} is the length of the key.

\drivernormative{\paragraph}{MAC Service Operation}{Device Types / Crypto Device / Device Operation / MAC Service Operation}

\begin{itemize*}
\item If the driver uses the session mode, then the driver MUST set \field{session_id}
    in struct virtio_crypto_op_header to a valid value assigned by the device when the
    session was created.
\item If the VIRTIO_CRYPTO_F_MAC_STATELESS_MODE feature bit is negotiated, 1) if the
    driver uses the stateless mode, then the driver MUST set the \field{flag} field
    in struct virtio_crypto_op_header to ZERO and MUST set the fields in struct
    virtio_crypto_mac_data_flf_stateless.sess_para, 2) if the driver uses the session
    mode, then the driver MUST set the \field{flag} field in struct virtio_crypto_op_header
    to VIRTIO_CRYPTO_FLAG_SESSION_MODE.
\item The driver MUST set \field{opcode} in struct virtio_crypto_op_header to VIRTIO_CRYPTO_MAC.
\end{itemize*}

\devicenormative{\paragraph}{MAC Service Operation}{Device Types / Crypto Device / Device Operation / MAC Service Operation}

\begin{itemize*}
\item If the VIRTIO_CRYPTO_F_MAC_STATELESS_MODE feature bit is negotiated, the device
    MUST parse \field{flag} field in struct virtio_crypto_op_header in order to decide
	which mode the driver uses.
\item The device MUST copy the results of MAC operations in the hash_result[] if HASH
    operations success.
\item The device MUST set \field{status} in struct virtio_crypto_inhdr to one of the
    following values of enum VIRTIO_CRYPTO_STATUS:
\begin{itemize*}
\item VIRTIO_CRYPTO_OK if the operation success.
\item VIRTIO_CRYPTO_NOTSUPP if the requested algorithm or operation is unsupported.
\item VIRTIO_CRYPTO_INVSESS if the session ID invalid when in session mode.
\item VIRTIO_CRYPTO_ERR if any failure not mentioned above occurs.
\end{itemize*}
\end{itemize*}

\subsubsection{Symmetric algorithms Operation}\label{sec:Device Types / Crypto Device / Device Operation / Symmetric algorithms Operation}

Session mode CIPHER service requests are as follows:

\begin{lstlisting}
struct virtio_crypto_cipher_data_flf {
    /*
     * Byte Length of valid IV/Counter data pointed to by the below iv data.
     *
     * For block ciphers in CBC or F8 mode, or for Kasumi in F8 mode, or for
     *   SNOW3G in UEA2 mode, this is the length of the IV (which
     *   must be the same as the block length of the cipher).
     * For block ciphers in CTR mode, this is the length of the counter
     *   (which must be the same as the block length of the cipher).
     */
    le32 iv_len;
    /* length of source data */
    le32 src_data_len;
    /* length of destination data */
    le32 dst_data_len;
    le32 padding;
};

struct virtio_crypto_cipher_data_vlf {
    /* Device read only portion */

    /*
     * Initialization Vector or Counter data.
     *
     * For block ciphers in CBC or F8 mode, or for Kasumi in F8 mode, or for
     *   SNOW3G in UEA2 mode, this is the Initialization Vector (IV)
     *   value.
     * For block ciphers in CTR mode, this is the counter.
     * For AES-XTS, this is the 128bit tweak, i, from IEEE Std 1619-2007.
     *
     * The IV/Counter will be updated after every partial cryptographic
     * operation.
     */
    u8 iv[iv_len];
    /* Source data */
    u8 src_data[src_data_len];

    /* Device write only portion */
    /* Destination data */
    u8 dst_data[dst_data_len];
};
\end{lstlisting}

Session mode requests of algorithm chaining are as follows:

\begin{lstlisting}
struct virtio_crypto_alg_chain_data_flf {
    le32 iv_len;
    /* Length of source data */
    le32 src_data_len;
    /* Length of destination data */
    le32 dst_data_len;
    /* Starting point for cipher processing in source data */
    le32 cipher_start_src_offset;
    /* Length of the source data that the cipher will be computed on */
    le32 len_to_cipher;
    /* Starting point for hash processing in source data */
    le32 hash_start_src_offset;
    /* Length of the source data that the hash will be computed on */
    le32 len_to_hash;
    /* Length of the additional auth data */
    le32 aad_len;
    /* Length of the hash result */
    le32 hash_result_len;
    le32 reserved;
};

struct virtio_crypto_alg_chain_data_vlf {
    /* Device read only portion */

    /* Initialization Vector or Counter data */
    u8 iv[iv_len];
    /* Source data */
    u8 src_data[src_data_len];
    /* Additional authenticated data if exists */
    u8 aad[aad_len];

    /* Device write only portion */

    /* Destination data */
    u8 dst_data[dst_data_len];
    /* Hash result data */
    u8 hash_result[hash_result_len];
};
\end{lstlisting}

Session mode requests of symmetric algorithm are as follows:

\begin{lstlisting}
struct virtio_crypto_sym_data_flf {
    /* Device read only portion */

#define VIRTIO_CRYPTO_SYM_DATA_REQ_HDR_SIZE    40
    u8 op_type_flf[VIRTIO_CRYPTO_SYM_DATA_REQ_HDR_SIZE];

    /* See above VIRTIO_CRYPTO_SYM_OP_* */
    le32 op_type;
    le32 padding;
};

struct virtio_crypto_sym_data_vlf {
    u8 op_type_vlf[sym_para_len];
};
\end{lstlisting}

Each request uses the virtio_crypto_sym_data_flf structure and the
virtio_crypto_sym_data_flf structure to store information used to run the
CIPHER operations.

\field{op_type_flf} is the \field{op_type} specific header, it MUST starts
with or be one of the following structures:
\begin{itemize*}
\item struct virtio_crypto_cipher_data_flf
\item struct virtio_crypto_alg_chain_data_flf
\end{itemize*}

The length of \field{op_type_flf} is fixed to 40 bytes, the data of unused
part (if has) will be ingored.

\field{op_type_vlf} is the \field{op_type} specific parameters, it MUST starts
with or be one of the following structures:
\begin{itemize*}
\item struct virtio_crypto_cipher_data_vlf
\item struct virtio_crypto_alg_chain_data_vlf
\end{itemize*}

\field{sym_para_len} is the size of the specific structure used.

Stateless mode CIPHER service requests are as follows:

\begin{lstlisting}
struct virtio_crypto_cipher_data_flf_stateless {
    struct {
        /* See VIRTIO_CRYPTO_CIPHER* above */
        le32 algo;
        /* length of key */
        le32 key_len;

        /* See VIRTIO_CRYPTO_OP_* above */
        le32 op;
    } sess_para;

    /*
     * Byte Length of valid IV/Counter data pointed to by the below iv data.
     */
    le32 iv_len;
    /* length of source data */
    le32 src_data_len;
    /* length of destination data */
    le32 dst_data_len;
};

struct virtio_crypto_cipher_data_vlf_stateless {
    /* Device read only portion */

    /* The cipher key */
    u8 cipher_key[key_len];

    /* Initialization Vector or Counter data. */
    u8 iv[iv_len];
    /* Source data */
    u8 src_data[src_data_len];

    /* Device write only portion */
    /* Destination data */
    u8 dst_data[dst_data_len];
};
\end{lstlisting}

Stateless mode requests of algorithm chaining are as follows:

\begin{lstlisting}
struct virtio_crypto_alg_chain_data_flf_stateless {
    struct {
        /* See VIRTIO_CRYPTO_SYM_ALG_CHAIN_ORDER_* above */
        le32 alg_chain_order;
        /* length of the additional authenticated data in bytes */
        le32 aad_len;

        struct {
            /* See VIRTIO_CRYPTO_CIPHER* above */
            le32 algo;
            /* length of key */
            le32 key_len;
            /* See VIRTIO_CRYPTO_OP_* above */
            le32 op;
        } cipher;

        struct {
            /* See VIRTIO_CRYPTO_HASH_* or VIRTIO_CRYPTO_MAC_* above */
            le32 algo;
            /* length of authenticated key */
            le32 auth_key_len;
            /* See VIRTIO_CRYPTO_SYM_HASH_MODE_* above */
            le32 hash_mode;
        } hash;
    } sess_para;

    le32 iv_len;
    /* Length of source data */
    le32 src_data_len;
    /* Length of destination data */
    le32 dst_data_len;
    /* Starting point for cipher processing in source data */
    le32 cipher_start_src_offset;
    /* Length of the source data that the cipher will be computed on */
    le32 len_to_cipher;
    /* Starting point for hash processing in source data */
    le32 hash_start_src_offset;
    /* Length of the source data that the hash will be computed on */
    le32 len_to_hash;
    /* Length of the additional auth data */
    le32 aad_len;
    /* Length of the hash result */
    le32 hash_result_len;
    le32 reserved;
};

struct virtio_crypto_alg_chain_data_vlf_stateless {
    /* Device read only portion */

    /* The cipher key */
    u8 cipher_key[key_len];
    /* The auth key */
    u8 auth_key[auth_key_len];
    /* Initialization Vector or Counter data */
    u8 iv[iv_len];
    /* Additional authenticated data if exists */
    u8 aad[aad_len];
    /* Source data */
    u8 src_data[src_data_len];

    /* Device write only portion */

    /* Destination data */
    u8 dst_data[dst_data_len];
    /* Hash result data */
    u8 hash_result[hash_result_len];
};
\end{lstlisting}

Stateless mode requests of symmetric algorithm are as follows:

\begin{lstlisting}
struct virtio_crypto_sym_data_flf_stateless {
    /* Device read only portion */
#define VIRTIO_CRYPTO_SYM_DATE_REQ_HDR_STATELESS_SIZE    72
    u8 op_type_flf[VIRTIO_CRYPTO_SYM_DATE_REQ_HDR_STATELESS_SIZE];

    /* Device write only portion */
    /* See above VIRTIO_CRYPTO_SYM_OP_* */
    le32 op_type;
};

struct virtio_crypto_sym_data_vlf_stateless {
    u8 op_type_vlf[sym_para_len];
};
\end{lstlisting}

\field{op_type_flf} is the \field{op_type} specific header, it MUST starts
with or be one of the following structures:
\begin{itemize*}
\item struct virtio_crypto_cipher_data_flf_stateless
\item struct virtio_crypto_alg_chain_data_flf_stateless
\end{itemize*}

The length of \field{op_type_flf} is fixed to 72 bytes, the data of unused
part (if has) will be ingored.

\field{op_type_vlf} is the \field{op_type} specific parameters, it MUST starts
with or be one of the following structures:
\begin{itemize*}
\item struct virtio_crypto_cipher_data_vlf_stateless
\item struct virtio_crypto_alg_chain_data_vlf_stateless
\end{itemize*}

\field{sym_para_len} is the size of the specific structure used.

\drivernormative{\paragraph}{Symmetric algorithms Operation}{Device Types / Crypto Device / Device Operation / Symmetric algorithms Operation}

\begin{itemize*}
\item If the driver uses the session mode, then the driver MUST set \field{session_id}
    in struct virtio_crypto_op_header to a valid value assigned by the device when the
    session was created.
\item If the VIRTIO_CRYPTO_F_CIPHER_STATELESS_MODE feature bit is negotiated, 1) if the
    driver uses the stateless mode, then the driver MUST set the \field{flag} field in
    struct virtio_crypto_op_header to ZERO and MUST set the fields in struct
    virtio_crypto_cipher_data_flf_stateless.sess_para or struct
    virtio_crypto_alg_chain_data_flf_stateless.sess_para, 2) if the driver uses the
    session mode, then the driver MUST set the \field{flag} field in struct
    virtio_crypto_op_header to VIRTIO_CRYPTO_FLAG_SESSION_MODE.
\item The driver MUST set the \field{opcode} field in struct virtio_crypto_op_header
    to VIRTIO_CRYPTO_CIPHER_ENCRYPT or VIRTIO_CRYPTO_CIPHER_DECRYPT.
\item The driver MUST specify the fields of struct virtio_crypto_cipher_data_flf in
    struct virtio_crypto_sym_data_flf and struct virtio_crypto_cipher_data_vlf in
    struct virtio_crypto_sym_data_vlf if the request is based on VIRTIO_CRYPTO_SYM_OP_CIPHER.
\item The driver MUST specify the fields of struct virtio_crypto_alg_chain_data_flf
    in struct virtio_crypto_sym_data_flf and struct virtio_crypto_alg_chain_data_vlf
    in struct virtio_crypto_sym_data_vlf if the request is of the VIRTIO_CRYPTO_SYM_OP_ALGORITHM_CHAINING
    type.
\end{itemize*}

\devicenormative{\paragraph}{Symmetric algorithms Operation}{Device Types / Crypto Device / Device Operation / Symmetric algorithms Operation}

\begin{itemize*}
\item If the VIRTIO_CRYPTO_F_CIPHER_STATELESS_MODE feature bit is negotiated, the device
    MUST parse \field{flag} field in struct virtio_crypto_op_header in order to decide
	which mode the driver uses.
\item The device MUST parse the virtio_crypto_sym_data_req based on the \field{opcode}
    field in general header.
\item The device MUST parse the fields of struct virtio_crypto_cipher_data_flf in
    struct virtio_crypto_sym_data_flf and struct virtio_crypto_cipher_data_vlf in
    struct virtio_crypto_sym_data_vlf if the request is based on VIRTIO_CRYPTO_SYM_OP_CIPHER.
\item The device MUST parse the fields of struct virtio_crypto_alg_chain_data_flf
    in struct virtio_crypto_sym_data_flf and struct virtio_crypto_alg_chain_data_vlf
    in struct virtio_crypto_sym_data_vlf if the request is of the VIRTIO_CRYPTO_SYM_OP_ALGORITHM_CHAINING
    type.
\item The device MUST copy the result of cryptographic operation in the dst_data[] in
    both plain CIPHER mode and algorithms chain mode.
\item The device MUST check the \field{para}.\field{add_len} is bigger than 0 before
    parse the additional authenticated data in plain algorithms chain mode.
\item The device MUST copy the result of HASH/MAC operation in the hash_result[] is
    of the VIRTIO_CRYPTO_SYM_OP_ALGORITHM_CHAINING type.
\item The device MUST set the \field{status} field in struct virtio_crypto_inhdr to
    one of the following values of enum VIRTIO_CRYPTO_STATUS:
\begin{itemize*}
\item VIRTIO_CRYPTO_OK if the operation success.
\item VIRTIO_CRYPTO_NOTSUPP if the requested algorithm or operation is unsupported.
\item VIRTIO_CRYPTO_INVSESS if the session ID is invalid in session mode.
\item VIRTIO_CRYPTO_ERR if failure not mentioned above occurs.
\end{itemize*}
\end{itemize*}

\subsubsection{AEAD Service Operation}\label{sec:Device Types / Crypto Device / Device Operation / AEAD Service Operation}

Session mode requests of symmetric algorithm are as follows:

\begin{lstlisting}
struct virtio_crypto_aead_data_flf {
    /*
     * Byte Length of valid IV data.
     *
     * For GCM mode, this is either 12 (for 96-bit IVs) or 16, in which
     *   case iv points to J0.
     * For CCM mode, this is the length of the nonce, which can be in the
     *   range 7 to 13 inclusive.
     */
    le32 iv_len;
    /* length of additional auth data */
    le32 aad_len;
    /* length of source data */
    le32 src_data_len;
    /* length of dst data, this should be at least src_data_len + tag_len */
    le32 dst_data_len;
    /* Authentication tag length */
    le32 tag_len;
    le32 reserved;
};

struct virtio_crypto_aead_data_vlf {
    /* Device read only portion */

    /*
     * Initialization Vector data.
     *
     * For GCM mode, this is either the IV (if the length is 96 bits) or J0
     *   (for other sizes), where J0 is as defined by NIST SP800-38D.
     *   Regardless of the IV length, a full 16 bytes needs to be allocated.
     * For CCM mode, the first byte is reserved, and the nonce should be
     *   written starting at &iv[1] (to allow space for the implementation
     *   to write in the flags in the first byte).  Note that a full 16 bytes
     *   should be allocated, even though the iv_len field will have
     *   a value less than this.
     *
     * The IV will be updated after every partial cryptographic operation.
     */
    u8 iv[iv_len];
    /* Source data */
    u8 src_data[src_data_len];
    /* Additional authenticated data if exists */
    u8 aad[aad_len];

    /* Device write only portion */
    /* Pointer to output data */
    u8 dst_data[dst_data_len];
};
\end{lstlisting}

Each request uses the virtio_crypto_aead_data_flf structure and the
virtio_crypto_aead_data_flf structure to store information used to run the
AEAD operations.

Stateless mode AEAD service requests are as follows:

\begin{lstlisting}
struct virtio_crypto_aead_data_flf_stateless {
    struct {
        /* See VIRTIO_CRYPTO_AEAD_* above */
        le32 algo;
        /* length of key */
        le32 key_len;
        /* encrypt or decrypt, See above VIRTIO_CRYPTO_OP_* */
        le32 op;
    } sess_para;

    /* Byte Length of valid IV data. */
    le32 iv_len;
    /* Authentication tag length */
    le32 tag_len;
    /* length of additional auth data */
    le32 aad_len;
    /* length of source data */
    le32 src_data_len;
    /* length of dst data, this should be at least src_data_len + tag_len */
    le32 dst_data_len;
};

struct virtio_crypto_aead_data_vlf_stateless {
    /* Device read only portion */

    /* The cipher key */
    u8 key[key_len];
    /* Initialization Vector data. */
    u8 iv[iv_len];
    /* Source data */
    u8 src_data[src_data_len];
    /* Additional authenticated data if exists */
    u8 aad[aad_len];

    /* Device write only portion */
    /* Pointer to output data */
    u8 dst_data[dst_data_len];
};
\end{lstlisting}

\drivernormative{\paragraph}{AEAD Service Operation}{Device Types / Crypto Device / Device Operation / AEAD Service Operation}

\begin{itemize*}
\item If the driver uses the session mode, then the driver MUST set
    \field{session_id} in struct virtio_crypto_op_header to a valid value assigned
    by the device when the session was created.
\item If the VIRTIO_CRYPTO_F_AEAD_STATELESS_MODE feature bit is negotiated, 1) if
    the driver uses the stateless mode, then the driver MUST set the \field{flag}
    field in struct virtio_crypto_op_header to ZERO and MUST set the fields in
    struct virtio_crypto_aead_data_flf_stateless.sess_para, 2) if the driver uses
    the session mode, then the driver MUST set the \field{flag} field in struct
    virtio_crypto_op_header to VIRTIO_CRYPTO_FLAG_SESSION_MODE.
\item The driver MUST set the \field{opcode} field in struct virtio_crypto_op_header
    to VIRTIO_CRYPTO_AEAD_ENCRYPT or VIRTIO_CRYPTO_AEAD_DECRYPT.
\end{itemize*}

\devicenormative{\paragraph}{AEAD Service Operation}{Device Types / Crypto Device / Device Operation / AEAD Service Operation}

\begin{itemize*}
\item If the VIRTIO_CRYPTO_F_AEAD_STATELESS_MODE feature bit is negotiated, the
    device MUST parse the virtio_crypto_aead_data_vlf_stateless based on the \field{opcode}
	field in general header.
\item The device MUST copy the result of cryptographic operation in the dst_data[].
\item The device MUST copy the authentication tag in the dst_data[] offset the cipher result.
\item The device MUST set the \field{status} field in struct virtio_crypto_inhdr to
    one of the following values of enum VIRTIO_CRYPTO_STATUS:
\item When the \field{opcode} field is VIRTIO_CRYPTO_AEAD_DECRYPT, the device MUST
    verify and return the verification result to the driver.
\begin{itemize*}
\item VIRTIO_CRYPTO_OK if the operation success.
\item VIRTIO_CRYPTO_NOTSUPP if the requested algorithm or operation is unsupported.
\item VIRTIO_CRYPTO_BADMSG if the verification result is incorrect.
\item VIRTIO_CRYPTO_INVSESS if the session ID invalid when in session mode.
\item VIRTIO_CRYPTO_ERR if any failure not mentioned above occurs.
\end{itemize*}
\end{itemize*}

\subsubsection{AKCIPHER Service Operation}\label{sec:Device Types / Crypto Device / Device Operation / AKCIPHER Service Operation}

Session mode AKCIPHER requests are as follows:

\begin{lstlisting}
struct virtio_crypto_akcipher_data_flf {
    /* length of source data */
    le32 src_data_len;
    /* length of dst data */
    le32 dst_data_len;
};

struct virtio_crypto_akcipher_data_vlf {
    /* Device read only portion */
    /* Source data */
    u8 src_data[src_data_len];

    /* Device write only portion */
    /* Pointer to output data */
    u8 dst_data[dst_data_len];
};
\end{lstlisting}

Each data request uses the virtio_crypto_akcipher_flf structure and the virtio_crypto_akcipher_data_vlf
structure to store information used to run the AKCIPHER operations.

For encryption, decryption, and signing:
\field{src_data} is the source data that will be processed, note that for signing operations,
src_data stores the data to be signed, which usually is the digest of some data rather than the
data itself.
\field{src_data_len} is the length of source data.
\field{dst_result} is the result data and \field{dst_data_len} is the length of it. Note that the
length of the result is not always exactly equal to dst_data_len, the driver needs to check how
many bytes the device has written and calculate the actual length of the result.

For verification:
\field{src_data_len} refers to the length of the signature, and \field{dst_data_len} refers to
the length of signed data, where the signed data is usually the digest of some data.
\field{src_data} is spliced by the signature and the signed data, the src_data with the lower
address stores the signature, and the higher address stores the signed data.
\field{dst_data} is always empty for verification.

Different algorithms have different signature formats.
For the RSA algorithm, the result is determined by the padding algorithm specified by
\field{padding_algo} in structure virtio_crypto_rsa_session_para.

For the ECDSA algorithm, the signature is composed of the following
ASN.1 structure (see \hyperref[intro:rfc3279]{RFC3279})
and MUST be DER encoded (see \hyperref[intro:rfc6025]{rfc6025}).

\begin{lstlisting}
Ecdsa-Sig-Value ::= SEQUENCE {
    r INTEGER,
    s INTEGER
}
\end{lstlisting}

Stateless mode AKCIPHER service requests are as follows:
\begin{lstlisting}
struct virtio_crypto_akcipher_data_flf_stateless {
    struct {
        /* See VIRTIO_CYRPTO_AKCIPHER* above */
        le32 algo;
        /* See VIRTIO_CRYPTO_AKCIPHER_KEY_TYPE_* above */
        le32 key_type;
        /* length of key */
        le32 key_len;

        /* algothrim specific parameters described above */
        union {
            struct virtio_crypto_rsa_session_para rsa;
            struct virtio_crypto_ecdsa_session_para ecdsa;
        } u;
    } sess_para;

    /* length of source data */
    le32 src_data_len;
    /* length of destination data */
    le32 dst_data_len;
};

struct virtio_crypto_akcipher_data_vlf_stateless {
    /* Device read only portion */
    u8 akcipher_key[key_len];

    /* Source data */
    u8 src_data[src_data_len];

    /* Device write only portion */
    u8 dst_data[dst_data_len];
};
\end{lstlisting}

In stateless mode, the format of key and signature, the meaning of src_data and dst_data, are all the same
with session mode.

\drivernormative{\paragraph}{AKCIPHER Service Operation}{Device Types / Crypto Device / Device Operation / AKCIPHER Service Operation}

\begin{itemize*}
\item If the driver uses the session mode, then the driver MUST set
    \field{session_id} in struct virtio_crypto_op_header to a valid
    value assigned by the device when the session was created.
\item If the VIRTIO_CRYPTO_F_AKCIPHER_STATELESS_MODE feature bit is negotiated, 1) if the
    driver uses the stateless mode, then the driver MUST set the \field{flag} field in
    struct virtio_crypto_op_header to ZERO and MUST set the fields in struct
    virtio_crypto_akcipher_flf_stateless.sess_para, 2) if the driver uses the session
    mode, then the driver MUST set the \field{flag} field in struct virtio_crypto_op_header
    to VIRTIO_CRYPTO_FLAG_SESSION_MODE.
\item The driver MUST set the \field{opcode} field in struct virtio_crypto_op_header
    to one of VIRTIO_CRYPTO_AKCIPHER_ENCRYPT, VIRTIO_CRYPTO_AKCIPHER_DESTROY_SESSION,
    VIRTIO_CRYPTO_AKCIPHER_SIGN, and VIRTIO_CRYPTO_AKCIPHER_VERIFY.
\end{itemize*}

\devicenormative{\paragraph}{AKCIPHER Service Operation}{Device Types / Crypto Device / Device Operation / AKCIPHER Service Operation}

\begin{itemize*}
\item If the VIRTIO_CRYPTO_F_AKCIPHER_STATELESS_MODE feature bit is negotiated, the
    device MUST parse the virtio_crypto_akcipher_data_vlf_stateless based on the \field{opcode}
    field in general header.
\item The device MUST copy the result of cryptographic operation in the dst_data[].
\item The device MUST set the \field{status} field in struct virtio_crypto_inhdr to
    one of the following values of enum VIRTIO_CRYPTO_STATUS:
\begin{itemize*}
\item VIRTIO_CRYPTO_OK if the operation success.
\item VIRTIO_CRYPTO_NOTSUPP if the requested algorithm or operation is unsupported.
\item VIRTIO_CRYPTO_BADMSG if the verification result is incorrect.
\item VIRTIO_CRYPTO_INVSESS if the session ID invalid when in session mode.
\item VIRTIO_CRYPTO_KEY_REJECTED if the signature verification failed.
\item VIRTIO_CRYPTO_ERR if any failure not mentioned above occurs.
\end{itemize*}
\end{itemize*}
