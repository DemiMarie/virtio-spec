\section{I2C Adapter Device}\label{sec:Device Types / I2C Adapter Device}

virtio-i2c is a virtual I2C adapter device. It provides a way to flexibly
organize and use the host I2C controlled devices from the guest. By attaching
the host ACPI I2C controlled nodes to the virtual I2C adapter device, the guest can
communicate with them without changing or adding extra drivers for these
controlled I2C devices.

\subsection{Device ID}\label{sec:Device Types / I2C Adapter Device / Device ID}
34

\subsection{Virtqueues}\label{sec:Device Types / I2C Adapter Device / Virtqueues}

\begin{description}
\item[0] requestq
\end{description}

\subsection{Feature bits}\label{sec:Device Types / I2C Adapter Device / Feature bits}

None currently defined.

\subsection{Device configuration layout}\label{sec:Device Types / I2C Adapter Device / Device configuration layout}

None currently defined.

\subsection{Device Initialization}\label{sec:Device Types / I2C Adapter Device / Device Initialization}

\begin{enumerate}
\item The virtqueue is initialized.
\end{enumerate}

\subsection{Device Operation}\label{sec:Device Types / I2C Adapter Device / Device Operation}

\subsubsection{Device Operation: Request Queue}\label{sec:Device Types / I2C Adapter Device / Device Operation: Request Queue}

The driver queues requests to the virtqueue, and they are used by the
device. The request is the representation of segments of an I2C
transaction. Each request is of the form:

\begin{lstlisting}
struct virtio_i2c_out_hdr {
        le16 addr;
        le16 padding;
        le32 flags;
};
\end{lstlisting}

\begin{lstlisting}
struct virtio_i2c_in_hdr {
        u8 status;
};
\end{lstlisting}

\begin{lstlisting}
struct virtio_i2c_req {
        struct virtio_i2c_out_hdr out_hdr;
        u8 write_buf[];
        u8 read_buf[];
        struct virtio_i2c_in_hdr in_hdr;
};
\end{lstlisting}

The \field{addr} of the request is the address of the I2C controlled device.
For 7-bit address mode (A0 ... A6) and 10-bit address mode (A0 ... A9),
the format of \field{addr} is defined as follows:

\begin{tabular}{ |l||l|l|l|l|l|l|l|l|l|l|l|l|l|l|l|l| }
\hline
Bits           & 15 & 14 & 13 & 12 & 11 & 10 & 9  & 8  & 7  & 6  & 5  & 4  & 3  & 2  & 1  & 0 \\
\hline
7-bit address  & 0  & 0  & 0  & 0  & 0  & 0  & 0  & 0  & A6 & A5 & A4 & A3 & A2 & A1 & A0 & 0 \\
\hline
10-bit address & A7 & A6 & A5 & A4 & A3 & A2 & A1 & A0 & 1  & 1  & 1  & 1  & 0  & A9 & A8 & 0 \\
\hline
\end{tabular}

The \field{padding} is used to pad to full dword.

The \field{flags} of the request is defined as follows:

\begin{description}
\item[VIRTIO_I2C_FLAGS_FAIL_NEXT(0)] is used to group the requests.
    For a group requests, a driver clears this bit on the final request
    and sets it on the other requests. If this bit is set and a device fails
    to process the current request, it needs to fail the next request instead
    of attempting to execute it.
\end{description}

Other bits of \field{flags} are currently reserved as zero for future feature
extensibility.

The \field{write_buf} of the request contains one segment of an I2C transaction
being written to the device.

The \field{read_buf} of the request contains one segment of an I2C transaction
being read from the device.

The final \field{status} byte of the request is written by the device: either
VIRTIO_I2C_MSG_OK for success or VIRTIO_I2C_MSG_ERR for error.

\begin{lstlisting}
#define VIRTIO_I2C_MSG_OK     0
#define VIRTIO_I2C_MSG_ERR    1
\end{lstlisting}

If ``length of \field{read_buf}''=0 and ``length of \field{write_buf}''>0,
the request is called write request.

If ``length of \field{read_buf}''>0 and ``length of \field{write_buf}''=0,
the request is called read request.

If ``length of \field{read_buf}''>0 and ``length of \field{write_buf}''>0,
the request is called write-read request. It means an I2C write segment followed
by a read segment. Usually, the write segment provides the number of an I2C
controlled device register to be read.

The case when ``length of \field{write_buf}''=0, and at the same time,
``length of \field{read_buf}''=0 doesn't make any sense.

\subsubsection{Device Operation: Operation Status}\label{sec:Device Types / I2C Adapter Device / Device Operation: Operation Status}

\field{addr}, \field{flags}, ``length of \field{write_buf}'' and ``length of \field{read_buf}''
are determined by the driver, while \field{status} is determined by the processing
of the device. A driver puts the data written to the device into \field{write_buf}, while
a device puts the data of the corresponding length into \field{read_buf} according to the
request of the driver.

A driver may send one request or multiple requests to the device at a time.
The requests in the virtqueue are both queued and processed in order.

If a driver sends multiple requests at a time and a device fails to process
some of them, then a device needs to set the \field{status} of the first failed request
to be VIRTIO_I2C_MSG_ERR. For the remaining requests in the same group with
the first failed one, a driver needs to treat them as VIRTIO_I2C_MSG_ERR, no matter
what \field{status} of them, a device needs to fail them instead of attempting to
execute them according to the VIRTIO_I2C_FLAGS_FAIL_NEXT bit.

\drivernormative{\subsubsection}{Device Operation}{Device Types / I2C Adapter Device / Device Operation}

A driver MUST set \field{addr} and \field{flags} before sending the request.

A driver MUST set the reserved bits of \field{flags} to be zero.

The driver MUST NOT send a request with ``length of \field{write_buf}''=0 and
``length of \field{read_buf}''=0 at the same time.

A driver MUST NOT use \field{read_buf} if the final \field{status} returned
from the device is VIRTIO_I2C_MSG_ERR.

A driver MUST queue the requests in order if multiple requests are going to
be sent at a time.

If multiple requests are sent at a time and some of them returned from the
device have the \field{status} being VIRTIO_I2C_MSG_ERR, a driver MUST treat
the first failed request and the remaining requests in the same group with
the first failed one as VIRTIO_I2C_MSG_ERR.

\devicenormative{\subsubsection}{Device Operation}{Device Types / I2C Adapter Device / Device Operation}

A device SHOULD keep consistent behaviors with the hardware as described in
\hyperref[intro:I2C]{I2C}.

A device MUST NOT change the value of \field{addr}, reserved bits of \field{flags}
and \field{write_buf}.

A device MUST place one I2C segment of the corresponding length into \field{read_buf}
according the driver's request.

A device MUST guarantee the requests in the virtqueue being processed in order
if multiple requests are received at a time.

If multiple requests are received at a time and processing of some of the
requests fails, a device MUST set the \field{status} of the first failed
request to be VIRTIO_I2C_MSG_ERR and MAY set the \field{status} of the
remaining requests in the same group with the first failed one
to be VIRTIO_I2C_MSG_ERR.
