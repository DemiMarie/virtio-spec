\section{Packed Virtqueues}\label{sec:Basic Facilities of a Virtio Device / Packed Virtqueues}

Packed virtqueues is an alternative compact virtqueue layout using
read-write memory, that is memory that is both read and written
by both host and guest.

Use of packed virtqueues is negotiated by the VIRTIO_F_RING_PACKED
feature bit.

Packed virtqueues support up to $2^{15}$ entries each.

With current transports, virtqueues are located in guest memory
allocated by the driver.
Each packed virtqueue consists of three parts:

\begin{itemize}
\item Descriptor Ring - occupies the Descriptor Area
\item Driver Event Suppression - occupies the Driver Area
\item Device Event Suppression - occupies the Device Area
\end{itemize}

Where the Descriptor Ring in turn consists of descriptors,
and where each descriptor can contain the following parts:

\begin{itemize}
\item Buffer ID
\item Element Address
\item Element Length
\item Flags
\end{itemize}

A buffer consists of zero or more device-readable physically-contiguous
elements followed by zero or more physically-contiguous
device-writable elements (each buffer has at least one element).

When the driver wants to send such a buffer to the device, it
writes at least one available descriptor describing elements of
the buffer into the Descriptor Ring.  The descriptor(s) are
associated with a buffer by means of a Buffer ID stored within
the descriptor.

The driver then notifies the device. When the device has finished
processing the buffer, it writes a used device descriptor
including the Buffer ID into the Descriptor Ring (overwriting a
driver descriptor previously made available), and sends a
used event notification.

The Descriptor Ring is used in a circular manner: the driver writes
descriptors into the ring in order. After reaching the end of the ring,
the next descriptor is placed at the head of the ring.  Once the ring is
full of driver descriptors, the driver stops sending new requests and
waits for the device to start processing descriptors and to write out
some used descriptors before making new driver descriptors
available.

Similarly, the device reads descriptors from the ring in order and
detects that a driver descriptor has been made available.  As
processing of descriptors is completed, used descriptors are
written by the device back into the ring.

Note: after reading driver descriptors and starting their
processing in order, the device might complete their processing out
of order.  Used device descriptors are written in the order
in which their processing is complete.

The Device Event Suppression data structure is write-only by the
device. It includes information for reducing the number of
device events, i.e., sending fewer available buffer notifications
to the device.

The Driver Event Suppression data structure is read-only by the
device. It includes information for reducing the number of
driver events, i.e., sending fewer used buffer notifications 
to the driver.

\subsection{Driver and Device Ring Wrap Counters}
\label{sec:Packed Virtqueues / Driver and Device Ring Wrap Counters}
Each of the driver and the device are expected to maintain,
internally, a single-bit ring wrap counter initialized to 1.

The counter maintained by the driver is called the Driver
Ring Wrap Counter. The driver changes the value of this counter
each time it makes available the
last descriptor in the ring (after making the last descriptor
available).

The counter maintained by the device is called the Device Ring Wrap
Counter.  The device changes the value of this counter
each time it uses the last descriptor in
the ring (after marking the last descriptor used).

It is easy to see that the Driver Ring Wrap Counter in the driver matches
the Device Ring Wrap Counter in the device when both are processing the same
descriptor, or when all available descriptors have been used.

To mark a descriptor as available and used, both the driver and
the device use the following two flags:
\begin{lstlisting}
#define VIRTQ_DESC_F_AVAIL     (1 << 7)
#define VIRTQ_DESC_F_USED      (1 << 15)
\end{lstlisting}

To mark a descriptor as available, the driver sets the
VIRTQ_DESC_F_AVAIL bit in Flags to match the internal Driver
Ring Wrap Counter.  It also sets the VIRTQ_DESC_F_USED bit to match the
\emph{inverse} value (i.e. to not match the internal Driver Ring
Wrap Counter).

To mark a descriptor as used, the device sets the
VIRTQ_DESC_F_USED bit in Flags to match the internal Device
Ring Wrap Counter.  It also sets the VIRTQ_DESC_F_AVAIL bit to match the
\emph{same} value.

Thus VIRTQ_DESC_F_AVAIL and VIRTQ_DESC_F_USED bits are different
for an available descriptor and equal for a used descriptor.

Note that this observation is mostly useful for sanity-checking
as these are necessary but not sufficient conditions - for
example, all descriptors are zero-initialized. To detect used and
available descriptors it is possible for drivers and devices to
keep track of the last observed value of
VIRTQ_DESC_F_USED/VIRTQ_DESC_F_AVAIL.  Other techniques to detect
VIRTQ_DESC_F_AVAIL/VIRTQ_DESC_F_USED bit changes might also be
possible.

\subsection{Polling of available and used descriptors}
\label{sec:Packed Virtqueues / Polling of available and used descriptors}

Writes of device and driver descriptors can generally be
reordered, but each side (driver and device) are only required to
poll (or test) a single location in memory: the next device descriptor after
the one they processed previously, in circular order.

Sometimes the device needs to only write out a single used descriptor
after processing a batch of multiple available descriptors.  As
described in more detail below, this can happen when using
descriptor chaining or with in-order
use of descriptors.  In this case, the device writes out a used
descriptor with the buffer id of the last descriptor in the group.
After processing the used descriptor, both device and driver then
skip forward in the ring the number of the remaining descriptors
in the group until processing (reading for the driver and writing
for the device) the next used descriptor.

\subsection{Write Flag}
\label{sec:Packed Virtqueues / Write Flag}

In an available descriptor, the VIRTQ_DESC_F_WRITE bit within Flags
is used to mark a descriptor as corresponding to a write-only or
read-only element of a buffer.

\begin{lstlisting}
/* This marks a descriptor as device write-only (otherwise device read-only). */
#define VIRTQ_DESC_F_WRITE     2
\end{lstlisting}

In a used descriptor, this bit is used to specify whether any
data has been written by the device into any parts of the buffer.


\subsection{Element Address and Length}
\label{sec:Packed Virtqueues / Element Address and Length}

In an available descriptor, Element Address corresponds to the
physical address of the buffer element. The length of the element assumed
to be physically contiguous is stored in Element Length.

In a used descriptor, Element Address is unused. Element Length
specifies the length of the buffer that has been initialized
(written to) by the device.

Element Length is reserved for used descriptors without the
VIRTQ_DESC_F_WRITE flag, and is ignored by drivers.

\subsection{Scatter-Gather Support}
\label{sec:Packed Virtqueues / Scatter-Gather Support}

Some drivers need an ability to supply a list of multiple buffer
elements (also known as a scatter/gather list) with a request.
Two features support this: descriptor chaining and indirect descriptors.

If neither feature is in use by the driver, each buffer is
physically-contiguous, either read-only or write-only and is
described completely by a single descriptor.

While unusual (most implementations either create all lists
solely using non-indirect descriptors, or always use a single
indirect element), if both features have been negotiated, mixing
indirect and non-indirect descriptors in a ring is valid, as long as each
list only contains descriptors of a given type.

Scatter/gather lists only apply to available descriptors. A
single used descriptor corresponds to the whole list.

The device limits the number of descriptors in a list through a
transport-specific and/or device-specific value. If not limited,
the maximum number of descriptors in a list is the virt queue
size.

\subsection{Next Flag: Descriptor Chaining}
\label{sec:Packed Virtqueues / Next Flag: Descriptor Chaining}

The packed ring format allows the driver to supply
a scatter/gather list to the device
by using multiple descriptors, and setting the VIRTQ_DESC_F_NEXT bit in
Flags for all but the last available descriptor.

\begin{lstlisting}
/* This marks a buffer as continuing. */
#define VIRTQ_DESC_F_NEXT   1
\end{lstlisting}

Buffer ID is included in the last descriptor in the list.

The driver always makes the first descriptor in the list
available after the rest of the list has been written out into
the ring. This guarantees that the device will never observe a
partial scatter/gather list in the ring.

Note: all flags, including VIRTQ_DESC_F_AVAIL, VIRTQ_DESC_F_USED,
VIRTQ_DESC_F_WRITE must be set/cleared correctly in all
descriptors in the list, not just the first one.

The device only writes out a single used descriptor for the whole
list. It then skips forward according to the number of
descriptors in the list. The driver needs to keep track of the size
of the list corresponding to each buffer ID, to be able to skip
to where the next used descriptor is written by the device.

For example, if descriptors are used in the same order in which
they are made available, this will result in the used descriptor
overwriting the first available descriptor in the list, the used
descriptor for the next list overwriting the first available
descriptor in the next list, etc.

VIRTQ_DESC_F_NEXT is reserved in used descriptors, and
should be ignored by drivers.

\subsection{Indirect Flag: Scatter-Gather Support}
\label{sec:Packed Virtqueues / Indirect Flag: Scatter-Gather Support}

Some devices benefit by concurrently dispatching a large number
of large requests. The VIRTIO_F_INDIRECT_DESC feature allows this. To increase
ring capacity the driver can store a (read-only by the device) table of indirect
descriptors anywhere in memory, and insert a descriptor in the main
virtqueue (with \field{Flags} bit VIRTQ_DESC_F_INDIRECT on) that refers to
a buffer element
containing this indirect descriptor table; \field{addr} and \field{len}
refer to the indirect table address and length in bytes,
respectively.
\begin{lstlisting}
/* This means the element contains a table of descriptors. */
#define VIRTQ_DESC_F_INDIRECT   4
\end{lstlisting}

The indirect table layout structure looks like this
(\field{len} is the Buffer Length of the descriptor that refers to this table,
which is a variable):

\begin{lstlisting}
struct pvirtq_indirect_descriptor_table {
        /* The actual descriptor structures (struct pvirtq_desc each) */
        struct pvirtq_desc desc[len / sizeof(struct pvirtq_desc)];
};
\end{lstlisting}

The first descriptor is located at the start of the indirect
descriptor table, additional indirect descriptors come
immediately afterwards. The VIRTQ_DESC_F_WRITE \field{flags} bit is the
only valid flag for descriptors in the indirect table. Others
are reserved and are ignored by the device.
Buffer ID is also reserved and is ignored by the device.

In descriptors with VIRTQ_DESC_F_INDIRECT set VIRTQ_DESC_F_WRITE
is reserved and is ignored by the device.

\subsection{In-order use of descriptors}
\label{sec:Packed Virtqueues / In-order use of descriptors}

Some devices always use descriptors in the same order in which
they have been made available. These devices can offer the
VIRTIO_F_IN_ORDER feature. If negotiated, this knowledge allows
devices to notify the use of a batch of buffers to the driver by
only writing out a single used descriptor with the Buffer ID
corresponding to the last descriptor in the batch.

The device then skips forward in the ring according to the size of
the batch. The driver needs to look up the used Buffer ID and
calculate the batch size to be able to advance to where the next
used descriptor will be written by the device.

This will result in the used descriptor overwriting the first
available descriptor in the batch, the used descriptor for the
next batch overwriting the first available descriptor in the next
batch, etc.

The skipped buffers (for which no used descriptor was written)
are assumed to have been used (read or written) by the
device completely.

\subsection{Multi-buffer requests}
\label{sec:Packed Virtqueues / Multi-buffer requests}
Some devices combine multiple buffers as part of processing of a
single request.  These devices always mark the descriptor
corresponding to the first buffer in the request used after the
rest of the descriptors (corresponding to rest of the buffers) in
the request - which follow the first descriptor in ring order -
has been marked used and written out into the ring.  This
guarantees that the driver will never observe a partial request
in the ring.

\subsection{Driver and Device Event Suppression}
\label{sec:Packed Virtqueues / Driver and Device Event Suppression}
In many systems used and available buffer notifications involve
significant overhead. To mitigate this overhead,
each virtqueue includes two identical structures used for
controlling notifications between the device and the driver.

The Driver Event Suppression structure is read-only by the
device and controls the used buffer notifications sent by the device
to the driver.

The Device Event Suppression structure is read-only by
the driver and controls the available buffer notifications sent by the
driver to the device.

Each of these Event Suppression structures includes the following fields:

\begin{description}
\item [Descriptor Ring Change Event Flags] Takes values:
\begin{lstlisting}
/* Enable events */
#define RING_EVENT_FLAGS_ENABLE 0x0
/* Disable events */
#define RING_EVENT_FLAGS_DISABLE 0x1
/*
 * Enable events for a specific descriptor
 * (as specified by Descriptor Ring Change Event Offset/Wrap Counter).
 * Only valid if VIRTIO_F_EVENT_IDX has been negotiated.
 */
#define RING_EVENT_FLAGS_DESC 0x2
/* The value 0x3 is reserved */
\end{lstlisting}
\item [Descriptor Ring Change Event Offset] If Event Flags set to descriptor
specific event: offset within the ring (in units of descriptor
size). Event will only trigger when this descriptor is
made available/used respectively.
\item [Descriptor Ring Change Event Wrap Counter] If Event Flags set to descriptor
specific event: offset within the ring (in units of descriptor
size). Event will only trigger when Ring Wrap Counter
matches this value and a descriptor is
made available/used respectively.
\end{description}

After writing out some descriptors, both the device and the driver
are expected to consult the relevant structure to find out
whether a used respectively an available buffer notification should be sent.

\subsubsection{Structure Size and Alignment}
\label{sec:Packed Virtqueues / Structure Size and Alignment}

Each part of the virtqueue is physically-contiguous in guest memory,
and has different alignment requirements.

The memory alignment and size requirements, in bytes, of each part of the
virtqueue are summarized in the following table:

\begin{tabular}{|l|l|l|}
\hline
Virtqueue Part    & Alignment & Size \\
\hline \hline
Descriptor Ring  & 16        & $16 * $(Queue Size) \\
\hline
Device Event Suppression    & 4         & 4 \\
 \hline
Driver Event Suppression         & 4         & 4 \\
 \hline
\end{tabular}

The Alignment column gives the minimum alignment for each part
of the virtqueue.

The Size column gives the total number of bytes for each
part of the virtqueue.

Queue Size corresponds to the maximum number of descriptors in the
virtqueue\footnote{For example, if Queue Size is 4 then at most 4 buffers
can be queued at any given time.}.  The Queue Size value does not
have to be a power of 2.

\drivernormative{\subsection}{Virtqueues}{Basic Facilities of a
Virtio Device / Packed Virtqueues}
The driver MUST ensure that the physical address of the first byte
of each virtqueue part is a multiple of the specified alignment value
in the above table.

\devicenormative{\subsection}{Virtqueues}{Basic Facilities of a
Virtio Device / Packed Virtqueues}
The device MUST start processing driver descriptors in the order
in which they appear in the ring.
The device MUST start writing device descriptors into the ring in
the order in which they complete.
The device MAY reorder descriptor writes once they are started.

\subsection{The Virtqueue Descriptor Format}\label{sec:Basic
Facilities of a Virtio Device / Packed Virtqueues / The Virtqueue
Descriptor Format}

The available descriptor refers to the buffers the driver is sending
to the device. \field{addr} is a physical address, and the
descriptor is identified with a buffer using the \field{id} field.

\begin{lstlisting}
struct pvirtq_desc {
        /* Buffer Address. */
        le64 addr;
        /* Buffer Length. */
        le32 len;
        /* Buffer ID. */
        le16 id;
        /* The flags depending on descriptor type. */
        le16 flags;
};
\end{lstlisting}

The descriptor ring is zero-initialized.

\subsection{Event Suppression Structure Format}\label{sec:Basic
Facilities of a Virtio Device / Packed Virtqueues / Event Suppression Structure
Format}

The following structure is used to reduce the number of
notifications sent between driver and device.

\begin{lstlisting}
struct pvirtq_event_suppress {
        le16 {
             desc_event_off : 15; /* Descriptor Ring Change Event Offset */
             desc_event_wrap : 1; /* Descriptor Ring Change Event Wrap Counter */
        } desc; /* If desc_event_flags set to RING_EVENT_FLAGS_DESC */
        le16 {
             desc_event_flags : 2, /* Descriptor Ring Change Event Flags */
             reserved : 14; /* Reserved, set to 0 */
        } flags;
};
\end{lstlisting}

\devicenormative{\subsection}{The Virtqueue Descriptor Table}{Basic Facilities of a Virtio Device / Packed Virtqueues / The Virtqueue Descriptor Table}
A device MUST NOT write to a device-readable buffer, and a device SHOULD NOT
read a device-writable buffer.
A device MUST NOT use a descriptor unless it observes
the VIRTQ_DESC_F_AVAIL bit in its \field{flags} being changed
(e.g. as compared to the initial zero value).
A device MUST NOT change a descriptor after changing it's
the VIRTQ_DESC_F_USED bit in its \field{flags}.

\drivernormative{\subsection}{The Virtqueue Descriptor Table}{Basic Facilities of a Virtio Device / PAcked Virtqueues / The Virtqueue Descriptor Table}
A driver MUST NOT change a descriptor unless it observes
the VIRTQ_DESC_F_USED bit in its \field{flags} being changed.
A driver MUST NOT change a descriptor after changing
the VIRTQ_DESC_F_AVAIL bit in its \field{flags}.
When notifying the device, driver MUST set
\field{next_off} and
\field{next_wrap} to match the next descriptor
not yet made available to the device.
A driver MAY send multiple available buffer notifications without making
any new descriptors available to the device.

\drivernormative{\subsection}{Scatter-Gather Support}{Basic Facilities of a
Virtio Device / Packed Virtqueues / Scatter-Gather Support}
A driver MUST NOT create a descriptor list longer than allowed
by the device.

A driver MUST NOT create a descriptor list longer than the Queue
Size.

This implies that loops in the descriptor list are forbidden!

The driver MUST place any device-writable descriptor elements after
any device-readable descriptor elements.

A driver MUST NOT depend on the device to use more descriptors
to be able to write out all descriptors in a list. A driver
MUST make sure there's enough space in the ring
for the whole list before making the first descriptor in the list
available to the device.

A driver MUST NOT make the first descriptor in the list available
before all subsequent descriptors comprising the list are made
available.

\devicenormative{\subsection}{Scatter-Gather Support}{Basic Facilities of a
Virtio Device / Packed Virtqueues / Scatter-Gather Support}
The device MUST use descriptors in a list chained by the
VIRTQ_DESC_F_NEXT flag in the same order that they
were made available by the driver.

The device MAY limit the number of buffers it will allow in a
list.

\drivernormative{\subsection}{Indirect Descriptors}{Basic Facilities of a Virtio Device / Packed Virtqueues / The Virtqueue Descriptor Table / Indirect Descriptors}
The driver MUST NOT set the VIRTQ_DESC_F_INDIRECT flag unless the
VIRTIO_F_INDIRECT_DESC feature was negotiated.   The driver MUST NOT
set any flags except DESC_F_WRITE within an indirect descriptor.

A driver MUST NOT create a descriptor chain longer than allowed
by the device.

A driver MUST NOT write direct descriptors with
VIRTQ_DESC_F_INDIRECT set in a scatter-gather list linked by
VIRTQ_DESC_F_NEXT.
\field{flags}.

\subsection{Virtqueue Operation}\label{sec:Basic Facilities of a Virtio Device / Packed Virtqueues / Virtqueue Operation}

There are two parts to virtqueue operation: supplying new
available buffers to the device, and processing used buffers from
the device.

What follows is the requirements of each of these two parts
when using the packed virtqueue format in more detail.

\subsection{Supplying Buffers to The Device}\label{sec:Basic Facilities of a Virtio Device / Packed Virtqueues / Supplying Buffers to The Device}

The driver offers buffers to one of the device's virtqueues as follows:

\begin{enumerate}
\item The driver places the buffer into free descriptor(s) in the Descriptor Ring.

\item The driver performs a suitable memory barrier to ensure that it updates
  the descriptor(s) before checking for notification suppression.

\item If notifications are not suppressed, the driver notifies the device
    of the new available buffers.
\end{enumerate}

What follows are the requirements of each stage in more detail.

\subsubsection{Placing Available Buffers Into The Descriptor Ring}\label{sec:Basic Facilities of a Virtio Device / Virtqueues / Supplying Buffers to The Device / Placing Available Buffers Into The Descriptor Ring}

For each buffer element, b:

\begin{enumerate}
\item Get the next descriptor table entry, d
\item Get the next free buffer id value
\item Set \field{d.addr} to the physical address of the start of b
\item Set \field{d.len} to the length of b.
\item Set \field{d.id} to the buffer id
\item Calculate the flags as follows:
\begin{enumerate}
\item If b is device-writable, set the VIRTQ_DESC_F_WRITE bit to 1, otherwise 0
\item Set the VIRTQ_DESC_F_AVAIL bit to the current value of the Driver Ring Wrap Counter
\item Set the VIRTQ_DESC_F_USED bit to inverse value
\end{enumerate}
\item Perform a memory barrier to ensure that the descriptor has
      been initialized
\item Set \field{d.flags} to the calculated flags value
\item If d is the last descriptor in the ring, toggle the
      Driver Ring Wrap Counter
\item Otherwise, increment d to point at the next descriptor
\end{enumerate}

This makes a single descriptor buffer available. However, in
general the driver MAY make use of a batch of descriptors as part
of a single request. In that case, it defers updating
the descriptor flags for the first descriptor
(and the previous memory barrier) until after the rest of
the descriptors have been initialized.

Once the descriptor \field{flags} field is updated by the driver, this exposes
the descriptor and its contents.  The device MAY
access the descriptor and any following descriptors the driver created and the
memory they refer to immediately.

\drivernormative{\paragraph}{Updating flags}{Basic Facilities of
a Virtio Device / Packed Virtqueues / Supplying Buffers to The
Device / Updating flags}
The driver MUST perform a suitable memory barrier before the
\field{flags} update, to ensure the
device sees the most up-to-date copy.

\subsubsection{Sending Available Buffer Notifications}\label{sec:Basic Facilities
of a Virtio Device / Packed Virtqueues / Supplying Buffers to The Device
/ Sending Available Buffer Notifications}

The actual method of device notification is bus-specific, but generally
it can be expensive.  So the device MAY suppress such notifications if it
doesn't need them, using the Event Suppression structure comprising the
Device Area as detailed in section \ref{sec:Basic
Facilities of a Virtio Device / Packed Virtqueues / Event
Suppression Structure Format}.

The driver has to be careful to expose the new \field{flags}
value before checking if notifications are suppressed.

\subsubsection{Implementation Example}\label{sec:Basic Facilities of a Virtio Device / Packed Virtqueues / Supplying Buffers to The Device / Implementation Example}

Below is a driver code example. It does not attempt to reduce
the number of available buffer notifications, neither does it support
the VIRTIO_F_EVENT_IDX feature.

\begin{lstlisting}
/* Note: vq->avail_wrap_count is initialized to 1 */
/* Note: vq->sgs is an array same size as the ring */

id = alloc_id(vq);

first = vq->next_avail;
sgs = 0;
for (each buffer element b) {
        sgs++;

        vq->ids[vq->next_avail] = -1;
        vq->desc[vq->next_avail].address = get_addr(b);
        vq->desc[vq->next_avail].len = get_len(b);

        avail = vq->avail_wrap_count ? VIRTQ_DESC_F_AVAIL : 0;
        used = !vq->avail_wrap_count ? VIRTQ_DESC_F_USED : 0;
        f = get_flags(b) | avail | used;
        if (b is not the last buffer element) {
                f |= VIRTQ_DESC_F_NEXT;
        }

        /* Don't mark the 1st descriptor available until all of them are ready. */
        if (vq->next_avail == first) {
                flags = f;
        } else {
                vq->desc[vq->next_avail].flags = f;
        }

        last = vq->next_avail;

        vq->next_avail++;

        if (vq->next_avail >= vq->size) {
                vq->next_avail = 0;
                vq->avail_wrap_count ^= 1;
        }
}
vq->sgs[id] = sgs;
/* ID included in the last descriptor in the list */
vq->desc[last].id = id;
write_memory_barrier();
vq->desc[first].flags = flags;

memory_barrier();

if (vq->device_event.flags != RING_EVENT_FLAGS_DISABLE) {
        notify_device(vq);
}

\end{lstlisting}


\drivernormative{\paragraph}{Sending Available Buffer Notifications} {Basic Facilities of a Virtio Device / Packed Virtqueues / Supplying Buffers to The Device / Sending Available Buffer Notifications}
The driver MUST perform a suitable memory barrier before reading
the Event Suppression structure occupying the Device Area. Failing
to do so could result in mandatory available buffer notifications
not being sent.

\subsection{Receiving Used Buffers From The Device}\label{sec:Basic Facilities of a Virtio Device / Packed Virtqueues / Receiving Used Buffers From The Device}

Once the device has used buffers referred to by a descriptor (read from or written to them, or
parts of both, depending on the nature of the virtqueue and the
device), it sends a used buffer notification to the driver
as detailed in section \ref{sec:Basic
Facilities of a Virtio Device / Packed Virtqueues / Event
Suppression Structure Format}.

\begin{note}

For optimal performance, a driver MAY disable used buffer notifications
while processing the used buffers, but beware the problem of missing
notifications between emptying the ring and reenabling used buffer
notifications.  This is usually handled by re-checking for more used
buffers after notifications are re-enabled:

\end{note}

\begin{lstlisting}
/* Note: vq->used_wrap_count is initialized to 1 */

vq->driver_event.flags = RING_EVENT_FLAGS_DISABLE;

for (;;) {
        struct pvirtq_desc *d = vq->desc[vq->next_used];

        /*
         * Check that
         * 1. Descriptor has been made available. This check is necessary
         *    if the driver is making new descriptors available in parallel
         *    with this processing of used descriptors (e.g. from another thread).
         *    Note: there are many other ways to check this, e.g.
         *    track the number of outstanding available descriptors or buffers
         *    and check that it's not 0.
         * 2. Descriptor has been used by the device.
         */
        flags = d->flags;
        bool avail = flags & VIRTQ_DESC_F_AVAIL;
        bool used = flags & VIRTQ_DESC_F_USED;
        if (avail != vq->used_wrap_count || used != vq->used_wrap_count) {
                vq->driver_event.flags = RING_EVENT_FLAGS_ENABLE;
                memory_barrier();

                /*
                 * Re-test in case the driver made more descriptors available in
                 * parallel with the used descriptor processing (e.g. from another
                 * thread) and/or the device used more descriptors before the driver
                 * enabled events.
                 */
                flags = d->flags;
                bool avail = flags & VIRTQ_DESC_F_AVAIL;
                bool used = flags & VIRTQ_DESC_F_USED;
                if (avail != vq->used_wrap_count || used != vq->used_wrap_count) {
                        break;
                }

                vq->driver_event.flags = RING_EVENT_FLAGS_DISABLE;
        }

        read_memory_barrier();

        /* skip descriptors until the next buffer */
        id = d->id;
        assert(id < vq->size);
        sgs = vq->sgs[id];
        vq->next_used += sgs;
        if (vq->next_used >= vq->size) {
                vq->next_used -= vq->size;
                vq->used_wrap_count ^= 1;
        }

        free_id(vq, id);

        process_buffer(d);
}
\end{lstlisting}
