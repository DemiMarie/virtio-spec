\section{Virtio Over PCI Bus}\label{sec:Virtio Transport Options / Virtio Over PCI Bus}

Virtio devices are commonly implemented as PCI devices.

A Virtio device can be implemented as any kind of PCI device:
a Conventional PCI device or a PCI Express
device.  To assure designs meet the latest level
requirements, see
the PCI-SIG home page at \url{http://www.pcisig.com} for any
approved changes.

\devicenormative{\subsection}{Virtio Over PCI Bus}{Virtio Transport Options / Virtio Over PCI Bus}
A Virtio device using Virtio Over PCI Bus MUST expose to
guest an interface that meets the specification requirements of
the appropriate PCI specification: \hyperref[intro:PCI]{[PCI]}
and \hyperref[intro:PCIe]{[PCIe]}
respectively.

\subsection{PCI Device Discovery}\label{sec:Virtio Transport Options / Virtio Over PCI Bus / PCI Device Discovery}

Any PCI device with PCI Vendor ID 0x1AF4, and PCI Device ID 0x1000 through
0x107F inclusive is a virtio device. The actual value within this range
indicates which virtio device is supported by the device.
The PCI Device ID is calculated by adding 0x1040 to the Virtio Device ID,
as indicated in section \ref{sec:Device Types}.
Additionally, devices MAY utilize a Transitional PCI Device ID range,
0x1000 to 0x103F depending on the device type.

\devicenormative{\subsubsection}{PCI Device Discovery}{Virtio Transport Options / Virtio Over PCI Bus / PCI Device Discovery}

Devices MUST have the PCI Vendor ID 0x1AF4.
Devices MUST either have the PCI Device ID calculated by adding 0x1040
to the Virtio Device ID, as indicated in section \ref{sec:Device
Types} or have the Transitional PCI Device ID depending on the device type,
as follows:

\begin{tabular}{|l|c|}
\hline
Transitional PCI Device ID  &  Virtio Device    \\
\hline \hline
0x1000      &   network device     \\
\hline
0x1001     &   block device     \\
\hline
0x1002     & memory ballooning (traditional)  \\
\hline
0x1003     &      console       \\
\hline
0x1004     &     SCSI host      \\
\hline
0x1005     &  entropy source    \\
\hline
0x1009     &   9P transport     \\
\hline
\end{tabular}

For example, the network device with the Virtio Device ID 1
has the PCI Device ID 0x1041 or the Transitional PCI Device ID 0x1000.

The PCI Subsystem Vendor ID and the PCI Subsystem Device ID MAY reflect
the PCI Vendor and Device ID of the environment (for informational purposes by the driver).

Non-transitional devices SHOULD have a PCI Device ID in the range
0x1040 to 0x107f.
Non-transitional devices SHOULD have a PCI Revision ID of 1 or higher.
Non-transitional devices SHOULD have a PCI Subsystem Device ID of 0x40 or higher.

This is to reduce the chance of a legacy driver attempting
to drive the device.

\drivernormative{\subsubsection}{PCI Device Discovery}{Virtio Transport Options / Virtio Over PCI Bus / PCI Device Discovery}
Drivers MUST match devices with the PCI Vendor ID 0x1AF4 and
the PCI Device ID in the range 0x1040 to 0x107f,
calculated by adding 0x1040 to the Virtio Device ID,
as indicated in section \ref{sec:Device Types}.
Drivers for device types listed in section \ref{sec:Virtio
Transport Options / Virtio Over PCI Bus / PCI Device Discovery}
MUST match devices with the PCI Vendor ID 0x1AF4 and
the Transitional PCI Device ID indicated in section
 \ref{sec:Virtio
Transport Options / Virtio Over PCI Bus / PCI Device Discovery}.

Drivers MUST match any PCI Revision ID value.
Drivers MAY match any PCI Subsystem Vendor ID and any
PCI Subsystem Device ID value.

\subsubsection{Legacy Interfaces: A Note on PCI Device Discovery}\label{sec:Virtio Transport Options / Virtio Over PCI Bus / PCI Device Discovery / Legacy Interfaces: A Note on PCI Device Discovery}
Transitional devices MUST have a PCI Revision ID of 0.
Transitional devices MUST have the PCI Subsystem Device ID
matching the Virtio Device ID, as indicated in section \ref{sec:Device Types}.
Transitional devices MUST have the Transitional PCI Device ID in
the range 0x1000 to 0x103f.

This is to match legacy drivers.

\subsection{PCI Device Layout}\label{sec:Virtio Transport Options / Virtio Over PCI Bus / PCI Device Layout}

The device is configured via I/O and/or memory regions (though see
\ref{sec:Virtio Transport Options / Virtio Over PCI Bus / PCI Device Layout / PCI configuration access capability}
for access via the PCI configuration space), as specified by Virtio
Structure PCI Capabilities.

Fields of different sizes are present in the device
configuration regions.
All 64-bit, 32-bit and 16-bit fields are little-endian.
64-bit fields are to be treated as two 32-bit fields,
with low 32 bit part followed by the high 32 bit part.

\drivernormative{\subsubsection}{PCI Device Layout}{Virtio Transport Options / Virtio Over PCI Bus / PCI Device Layout}

For device configuration access, the driver MUST use 8-bit wide
accesses for 8-bit wide fields, 16-bit wide and aligned accesses
for 16-bit wide fields and 32-bit wide and aligned accesses for
32-bit and 64-bit wide fields. For 64-bit fields, the driver MAY
access each of the high and low 32-bit parts of the field
independently.

\devicenormative{\subsubsection}{PCI Device Layout}{Virtio Transport Options / Virtio Over PCI Bus / PCI Device Layout}

For 64-bit device configuration fields, the device MUST allow driver
independent access to high and low 32-bit parts of the field.

\subsection{Virtio Structure PCI Capabilities}\label{sec:Virtio Transport Options / Virtio Over PCI Bus / Virtio Structure PCI Capabilities}

The virtio device configuration layout includes several structures:
\begin{itemize}
\item Common configuration
\item Notifications
\item ISR Status
\item Device-specific configuration (optional)
\item PCI configuration access
\end{itemize}

Each structure can be mapped by a Base Address register (BAR) belonging to
the function, or accessed via the special VIRTIO_PCI_CAP_PCI_CFG field in the PCI configuration space.

The location of each structure is specified using a vendor-specific PCI capability located
on the capability list in PCI configuration space of the device.
This virtio structure capability uses little-endian format; all fields are
read-only for the driver unless stated otherwise:

\begin{lstlisting}
struct virtio_pci_cap {
        u8 cap_vndr;    /* Generic PCI field: PCI_CAP_ID_VNDR */
        u8 cap_next;    /* Generic PCI field: next ptr. */
        u8 cap_len;     /* Generic PCI field: capability length */
        u8 cfg_type;    /* Identifies the structure. */
        u8 bar;         /* Where to find it. */
        u8 id;          /* Multiple capabilities of the same type */
        u8 padding[2];  /* Pad to full dword. */
        le32 offset;    /* Offset within bar. */
        le32 length;    /* Length of the structure, in bytes. */
};
\end{lstlisting}

This structure can be followed by extra data, depending on
\field{cfg_type}, as documented below.

The fields are interpreted as follows:

\begin{description}
\item[\field{cap_vndr}]
        0x09; Identifies a vendor-specific capability.

\item[\field{cap_next}]
        Link to next capability in the capability list in the PCI configuration space.

\item[\field{cap_len}]
        Length of this capability structure, including the whole of
        struct virtio_pci_cap, and extra data if any.
        This length MAY include padding, or fields unused by the driver.

\item[\field{cfg_type}]
        identifies the structure, according to the following table:

\begin{lstlisting}
/* Common configuration */
#define VIRTIO_PCI_CAP_COMMON_CFG        1
/* Notifications */
#define VIRTIO_PCI_CAP_NOTIFY_CFG        2
/* ISR Status */
#define VIRTIO_PCI_CAP_ISR_CFG           3
/* Device specific configuration */
#define VIRTIO_PCI_CAP_DEVICE_CFG        4
/* PCI configuration access */
#define VIRTIO_PCI_CAP_PCI_CFG           5
/* Shared memory region */
#define VIRTIO_PCI_CAP_SHARED_MEMORY_CFG 8
/* Vendor-specific data */
#define VIRTIO_PCI_CAP_VENDOR_CFG        9
\end{lstlisting}

        Any other value is reserved for future use.

        Each structure is detailed individually below.

        The device MAY offer more than one structure of any type - this makes it
        possible for the device to expose multiple interfaces to drivers.  The order of
        the capabilities in the capability list specifies the order of preference
        suggested by the device.  A device may specify that this ordering mechanism be
        overridden by the use of the \field{id} field.
        \begin{note}
          For example, on some hypervisors, notifications using IO accesses are
        faster than memory accesses. In this case, the device would expose two
        capabilities with \field{cfg_type} set to VIRTIO_PCI_CAP_NOTIFY_CFG:
        the first one addressing an I/O BAR, the second one addressing a memory BAR.
        In this example, the driver would use the I/O BAR if I/O resources are available, and fall back on
        memory BAR when I/O resources are unavailable.
        \end{note}

\item[\field{bar}]
        values 0x0 to 0x5 specify a Base Address register (BAR) belonging to
        the function located beginning at 10h in PCI Configuration Space
        and used to map the structure into Memory or I/O Space.
        The BAR is permitted to be either 32-bit or 64-bit, it can map Memory Space
        or I/O Space.

        Any other value is reserved for future use.

\item[\field{id}]
        Used by some device types to uniquely identify multiple capabilities
        of a certain type. If the device type does not specify the meaning of
        this field, its contents are undefined.


\item[\field{offset}]
        indicates where the structure begins relative to the base address associated
        with the BAR.  The alignment requirements of \field{offset} are indicated
        in each structure-specific section below.

\item[\field{length}]
        indicates the length of the structure.

        \field{length} MAY include padding, or fields unused by the driver, or
        future extensions.

        \begin{note}
        For example, a future device might present a large structure size of several
        MBytes.
        As current devices never utilize structures larger than 4KBytes in size,
        driver MAY limit the mapped structure size to e.g.
        4KBytes (thus ignoring parts of structure after the first
        4KBytes) to allow forward compatibility with such devices without loss of
        functionality and without wasting resources.
        \end{note}
\end{description}

A variant of this type, struct virtio_pci_cap64, is defined for
those capabilities that require offsets or lengths larger than
4GiB:

\begin{lstlisting}
struct virtio_pci_cap64 {
        struct virtio_pci_cap cap;
        u32 offset_hi;
        u32 length_hi;
};
\end{lstlisting}

Given that the \field{cap.length} and \field{cap.offset} fields
are only 32 bit, the additional \field{offset_hi} and \field{length_hi}
fields provide the most significant 32 bits of a total 64 bit offset and
length within the BAR specified by \field{cap.bar}.

\drivernormative{\subsubsection}{Virtio Structure PCI Capabilities}{Virtio Transport Options / Virtio Over PCI Bus / Virtio Structure PCI Capabilities}

The driver MUST ignore any vendor-specific capability structure which has
a reserved \field{cfg_type} value.

The driver SHOULD use the first instance of each virtio structure type they can
support.

The driver MUST accept a \field{cap_len} value which is larger than specified here.

The driver MUST ignore any vendor-specific capability structure which has
a reserved \field{bar} value.

        The drivers SHOULD only map part of configuration structure
        large enough for device operation.  The drivers MUST handle
        an unexpectedly large \field{length}, but MAY check that \field{length}
        is large enough for device operation.

The driver MUST NOT write into any field of the capability structure,
with the exception of those with \field{cap_type} VIRTIO_PCI_CAP_PCI_CFG as
detailed in \ref{drivernormative:Virtio Transport Options / Virtio Over PCI Bus / PCI Device Layout / PCI configuration access capability}.

\devicenormative{\subsubsection}{Virtio Structure PCI Capabilities}{Virtio Transport Options / Virtio Over PCI Bus / Virtio Structure PCI Capabilities}

The device MUST include any extra data (from the beginning of the \field{cap_vndr} field
through end of the extra data fields if any) in \field{cap_len}.
The device MAY append extra data
or padding to any structure beyond that.

If the device presents multiple structures of the same type, it SHOULD order
them from optimal (first) to least-optimal (last).

\subsubsection{Common configuration structure layout}\label{sec:Virtio Transport Options / Virtio Over PCI Bus / PCI Device Layout / Common configuration structure layout}

The common configuration structure is found at the \field{bar} and \field{offset} within the VIRTIO_PCI_CAP_COMMON_CFG capability; its layout is below.

\begin{lstlisting}
struct virtio_pci_common_cfg {
        /* About the whole device. */
        le32 device_feature_select;     /* read-write */
        le32 device_feature;            /* read-only for driver */
        le32 driver_feature_select;     /* read-write */
        le32 driver_feature;            /* read-write */
        le16 config_msix_vector;        /* read-write */
        le16 num_queues;                /* read-only for driver */
        u8 device_status;               /* read-write */
        u8 config_generation;           /* read-only for driver */

        /* About a specific virtqueue. */
        le16 queue_select;              /* read-write */
        le16 queue_size;                /* read-write */
        le16 queue_msix_vector;         /* read-write */
        le16 queue_enable;              /* read-write */
        le16 queue_notify_off;          /* read-only for driver */
        le64 queue_desc;                /* read-write */
        le64 queue_driver;              /* read-write */
        le64 queue_device;              /* read-write */
        le16 queue_notif_config_data;   /* read-only for driver */
        le16 queue_reset;               /* read-write */

        /* About the administration virtqueue. */
        le16 admin_queue_index;         /* read-only for driver */
        le16 admin_queue_num;         /* read-only for driver */
};
\end{lstlisting}

\begin{description}
\item[\field{device_feature_select}]
        The driver uses this to select which feature bits \field{device_feature} shows.
        Value 0x0 selects Feature Bits 0 to 31, 0x1 selects Feature Bits 32 to 63, etc.

\item[\field{device_feature}]
        The device uses this to report which feature bits it is
        offering to the driver: the driver writes to
        \field{device_feature_select} to select which feature bits are presented.

\item[\field{driver_feature_select}]
        The driver uses this to select which feature bits \field{driver_feature} shows.
        Value 0x0 selects Feature Bits 0 to 31, 0x1 selects Feature Bits 32 to 63, etc.

\item[\field{driver_feature}]
        The driver writes this to accept feature bits offered by the device.
        Driver Feature Bits selected by \field{driver_feature_select}.

\item[\field{config_msix_vector}]
        Set by the driver to the MSI-X vector for configuration change notifications.

\item[\field{num_queues}]
        The device specifies the maximum number of virtqueues supported here.
        This excludes administration virtqueues if any are supported.

\item[\field{device_status}]
        The driver writes the device status here (see \ref{sec:Basic Facilities of a Virtio Device / Device Status Field}). Writing 0 into this
        field resets the device.

\item[\field{config_generation}]
        Configuration atomicity value.  The device changes this every time the
        configuration noticeably changes.

\item[\field{queue_select}]
        Queue Select. The driver selects which virtqueue the following
        fields refer to.

\item[\field{queue_size}]
        Queue Size.  On reset, specifies the maximum queue size supported by
        the device. This can be modified by the driver to reduce memory requirements.
        A 0 means the queue is unavailable.

\item[\field{queue_msix_vector}]
        Set by the driver to the MSI-X vector for virtqueue notifications.

\item[\field{queue_enable}]
        The driver uses this to selectively prevent the device from executing requests from this virtqueue.
        1 - enabled; 0 - disabled.

\item[\field{queue_notify_off}]
        The driver reads this to calculate the offset from start of Notification structure at
        which this virtqueue is located.
        \begin{note} this is \em{not} an offset in bytes.
        See \ref{sec:Virtio Transport Options / Virtio Over PCI Bus / PCI Device Layout / Notification capability} below.
        \end{note}

\item[\field{queue_desc}]
        The driver writes the physical address of Descriptor Area here.  See section \ref{sec:Basic Facilities of a Virtio Device / Virtqueues}.

\item[\field{queue_driver}]
        The driver writes the physical address of Driver Area here.  See section \ref{sec:Basic Facilities of a Virtio Device / Virtqueues}.

\item[\field{queue_device}]
        The driver writes the physical address of Device Area here.  See section \ref{sec:Basic Facilities of a Virtio Device / Virtqueues}.

\item[\field{queue_notif_config_data}]
        This field exists only if VIRTIO_F_NOTIF_CONFIG_DATA has been negotiated.
        The driver will use this value when driver sends available buffer
        notification to the device.
        See section \ref{sec:Virtio Transport Options / Virtio Over PCI Bus / PCI-specific Initialization And Device Operation / Available Buffer Notifications}.
        \begin{note}
        This field provides the device with flexibility to determine how virtqueues
        will be referred to in available buffer notifications.
        In a trivial case the device can set \field{queue_notif_config_data} to
        the virtqueue index. Some devices may benefit from providing another value,
        for example an internal virtqueue identifier, or an internal offset
        related to the virtqueue index.
        \end{note}
        \begin{note}
        This field was previously known as queue_notify_data.
        \end{note}

\item[\field{queue_reset}]
        The driver uses this to selectively reset the queue.
        This field exists only if VIRTIO_F_RING_RESET has been
        negotiated. (see \ref{sec:Basic Facilities of a Virtio Device / Virtqueues / Virtqueue Reset}).

\item[\field{admin_queue_index}]
        The device uses this to report the index of the first administration virtqueue.
        This field is valid only if VIRTIO_F_ADMIN_VQ has been negotiated.
\item[\field{admin_queue_num}]
	The device uses this to report the number of the
	supported administration virtqueues.
	Virtqueues with index
	between \field{admin_queue_index} and (\field{admin_queue_index} +
	\field{admin_queue_num} - 1) inclusive serve as administration
	virtqueues.
	The value 0 indicates no supported administration virtqueues.
	This field is valid only if VIRTIO_F_ADMIN_VQ has been
	negotiated.
\end{description}

\devicenormative{\paragraph}{Common configuration structure layout}{Virtio Transport Options / Virtio Over PCI Bus / PCI Device Layout / Common configuration structure layout}
\field{offset} MUST be 4-byte aligned.

The device MUST present at least one common configuration capability.

The device MUST present the feature bits it is offering in \field{device_feature}, starting at bit \field{device_feature_select} $*$ 32 for any \field{device_feature_select} written by the driver.
\begin{note}
  This means that it will present 0 for any \field{device_feature_select} other than 0 or 1, since no feature defined here exceeds 63.
\end{note}

The device MUST present any valid feature bits the driver has written in \field{driver_feature}, starting at bit \field{driver_feature_select} $*$ 32 for any \field{driver_feature_select} written by the driver.  Valid feature bits are those which are subset of the corresponding \field{device_feature} bits.  The device MAY present invalid bits written by the driver.

\begin{note}
  This means that a device can ignore writes for feature bits it never
  offers, and simply present 0 on reads.  Or it can just mirror what the driver wrote
  (but it will still have to check them when the driver sets FEATURES_OK).
\end{note}

\begin{note}
  A driver shouldn't write invalid bits anyway, as per \ref{drivernormative:General Initialization And Device Operation / Device Initialization}, but this attempts to handle it.
\end{note}

The device MUST present a changed \field{config_generation} after the
driver has read a device-specific configuration value which has
changed since any part of the device-specific configuration was last
read.
\begin{note}
As \field{config_generation} is an 8-bit value, simply incrementing it
on every configuration change could violate this requirement due to wrap.
Better would be to set an internal flag when it has changed,
and if that flag is set when the driver reads from the device-specific
configuration, increment \field{config_generation} and clear the flag.
\end{note}

The device MUST reset when 0 is written to \field{device_status}, and
present a 0 in \field{device_status} once that is done.

The device MUST present a 0 in \field{queue_enable} on reset.

If VIRTIO_F_RING_RESET has been negotiated, the device MUST present a 0 in
\field{queue_reset} on reset.

If VIRTIO_F_RING_RESET has been negotiated, the device MUST present a 0 in
\field{queue_reset} after the virtqueue is enabled with \field{queue_enable}.

The device MUST reset the queue when 1 is written to \field{queue_reset}. The
device MUST continue to present 1 in \field{queue_reset} as long as the queue reset
is ongoing. The device MUST present 0 in both \field{queue_reset} and \field{queue_enable}
when queue reset has completed.
(see \ref{sec:Basic Facilities of a Virtio Device / Virtqueues / Virtqueue Reset}).

The device MUST present a 0 in \field{queue_size} if the virtqueue
corresponding to the current \field{queue_select} is unavailable.

If VIRTIO_F_RING_PACKED has not been negotiated, the device MUST
present either a value of 0 or a power of 2 in
\field{queue_size}.

If VIRTIO_F_ADMIN_VQ has been negotiated, the value
\field{admin_queue_index} MUST be equal to, or bigger than
\field{num_queues}; also, \field{admin_queue_num} MUST be
smaller than, or equal to 0x10000 - \field{admin_queue_index},
to ensure that indices of valid admin queues fit into
a 16 bit range beyond all other virtqueues.

\drivernormative{\paragraph}{Common configuration structure layout}{Virtio Transport Options / Virtio Over PCI Bus / PCI Device Layout / Common configuration structure layout}

The driver MUST NOT write to \field{device_feature}, \field{num_queues},
\field{config_generation}, \field{queue_notify_off} or
\field{queue_notif_config_data}.

If VIRTIO_F_RING_PACKED has been negotiated,
the driver MUST NOT write the value 0 to \field{queue_size}.
If VIRTIO_F_RING_PACKED has not been negotiated,
the driver MUST NOT write a value which is not a power of 2 to \field{queue_size}.

The driver MUST configure the other virtqueue fields before enabling the virtqueue
with \field{queue_enable}.

After writing 0 to \field{device_status}, the driver MUST wait for a read of
\field{device_status} to return 0 before reinitializing the device.

The driver MUST NOT write a 0 to \field{queue_enable}.

If VIRTIO_F_RING_RESET has been negotiated, after the driver writes 1 to
\field{queue_reset} to reset the queue, the driver MUST NOT consider queue
reset to be complete until it reads back 0 in \field{queue_reset}. The driver
MAY re-enable the queue by writing 1 to \field{queue_enable} after ensuring
that other virtqueue fields have been set up correctly. The driver MAY set
driver-writeable queue configuration values to different values than those that
were used before the queue reset.
(see \ref{sec:Basic Facilities of a Virtio Device / Virtqueues / Virtqueue Reset}).

If VIRTIO_F_ADMIN_VQ has been negotiated, and if the driver
configures any administration virtqueues, the driver MUST
configure the administration virtqueues using the index
in the range \field{admin_queue_index} to
\field{admin_queue_index} + \field{admin_queue_num} - 1 inclusive.
The driver MAY configure fewer administration virtqueues than
supported by the device.

\subsubsection{Notification structure layout}\label{sec:Virtio Transport Options / Virtio Over PCI Bus / PCI Device Layout / Notification capability}

The notification location is found using the VIRTIO_PCI_CAP_NOTIFY_CFG
capability.  This capability is immediately followed by an additional
field, like so:

\begin{lstlisting}
struct virtio_pci_notify_cap {
        struct virtio_pci_cap cap;
        le32 notify_off_multiplier; /* Multiplier for queue_notify_off. */
};
\end{lstlisting}

\field{notify_off_multiplier} is combined with the \field{queue_notify_off} to
derive the Queue Notify address within a BAR for a virtqueue:

\begin{lstlisting}
        cap.offset + queue_notify_off * notify_off_multiplier
\end{lstlisting}

The \field{cap.offset} and \field{notify_off_multiplier} are taken from the
notification capability structure above, and the \field{queue_notify_off} is
taken from the common configuration structure.

\begin{note}
For example, if \field{notifier_off_multiplier} is 0, the device uses
the same Queue Notify address for all queues.
\end{note}

\devicenormative{\paragraph}{Notification capability}{Virtio Transport Options / Virtio Over PCI Bus / PCI Device Layout / Notification capability}
The device MUST present at least one notification capability.

For devices not offering VIRTIO_F_NOTIFICATION_DATA:

The \field{cap.offset} MUST be 2-byte aligned.

The device MUST either present \field{notify_off_multiplier} as an even power of 2,
or present \field{notify_off_multiplier} as 0.

The value \field{cap.length} presented by the device MUST be at least 2
and MUST be large enough to support queue notification offsets
for all supported queues in all possible configurations.

For all queues, the value \field{cap.length} presented by the device MUST satisfy:
\begin{lstlisting}
cap.length >= queue_notify_off * notify_off_multiplier + 2
\end{lstlisting}

For devices offering VIRTIO_F_NOTIFICATION_DATA:

The device MUST either present \field{notify_off_multiplier} as a
number that is a power of 2 that is also a multiple 4,
or present \field{notify_off_multiplier} as 0.

The \field{cap.offset} MUST be 4-byte aligned.

The value \field{cap.length} presented by the device MUST be at least 4
and MUST be large enough to support queue notification offsets
for all supported queues in all possible configurations.

For all queues, the value \field{cap.length} presented by the device MUST satisfy:
\begin{lstlisting}
cap.length >= queue_notify_off * notify_off_multiplier + 4
\end{lstlisting}

\subsubsection{ISR status capability}\label{sec:Virtio Transport Options / Virtio Over PCI Bus / PCI Device Layout / ISR status capability}

The VIRTIO_PCI_CAP_ISR_CFG capability
refers to at least a single byte, which contains the 8-bit ISR status field
to be used for INT\#x interrupt handling.

The \field{offset} for the \field{ISR status} has no alignment requirements.

The ISR bits allow the driver to distinguish between device-specific configuration
change interrupts and normal virtqueue interrupts:

\begin{tabular}{ |l||l|l|l| }
\hline
Bits       & 0                               & 1               &  2 to 31 \\
\hline
Purpose    & Queue Interrupt  & Device Configuration Interrupt & Reserved \\
\hline
\end{tabular}

To avoid an extra access, simply reading this register resets it to 0 and
causes the device to de-assert the interrupt.

In this way, driver read of ISR status causes the device to de-assert
an interrupt.

See sections \ref{sec:Virtio Transport Options / Virtio Over PCI Bus / PCI-specific Initialization And Device Operation / Used Buffer Notifications} and \ref{sec:Virtio Transport Options / Virtio Over PCI Bus / PCI-specific Initialization And Device Operation / Notification of Device Configuration Changes} for how this is used.

\devicenormative{\paragraph}{ISR status capability}{Virtio Transport Options / Virtio Over PCI Bus / PCI Device Layout / ISR status capability}

The device MUST present at least one VIRTIO_PCI_CAP_ISR_CFG capability.

The device MUST set the Device Configuration Interrupt bit
in \field{ISR status} before sending a device configuration
change notification to the driver.

If MSI-X capability is disabled, the device MUST set the Queue
Interrupt bit in \field{ISR status} before sending a virtqueue
notification to the driver.

If MSI-X capability is disabled, the device MUST set the Interrupt Status
bit in the PCI Status register in the PCI Configuration Header of
the device to the logical OR of all bits in \field{ISR status} of
the device.  The device then asserts/deasserts INT\#x interrupts unless masked
according to standard PCI rules \hyperref[intro:PCI]{[PCI]}.

The device MUST reset \field{ISR status} to 0 on driver read.

\drivernormative{\paragraph}{ISR status capability}{Virtio Transport Options / Virtio Over PCI Bus / PCI Device Layout / ISR status capability}

If MSI-X capability is enabled, the driver SHOULD NOT access
\field{ISR status} upon detecting a Queue Interrupt.

\subsubsection{Device-specific configuration}\label{sec:Virtio Transport Options / Virtio Over PCI Bus / PCI Device Layout / Device-specific configuration}

The device MUST present at least one VIRTIO_PCI_CAP_DEVICE_CFG capability for
any device type which has a device-specific configuration.

\devicenormative{\paragraph}{Device-specific configuration}{Virtio Transport Options / Virtio Over PCI Bus / PCI Device Layout / Device-specific configuration}

The \field{offset} for the device-specific configuration MUST be 4-byte aligned.

\subsubsection{Shared memory capability}\label{sec:Virtio Transport Options / Virtio Over PCI Bus / PCI Device Layout / Shared memory capability}

Shared memory regions \ref{sec:Basic Facilities of a Virtio
Device / Shared Memory Regions} are enumerated on the PCI transport
as a sequence of VIRTIO_PCI_CAP_SHARED_MEMORY_CFG capabilities, one per region.

The capability is defined by a struct virtio_pci_cap64 and
utilises the \field{cap.id} to allow multiple shared memory
regions per device.
The identifier in \field{cap.id} does not denote a certain order of
preference; it is only used to uniquely identify a region.

\devicenormative{\paragraph}{Shared memory capability}{Virtio Transport Options / Virtio Over PCI Bus / PCI Device Layout / Shared memory capability}

The region defined by the combination of the \field{cap.offset},
\field{offset_hi}, and \field{cap.length}, \field{length_hi}
fields MUST be contained within the BAR specified by
\field{cap.bar}.

The \field{cap.id} MUST be unique for any one device instance.

\subsubsection{Vendor data capability}\label{sec:Virtio
Transport Options / Virtio Over PCI Bus / PCI Device Layout /
Vendor data capability}

The optional Vendor data capability allows the device to present
vendor-specific data to the driver, without
conflicts, for debugging and/or reporting purposes,
and without conflicting with standard functionality.

This capability augments but does not replace the standard
subsystem ID and subsystem vendor ID fields
(offsets 0x2C and 0x2E in the PCI configuration space header)
as specified by \hyperref[intro:PCI]{[PCI]}.

Vendor data capability is enumerated on the PCI transport
as a VIRTIO_PCI_CAP_VENDOR_CFG capability.

The capability has the following structure:
\begin{lstlisting}
struct virtio_pci_vndr_data {
        u8 cap_vndr;    /* Generic PCI field: PCI_CAP_ID_VNDR */
        u8 cap_next;    /* Generic PCI field: next ptr. */
        u8 cap_len;     /* Generic PCI field: capability length */
        u8 cfg_type;    /* Identifies the structure. */
        u16 vendor_id;  /* Identifies the vendor-specific format. */
	/* For Vendor Definition */
	/* Pads structure to a multiple of 4 bytes */
	/* Reads must not have side effects */
};
\end{lstlisting}

Where \field{vendor_id} identifies the PCI-SIG assigned Vendor ID
as specified by \hyperref[intro:PCI]{[PCI]}.

Note that the capability size is required to be a multiple of 4.

To make it safe for a generic driver to access the capability,
reads from this capability MUST NOT have any side effects.

\devicenormative{\paragraph}{Vendor data capability}{Virtio
Transport Options / Virtio Over PCI Bus / PCI Device Layout /
Vendor data capability}

Devices CAN present \field{vendor_id} that does not match
either the PCI Vendor ID or the PCI Subsystem Vendor ID.

Devices CAN present multiple Vendor data capabilities with
either different or identical \field{vendor_id} values.

The value \field{vendor_id} MUST NOT equal 0x1AF4.

The size of the Vendor data capability MUST be a multiple of 4 bytes.

Reads of the Vendor data capability by the driver MUST NOT have any
side effects.

\drivernormative{\paragraph}{Vendor data capability}{Virtio
Transport Options / Virtio Over PCI Bus / PCI Device Layout /
Vendor data capability}

The driver SHOULD NOT use the Vendor data capability except
for debugging and reporting purposes.

The driver MUST qualify the \field{vendor_id} before
interpreting or writing into the Vendor data capability.

\subsubsection{PCI configuration access capability}\label{sec:Virtio Transport Options / Virtio Over PCI Bus / PCI Device Layout / PCI configuration access capability}

The VIRTIO_PCI_CAP_PCI_CFG capability
creates an alternative (and likely suboptimal) access method to the
common configuration, notification, ISR and device-specific configuration regions.

The capability is immediately followed by an additional field like so:

\begin{lstlisting}
struct virtio_pci_cfg_cap {
        struct virtio_pci_cap cap;
        u8 pci_cfg_data[4]; /* Data for BAR access. */
};
\end{lstlisting}

The fields \field{cap.bar}, \field{cap.length}, \field{cap.offset} and
\field{pci_cfg_data} are read-write (RW) for the driver.

To access a device region, the driver writes into the capability
structure (ie. within the PCI configuration space) as follows:

\begin{itemize}
\item The driver sets the BAR to access by writing to \field{cap.bar}.

\item The driver sets the size of the access by writing 1, 2 or 4 to
  \field{cap.length}.

\item The driver sets the offset within the BAR by writing to
  \field{cap.offset}.
\end{itemize}

At that point, \field{pci_cfg_data} will provide a window of size
\field{cap.length} into the given \field{cap.bar} at offset \field{cap.offset}.

\devicenormative{\paragraph}{PCI configuration access capability}{Virtio Transport Options / Virtio Over PCI Bus / PCI Device Layout / PCI configuration access capability}

The device MUST present at least one VIRTIO_PCI_CAP_PCI_CFG capability.

Upon detecting driver write access
to \field{pci_cfg_data}, the device MUST execute a write access
at offset \field{cap.offset} at BAR selected by \field{cap.bar} using the first \field{cap.length}
bytes from \field{pci_cfg_data}.

Upon detecting driver read access
to \field{pci_cfg_data}, the device MUST
execute a read access of length cap.length at offset \field{cap.offset}
at BAR selected by \field{cap.bar} and store the first \field{cap.length} bytes in
\field{pci_cfg_data}.

\drivernormative{\paragraph}{PCI configuration access capability}{Virtio Transport Options / Virtio Over PCI Bus / PCI Device Layout / PCI configuration access capability}

The driver MUST NOT write a \field{cap.offset} which is not
a multiple of \field{cap.length} (ie. all accesses MUST be aligned).

The driver MUST NOT read or write \field{pci_cfg_data}
unless \field{cap.bar}, \field{cap.length} and \field{cap.offset}
address \field{cap.length} bytes within a BAR range
specified by some other Virtio Structure PCI Capability
of type other than \field{VIRTIO_PCI_CAP_PCI_CFG}.

\subsubsection{Legacy Interfaces: A Note on PCI Device Layout}\label{sec:Virtio Transport Options / Virtio Over PCI Bus / PCI Device Layout / Legacy Interfaces: A Note on PCI Device Layout}

Transitional devices MUST present part of configuration
registers in a legacy configuration structure in BAR0 in the first I/O
region of the PCI device, as documented below.
When using the legacy interface, transitional drivers
MUST use the legacy configuration structure in BAR0 in the first
I/O region of the PCI device, as documented below.

When using the legacy interface the driver MAY access
the device-specific configuration region using any width accesses, and
a transitional device MUST present driver with the same results as
when accessed using the ``natural'' access method (i.e.
32-bit accesses for 32-bit fields, etc).

Note that this is possible because while the virtio common configuration structure is PCI
(i.e. little) endian, when using the legacy interface the device-specific
configuration region is encoded in the native endian of the guest (where such distinction is
applicable).

When used through the legacy interface, the virtio common configuration structure looks as follows:

\begin{tabularx}{\textwidth}{ |X||X|X|X|X|X|X|X|X| }
\hline
 Bits & 32 & 32 & 32 & 16 & 16 & 16 & 8 & 8 \\
\hline
 Read / Write & R & R+W & R+W & R & R+W & R+W & R+W & R \\
\hline
 Purpose & Device Features bits 0:31 & Driver Features bits 0:31 &
  Queue Address & \field{queue_size} & \field{queue_select} & Queue Notify &
  Device Status & ISR \newline Status \\
\hline
\end{tabularx}

If MSI-X is enabled for the device, two additional fields
immediately follow this header:

\begin{tabular}{ |l||l|l| }
\hline
Bits       & 16             & 16     \\
\hline
Read/Write & R+W            & R+W    \\
\hline
Purpose (MSI-X) & \field{config_msix_vector}  & \field{queue_msix_vector} \\
\hline
\end{tabular}

Note: When MSI-X capability is enabled, device-specific configuration starts at
byte offset 24 in virtio common configuration structure. When MSI-X capability is not
enabled, device-specific configuration starts at byte offset 20 in virtio
header.  ie. once you enable MSI-X on the device, the other fields move.
If you turn it off again, they move back!

Any device-specific configuration space immediately follows
these general headers:

\begin{tabular}{|l||l|l|}
\hline
Bits & Device Specific & \multirow{3}{*}{\ldots} \\
\cline{1-2}
Read / Write & Device Specific & \\
\cline{1-2}
Purpose & Device Specific & \\
\hline
\end{tabular}

When accessing the device-specific configuration space
using the legacy interface, transitional
drivers MUST access the device-specific configuration space
at an offset immediately following the general headers.

When using the legacy interface, transitional
devices MUST present the device-specific configuration space
if any at an offset immediately following the general headers.

Note that only Feature Bits 0 to 31 are accessible through the
Legacy Interface. When used through the Legacy Interface,
Transitional Devices MUST assume that Feature Bits 32 to 63
are not acknowledged by Driver.

As legacy devices had no \field{config_generation} field,
see \ref{sec:Basic Facilities of a Virtio Device / Device
Configuration Space / Legacy Interface: Device Configuration
Space}~\nameref{sec:Basic Facilities of a Virtio Device / Device Configuration Space / Legacy Interface: Device Configuration Space} for workarounds.

\subsubsection{Non-transitional Device With Legacy Driver: A Note
on PCI Device Layout}\label{sec:Virtio Transport Options / Virtio
Over PCI Bus / PCI Device Layout / Non-transitional Device With
Legacy Driver: A Note on PCI Device Layout}

All known legacy drivers check either the PCI Revision or the
Device and Vendor IDs, and thus won't attempt to drive a
non-transitional device.

A buggy legacy driver might mistakenly attempt to drive a
non-transitional device. If support for such drivers is required
(as opposed to fixing the bug), the following would be the
recommended way to detect and handle them.
\begin{note}
Such buggy drivers are not currently known to be used in
production.
\end{note}

\subparagraph{Device Requirements: Non-transitional Device With Legacy Driver}
\label{drivernormative:Virtio Transport Options / Virtio Over PCI
Bus / PCI-specific Initialization And Device Operation /
Device Initialization / Non-transitional Device With Legacy
Driver}
\label{devicenormative:Virtio Transport Options / Virtio Over PCI
Bus / PCI-specific Initialization And Device Operation /
Device Initialization / Non-transitional Device With Legacy
Driver}

Non-transitional devices, on a platform where a legacy driver for
a legacy device with the same ID (including PCI Revision, Device
and Vendor IDs) is known to have previously existed,
SHOULD take the following steps to cause the legacy driver to
fail gracefully when it attempts to drive them:

\begin{enumerate}
\item Present an I/O BAR in BAR0, and
\item Respond to a single-byte zero write to offset 18
   (corresponding to Device Status register in the legacy layout)
   of BAR0 by presenting zeroes on every BAR and ignoring writes.
\end{enumerate}

\subsection{PCI-specific Initialization And Device Operation}\label{sec:Virtio Transport Options / Virtio Over PCI Bus / PCI-specific Initialization And Device Operation}

\subsubsection{Device Initialization}\label{sec:Virtio Transport Options / Virtio Over PCI Bus / PCI-specific Initialization And Device Operation / Device Initialization}

This documents PCI-specific steps executed during Device Initialization.

\paragraph{Virtio Device Configuration Layout Detection}\label{sec:Virtio Transport Options / Virtio Over PCI Bus / PCI-specific Initialization And Device Operation / Device Initialization / Virtio Device Configuration Layout Detection}

As a prerequisite to device initialization, the driver scans the
PCI capability list, detecting virtio configuration layout using Virtio
Structure PCI capabilities as detailed in \ref{sec:Virtio Transport Options / Virtio Over PCI Bus / Virtio Structure PCI Capabilities}

\subparagraph{Legacy Interface: A Note on Device Layout Detection}\label{sec:Virtio Transport Options / Virtio Over PCI Bus / PCI-specific Initialization And Device Operation / Device Initialization / Virtio Device Configuration Layout Detection / Legacy Interface: A Note on Device Layout Detection}

Legacy drivers skipped the Device Layout Detection step, assuming legacy
device configuration space in BAR0 in I/O space unconditionally.

Legacy devices did not have the Virtio PCI Capability in their
capability list.

Therefore:

Transitional devices MUST expose the Legacy Interface in I/O
space in BAR0.

Transitional drivers MUST look for the Virtio PCI
Capabilities on the capability list.
If these are not present, driver MUST assume a legacy device,
and use it through the legacy interface.

Non-transitional drivers MUST look for the Virtio PCI
Capabilities on the capability list.
If these are not present, driver MUST assume a legacy device,
and fail gracefully.

\paragraph{MSI-X Vector Configuration}\label{sec:Virtio Transport Options / Virtio Over PCI Bus / PCI-specific Initialization And Device Operation / Device Initialization / MSI-X Vector Configuration}

When MSI-X capability is present and enabled in the device
(through standard PCI configuration space) \field{config_msix_vector} and \field{queue_msix_vector} are used to map configuration change and queue
interrupts to MSI-X vectors. In this case, the ISR Status is unused.

Writing a valid MSI-X Table entry number, 0 to 0x7FF, to
\field{config_msix_vector}/\field{queue_msix_vector} maps interrupts triggered
by the configuration change/selected queue events respectively to
the corresponding MSI-X vector. To disable interrupts for an
event type, the driver unmaps this event by writing a special NO_VECTOR
value:

\begin{lstlisting}
/* Vector value used to disable MSI for queue */
#define VIRTIO_MSI_NO_VECTOR            0xffff
\end{lstlisting}

Note that mapping an event to vector might require device to
allocate internal device resources, and thus could fail.

\devicenormative{\subparagraph}{MSI-X Vector Configuration}{Virtio Transport Options / Virtio Over PCI Bus / PCI-specific Initialization And Device Operation / Device Initialization / MSI-X Vector Configuration}

A device that has an MSI-X capability SHOULD support at least 2
and at most 0x800 MSI-X vectors.
Device MUST report the number of vectors supported in
\field{Table Size} in the MSI-X Capability as specified in
\hyperref[intro:PCI]{[PCI]}.
The device SHOULD restrict the reported MSI-X Table Size field
to a value that might benefit system performance.
\begin{note}
For example, a device which does not expect to send
interrupts at a high rate might only specify 2 MSI-X vectors.
\end{note}
Device MUST support mapping any event type to any valid
vector 0 to MSI-X \field{Table Size}.
Device MUST support unmapping any event type.

The device MUST return vector mapped to a given event,
(NO_VECTOR if unmapped) on read of \field{config_msix_vector}/\field{queue_msix_vector}.
The device MUST have all queue and configuration change
events are unmapped upon reset.

Devices SHOULD NOT cause mapping an event to vector to fail
unless it is impossible for the device to satisfy the mapping
request.  Devices MUST report mapping
failures by returning the NO_VECTOR value when the relevant
\field{config_msix_vector}/\field{queue_msix_vector} field is read.

\drivernormative{\subparagraph}{MSI-X Vector Configuration}{Virtio Transport Options / Virtio Over PCI Bus / PCI-specific Initialization And Device Operation / Device Initialization / MSI-X Vector Configuration}

Driver MUST support device with any MSI-X Table Size 0 to 0x7FF.
Driver MAY fall back on using INT\#x interrupts for a device
which only supports one MSI-X vector (MSI-X Table Size = 0).

Driver MAY interpret the Table Size as a hint from the device
for the suggested number of MSI-X vectors to use.

Driver MUST NOT attempt to map an event to a vector
outside the MSI-X Table supported by the device,
as reported by \field{Table Size} in the MSI-X Capability.

After mapping an event to vector, the
driver MUST verify success by reading the Vector field value: on
success, the previously written value is returned, and on
failure, NO_VECTOR is returned. If a mapping failure is detected,
the driver MAY retry mapping with fewer vectors, disable MSI-X
or report device failure.

\paragraph{Virtqueue Configuration}\label{sec:Virtio Transport Options / Virtio Over PCI Bus / PCI-specific Initialization And Device Operation / Device Initialization / Virtqueue Configuration}

As a device can have zero or more virtqueues for bulk data
transport\footnote{For example, the simplest network device has two virtqueues.}, the driver
needs to configure them as part of the device-specific
configuration.

The driver typically does this as follows, for each virtqueue a device has:

\begin{enumerate}
\item Write the virtqueue index (first queue is 0) to \field{queue_select}.

\item Read the virtqueue size from \field{queue_size}. This controls how big the virtqueue is
  (see \ref{sec:Basic Facilities of a Virtio Device / Virtqueues}~\nameref{sec:Basic Facilities of a Virtio Device / Virtqueues}). If this field is 0, the virtqueue does not exist.

\item Optionally, select a smaller virtqueue size and write it to \field{queue_size}.

\item Allocate and zero Descriptor Table, Available and Used rings for the
   virtqueue in contiguous physical memory.

\item Optionally, if MSI-X capability is present and enabled on the
  device, select a vector to use to request interrupts triggered
  by virtqueue events. Write the MSI-X Table entry number
  corresponding to this vector into \field{queue_msix_vector}. Read
  \field{queue_msix_vector}: on success, previously written value is
  returned; on failure, NO_VECTOR value is returned.
\end{enumerate}

\subparagraph{Legacy Interface: A Note on Virtqueue Configuration}\label{sec:Virtio Transport Options / Virtio Over PCI Bus / PCI-specific Initialization And Device Operation / Device Initialization / Virtqueue Configuration / Legacy Interface: A Note on Virtqueue Configuration}
When using the legacy interface, the queue layout follows \ref{sec:Basic Facilities of a Virtio Device / Virtqueues / Legacy Interfaces: A Note on Virtqueue Layout}~\nameref{sec:Basic Facilities of a Virtio Device / Virtqueues / Legacy Interfaces: A Note on Virtqueue Layout} with an alignment of 4096.
Driver writes the physical address, divided
by 4096 to the Queue Address field\footnote{The 4096 is based on the x86 page size, but it's also large
enough to ensure that the separate parts of the virtqueue are on
separate cache lines.
}.  There was no mechanism to negotiate the queue size.

\subsubsection{Available Buffer Notifications}\label{sec:Virtio Transport Options / Virtio Over PCI Bus / PCI-specific Initialization And Device Operation / Available Buffer Notifications}

When VIRTIO_F_NOTIFICATION_DATA has not been negotiated,
the driver sends an available buffer notification to the device by writing
only the 16-bit notification value to the Queue Notify address of the
virtqueue. A notification value depends on the negotiation of
VIRTIO_F_NOTIF_CONFIG_DATA.

If VIRTIO_F_NOTIFICATION_DATA has been negotiated, the driver sends an
available buffer notification to the device by writing the following 32-bit
value to the Queue Notify address:
\lstinputlisting{notifications-data-le.c}

\begin{itemize}
\item When VIRTIO_F_NOTIF_CONFIG_DATA is not negotiated \field{vq_index} is set
to the virtqueue index.

\item When VIRTIO_F_NOTIFICATION_DATA is negotiated,
\field{vq_notif_config_data} is set to \field{queue_notif_config_data}.
\end{itemize}

See \ref{sec:Basic Facilities of a Virtio Device / Driver notifications}~\nameref{sec:Basic Facilities of a Virtio Device / Driver notifications}
for the definition of the components.

See \ref{sec:Virtio Transport Options / Virtio Over PCI Bus / PCI Device Layout / Notification capability}
for how to calculate the Queue Notify address.

\drivernormative{\paragraph}{Available Buffer Notifications}{Virtio Transport Options / Virtio Over PCI Bus / PCI-specific Initialization And Device Operation / Available Buffer Notifications}

If VIRTIO_F_NOTIFICATION_DATA is not negotiated, the driver notification
MUST be a 16-bit notification.

If VIRTIO_F_NOTIFICATION_DATA is negotiated, the driver notification
MUST be a 32-bit notification.

If VIRTIO_F_NOTIF_CONFIG_DATA is not negotiated:
\begin{itemize}
\item If VIRTIO_F_NOTIFICATION_DATA is not negotiated, the driver MUST set the
notification value to the virtqueue index.

\item If VIRTIO_F_NOTIFICATION_DATA is negotiated, the driver MUST set the
\field{vq_index} to the virtqueue index.

\end{itemize}

If VIRTIO_F_NOTIF_CONFIG_DATA is negotiated:
\begin{itemize}
\item If VIRTIO_F_NOTIFICATION_DATA is not negotiated, the driver MUST set
the notification value to \field{queue_notif_config_data}.

\item If VIRTIO_F_NOTIFICATION_DATA is negotiated, the driver MUST set the
\field{vq_notify_config_data} to the \field{queue_notif_config_data} value.
\end{itemize}

\subsubsection{Used Buffer Notifications}\label{sec:Virtio Transport Options / Virtio Over PCI Bus / PCI-specific Initialization And Device Operation / Used Buffer Notifications}

If a used buffer notification is necessary for a virtqueue, the device would typically act as follows:

\begin{itemize}
  \item If MSI-X capability is disabled:
    \begin{enumerate}
    \item Set the lower bit of the ISR Status field for the device.

    \item Send the appropriate PCI interrupt for the device.
    \end{enumerate}

  \item If MSI-X capability is enabled:
    \begin{enumerate}
    \item If \field{queue_msix_vector} is not NO_VECTOR,
      request the appropriate MSI-X interrupt message for the
      device, \field{queue_msix_vector} sets the MSI-X Table entry
      number.
    \end{enumerate}
\end{itemize}

\devicenormative{\paragraph}{Used Buffer Notifications}{Virtio Transport Options / Virtio Over PCI Bus / PCI-specific Initialization And Device Operation / Used Buffer Notifications}

If MSI-X capability is enabled and \field{queue_msix_vector} is
NO_VECTOR for a virtqueue, the device MUST NOT deliver an interrupt
for that virtqueue.

\subsubsection{Notification of Device Configuration Changes}\label{sec:Virtio Transport Options / Virtio Over PCI Bus / PCI-specific Initialization And Device Operation / Notification of Device Configuration Changes}

Some virtio PCI devices can change the device configuration
state, as reflected in the device-specific configuration region of the device. In this case:

\begin{itemize}
  \item If MSI-X capability is disabled:
    \begin{enumerate}
    \item Set the second lower bit of the ISR Status field for the device.

    \item Send the appropriate PCI interrupt for the device.
    \end{enumerate}

  \item If MSI-X capability is enabled:
    \begin{enumerate}
    \item If \field{config_msix_vector} is not NO_VECTOR,
      request the appropriate MSI-X interrupt message for the
      device, \field{config_msix_vector} sets the MSI-X Table entry
      number.
    \end{enumerate}
\end{itemize}

A single interrupt MAY indicate both that one or more virtqueue has
been used and that the configuration space has changed.

\devicenormative{\paragraph}{Notification of Device Configuration Changes}{Virtio Transport Options / Virtio Over PCI Bus / PCI-specific Initialization And Device Operation / Notification of Device Configuration Changes}

If MSI-X capability is enabled and \field{config_msix_vector} is
NO_VECTOR, the device MUST NOT deliver an interrupt
for device configuration space changes.

\drivernormative{\paragraph}{Notification of Device Configuration Changes}{Virtio Transport Options / Virtio Over PCI Bus / PCI-specific Initialization And Device Operation / Notification of Device Configuration Changes}

A driver MUST handle the case where the same interrupt is used to indicate
both device configuration space change and one or more virtqueues being used.

\subsubsection{Driver Handling Interrupts}\label{sec:Virtio Transport Options / Virtio Over PCI Bus / PCI-specific Initialization And Device Operation / Driver Handling Interrupts}
The driver interrupt handler would typically:

\begin{itemize}
  \item If MSI-X capability is disabled:
    \begin{itemize}
      \item Read the ISR Status field, which will reset it to zero.
      \item If the lower bit is set:
        look through all virtqueues for the
        device, to see if any progress has been made by the device
        which requires servicing.
      \item If the second lower bit is set:
        re-examine the configuration space to see what changed.
    \end{itemize}
  \item If MSI-X capability is enabled:
    \begin{itemize}
      \item
        Look through all virtqueues mapped to that MSI-X vector for the
        device, to see if any progress has been made by the device
        which requires servicing.
      \item
        If the MSI-X vector is equal to \field{config_msix_vector},
        re-examine the configuration space to see what changed.
    \end{itemize}
\end{itemize}
