\subsubsection{Legacy Interfaces}\label{sec:Basic Facilities of a Virtio Device / Device groups / Group
administration commands / Legacy Interface}

In some systems, there is a need to support utilizing a legacy driver with
a device that does not directly support the legacy interface. In such scenarios,
a group owner device can provide the legacy interface functionality for the
group member devices. The driver of the owner device can then access the legacy
interface of a member device on behalf of the legacy member device driver.

For example, with the SR-IOV group type, group members (VFs) can not present
the legacy interface in an I/O BAR in BAR0 as expected by the legacy pci driver.
If the legacy driver is running inside a virtual machine, the hypervisor
executing the virtual machine can present a virtual device with an I/O BAR in
BAR0. The hypervisor intercepts the legacy driver accesses to this I/O BAR and
forwards them to the group owner device (PF) using group administration commands.

The following commands support such a legacy interface functionality:

\begin{enumerate}
\item Legacy Common Configuration Write Command
\item Legacy Common Configuration Read Command
\item Legacy Device Configuration Write Command
\item Legacy Device Configuration Read Command
\end{enumerate}

These commands are currently only defined for the SR-IOV group type and
have, generally, the same effect as member device accesses through a legacy
interface listed in section \ref{sec:Virtio Transport Options / Virtio Over PCI
Bus / PCI Device Layout / Legacy Interfaces: A Note on PCI Device Layout} except
that little-endian format is assumed unconditionally.

\paragraph{Legacy Common Configuration Write Command}\label{par:Basic Facilities of a Virtio Device / Device groups / Group
administration commands / Legacy Interface / Legacy Common Configuration Write Command}

This command has the same effect as writing into the virtio common configuration
structure through the legacy interface. The \field{command_specific_data} is in
the format \field{struct virtio_admin_cmd_legacy_common_cfg_wr_data} describing
the access to be performed.

\begin{lstlisting}
struct virtio_admin_cmd_legacy_common_cfg_wr_data {
        u8 offset; /* Starting byte offset within the common configuration structure to write */
        u8 reserved[7];
        u8 data[];
};
\end{lstlisting}

For the command VIRTIO_ADMIN_CMD_LEGACY_COMMON_CFG_WRITE, \field{opcode}
is set to 0x2.
The \field{group_member_id} refers to the member device to be accessed.
The \field{offset} refers to the offset for the write within the virtio common
configuration structure, and excluding the device-specific configuration.
The length of the data to write is simply the length of \field{data}.

No length or alignment restrictions are placed on the value of the
\field{offset} and the length of the \field{data}, except that the resulting
access refers to a single field and is completely within the virtio common
configuration structure, excluding the device-specific configuration.

This command has no command specific result.

\paragraph{Legacy Common Configuration Read Command}\label{par:Basic Facilities of a Virtio Device / Device groups / Group administration commands / Legacy Interface / Legacy Common Configuration Read Command}

This command has the same effect as reading from the virtio common configuration
structure through the legacy interface. The \field{command_specific_data} is in
the format \field{struct virtio_admin_cmd_legacy_common_cfg_rd_data} describing
the access to be performed.

\begin{lstlisting}
struct virtio_admin_cmd_legacy_common_cfg_rd_data {
        u8 offset; /* Starting byte offset within the common configuration structure to read */
};
\end{lstlisting}

For the command VIRTIO_ADMIN_CMD_LEGACY_COMMON_CFG_READ, \field{opcode}
is set to 0x3.
The \field{group_member_id} refers to the member device to be accessed.
The \field{offset} refers to the offset for the read from the virtio common
configuration structure, and excluding the device-specific configuration.

\begin{lstlisting}
struct virtio_admin_cmd_legacy_common_cfg_rd_result {
        u8 data[];
};
\end{lstlisting}

No length or alignment restrictions are placed on the value of the
\field{offset} and the length of the \field{data}, except that the resulting
access refers to a single field and is completely within the virtio common
configuration structure, excluding the device-specific configuration.

When the command completes successfully, \field{command_specific_result}
is in the format \field{struct virtio_admin_cmd_legacy_common_cfg_rd_result}
returned by the device. The length of the data read is simply the length of
\field{data}.

\paragraph{Legacy Device Configuration Write Command}\label{par:Basic Facilities of a Virtio Device / Device groups / Group administration commands / Legacy Interface / Legacy Device Configuration Write Command}

This command has the same effect as writing into the virtio device-specific
configuration through the legacy interface. The \field{command_specific_data} is in
the format \field{struct virtio_admin_cmd_legacy_dev_reg_wr_data} describing
the access to be performed.

\begin{lstlisting}
struct virtio_admin_cmd_legacy_dev_reg_wr_data {
        u8 offset; /* Starting byte offset within the device-specific configuration to write */
        u8 reserved[7];
        u8 data[];
};
\end{lstlisting}

For the command VIRTIO_ADMIN_CMD_LEGACY_DEV_CFG_WRITE, \field{opcode}
is set to 0x4.
The \field{group_member_id} refers to the member device to be accessed.
The \field{offset} refers to the offset for the write within the virtio
device-specific configuration. The length of the data to write is simply
the length of \field{data}.

No length or alignment restrictions are placed on the value of the
\field{offset} and the length of the \field{data}, except that the resulting
access refers to a single field and is completely within the device-specific
configuration.

This command has no command specific result.

\paragraph{Legacy Device Configuration Read Command}\label{par:Basic Facilities of a Virtio Device / Device groups / Group administration commands / Legacy Interface / Legacy Device Configuration Read Command}

This command has the same effect as reading from the virtio device-specific
configuration through the legacy interface. The \field{command_specific_data} is in
the format \field{struct virtio_admin_cmd_legacy_common_cfg_rd_data} describing
the access to be performed.

\begin{lstlisting}
struct virtio_admin_cmd_legacy_dev_cfg_rd_data {
        u8 offset; /* Starting byte offset within the device-specific configuration to read */
};
\end{lstlisting}

For the command VIRTIO_ADMIN_CMD_LEGACY_DEV_CFG_READ, \field{opcode}
is set to 0x5.
The \field{group_member_id} refers to the member device to be accessed.
The \field{offset} refers to the offset for the read from the virtio device-specific
configuration.

\begin{lstlisting}
struct virtio_admin_cmd_legacy_dev_reg_rd_result {
        u8 data[];
};
\end{lstlisting}

No length or alignment restrictions are placed on the value of the
\field{offset} and the length of the \field{data}, except that the resulting
access refers to a single field and is completely within the device-specific
configuration.

When the command completes successfully, \field{command_specific_result} is in
the format \field{struct virtio_admin_cmd_legacy_dev_reg_rd_result}
returned by the device.

The length of the data read is simply the length of \field{data}.

\paragraph{Legacy Driver Notification}\label{par:Basic Facilities of a Virtio Device / Device groups / Group administration commands / Legacy Interface / Legacy Driver Notifications}

The driver of the owner device can send a driver notification to the member
device operated using the legacy interface by executing
VIRTIO_ADMIN_CMD_LEGACY_COMMON_CFG_WRITE with the \field{offset} matching
\field{Queue Notify} and the \field{data} containing a 16-bit virtqueue index to
be notified.

However, as VIRTIO_ADMIN_CMD_LEGACY_COMMON_CFG_WRITE is also used for slow path
configuration a separate dedicated mechanism for sending such driver
notifications to the member device can be made available by the owner device.
For the SR-IOV group type, the optional command
VIRTIO_ADMIN_CMD_LEGACY_NOTIFY_INFO addresses this need by returning to the
driver one or more addresses which can be used to send such driver
notifications. The notification address returned can be in the device memory
(PCI BAR or VF BAR) of the device.

In this alternative approach, driver notifications are sent by
writing a 16-bit virtqueue index to be notified, in the little-endian
format, to the notification address returned by
the VIRTIO_ADMIN_CMD_LEGACY_NOTIFY_INFO command.

Any driver notification sent through the notification address has the same effect
as if it was sent using the VIRTIO_ADMIN_CMD_LEGACY_COMMON_CFG_WRITE command with
the \field{offset} matching \field{Queue Notify}.

This command is only defined for the SR-IOV group type.

For the command VIRTIO_ADMIN_CMD_LEGACY_NOTIFY_INFO, \field{opcode}
is set to 0x6.
The \field{group_member_id} refers to the member device to be accessed.
This command does not use \field{command_specific_data}.

When the device supports the VIRTIO_ADMIN_CMD_LEGACY_NOTIFY_INFO command, the
group owner device hardwires VF BAR0 to zero in the SR-IOV Extended capability.

\begin{lstlisting}
struct virtio_pci_legacy_notify_info {
        u8 flags;  /* 0 = end of list, 1 = owner device, 2 = member device */
        u8 bar;    /* BAR of the member or the owner device */
        u8 padding[6];
        le64 offset; /* Offset within bar. */
};

struct virtio_admin_cmd_legacy_notify_info_result {
        struct virtio_pci_legacy_notify_info entries[4];
};
\end{lstlisting}

A \field{flags} value of 0x1 indicates that the notification address is of
the owner device, the value of 0x2 indicates that the notification address is of
the member device and the value of 0x0 indicates that all the entries starting
from that entry are invalid entries in \field{entries}. All other values in
\field{flags} are reserved.

The \field{bar} values 0x1 to 0x5 specify BAR1 to BAR5 respectively:
when the \field{flags} is 0x1 this is specified by the Base Address Registers
in the PCI header of the device,
when the \field{flags} is 0x2 this is specified by the VF BARn
registers in the SR-IOV Extended Capability of the device.

The \field{offset} indicates the notification address relative to BAR indicated
in \field{bar}. This value is 2-byte aligned.

When the command completes successfully, \field{command_specific_result} is in
the format \field{struct virtio_admin_cmd_legacy_notify_info_result}. The
device can supply up to 4 entries each with a different notification
address. In this case, any of the entries can be used by the driver. The order
of the entries serves as a preference hint to the driver. The driver is expected
to utilize the entries placed earlier in the array in preference to the later
ones. The driver is also expected to ignore any invalid entries, as well as
the end of list entry if present and any entries following the end of list.

\devicenormative{\paragraph}{Legacy Interface}{Basic Facilities of a Virtio Device / Device groups / Group administration commands / Legacy Interface}

A device MUST either support all of, or none of
VIRTIO_ADMIN_CMD_LEGACY_COMMON_CFG_WRITE,
VIRTIO_ADMIN_CMD_LEGACY_COMMON_CFG_READ,
VIRTIO_ADMIN_CMD_LEGACY_DEV_CFG_WRITE and
VIRTIO_ADMIN_CMD_LEGACY_DEV_CFG_READ commands.

For VIRTIO_ADMIN_CMD_LEGACY_COMMON_CFG_WRITE,
VIRTIO_ADMIN_CMD_LEGACY_COMMON_CFG_READ,
VIRTIO_ADMIN_CMD_LEGACY_DEV_CFG_WRITE and
VIRTIO_ADMIN_CMD_LEGACY_DEV_CFG_READ commands,
the device MUST decode and encode (respectively) the value of the
\field{data} using the little-endian format.

For the VIRTIO_ADMIN_CMD_LEGACY_COMMON_CFG_WRITE and
VIRTIO_ADMIN_CMD_LEGACY_COMMON_CFG_READ commands, 
the device MUST fail the command when the value of the
\field{offset} and the length of the \field{data} do not refer to a
single field or are not completely within the virtio common configuration
excluding the device-specific configuration.

For the VIRTIO_ADMIN_CMD_LEGACY_DEV_CFG_WRITE and
VIRTIO_ADMIN_CMD_LEGACY_DEV_CFG_READ commands,
the device MUST fail the command when the value of the
\field{offset} and the length of the \field{data} do not refer to a
single field or are not completely within the virtio device-specific
configuration.

The command VIRTIO_ADMIN_CMD_LEGACY_COMMON_CFG_WRITE MUST have the same effect
as writing into the virtio common configuration structure through the legacy
interface.

The command VIRTIO_ADMIN_CMD_LEGACY_COMMON_CFG_READ MUST have the same effect as
reading from the virtio common configuration structure through the legacy
interface.

The command VIRTIO_ADMIN_CMD_LEGACY_DEV_CFG_WRITE MUST have the same effect as
writing into the virtio device-specific configuration through the legacy
interface.

The command VIRTIO_ADMIN_CMD_LEGACY_DEV_CFG_READ MUST have the same effect as
reading from the virtio device-specific configuration through the legacy
interface.

For the SR-IOV group type, when the owner device supports
VIRTIO_ADMIN_CMD_LEGACY_COMMON_CFG_READ,
VIRTIO_ADMIN_CMD_LEGACY_COMMON_CFG_WRITE, VIRTIO_ADMIN_CMD_LEGACY_DEV_CFG_READ,
VIRTIO_ADMIN_CMD_LEGACY_DEV_CFG_WRITE and VIRTIO_ADMIN_CMD_LEGACY_NOTIFY_INFO
commands,
\begin{itemize}
\item the owner device and the group member device SHOULD follow the rules
for the PCI Revision ID and Subsystem Device ID of the non-transitional devices
documented in section \ref{sec:Virtio Transport Options / Virtio Over PCI Bus / PCI Device Discovery}.

\item the owner device SHOULD follow the rules for the PCI Device ID of the non-transitional
devices documented in section
\ref{sec:Virtio Transport Options / Virtio Over PCI Bus / PCI Device Discovery}.

\item any driver notification received by the device at any of the notification
address supplied in the command result of
VIRTIO_ADMIN_CMD_LEGACY_NOTIFY_INFO MUST function as if the device received
the notification through VIRTIO_ADMIN_CMD_LEGACY_COMMON_CFG_WRITE
command at an offset \field{offset} matching \field{Queue Notify}.
\end{itemize}

If the device supports the VIRTIO_ADMIN_CMD_LEGACY_NOTIFY_INFO command,
\begin{itemize}
\item the device MUST also support all of VIRTIO_ADMIN_CMD_LEGACY_COMMON_CFG_WRITE,
VIRTIO_ADMIN_CMD_LEGACY_COMMON_CFG_READ,
VIRTIO_ADMIN_CMD_LEGACY_DEV_CFG_WRITE and
VIRTIO_ADMIN_CMD_LEGACY_DEV_CFG_READ commands.

\item in the command result of VIRTIO_ADMIN_CMD_LEGACY_NOTIFY_INFO, the last
\field{struct virtio_pci_legacy_notify_info} entry MUST have \field{flags} of
zero.

\item in the command result of VIRTIO_ADMIN_CMD_LEGACY_NOTIFY_INFO, valid
entries MUST have a \field{bar} which is not hardwired to zero.

\item in the command result of VIRTIO_ADMIN_CMD_LEGACY_NOTIFY_INFO, valid
entries MUST have an \field{offset} aligned to 2-byte.

\item the device MAY support VIRTIO_ADMIN_CMD_LEGACY_NOTIFY_INFO with entries
of the owner device or the member device or both of them.

\item for the SR-IOV group type, the group owner device MUST hardwire VF BAR0
to zero in the SR-IOV Extended capability.
\end{itemize}

\drivernormative{\paragraph}{Legacy Interface}{Basic Facilities of a Virtio Device / Device groups / Group administration commands / Legacy Interface}

For VIRTIO_ADMIN_CMD_LEGACY_COMMON_CFG_WRITE,
VIRTIO_ADMIN_CMD_LEGACY_COMMON_CFG_READ,
VIRTIO_ADMIN_CMD_LEGACY_DEV_CFG_WRITE and
VIRTIO_ADMIN_CMD_LEGACY_DEV_CFG_READ commands,
the driver MUST encode and decode (respectively) the value of the \field{data}
using the little-endian format.

For the VIRTIO_ADMIN_CMD_LEGACY_COMMON_CFG_WRITE and
VIRTIO_ADMIN_CMD_LEGACY_COMMON_CFG_READ commands,
the driver SHOULD set \field{offset} and the length of the \field{data}
to refer to a single field within the virtio common configuration structure
excluding the device-specific configuration.

For the VIRTIO_ADMIN_CMD_LEGACY_DEV_CFG_WRITE and
VIRTIO_ADMIN_CMD_LEGACY_DEV_CFG_READ commands,
the driver SHOULD set \field{offset} and the length of the \field{data}
to refer to a single field within device specific configuration.

If VIRTIO_ADMIN_CMD_LEGACY_NOTIFY_INFO command is supported, the driver
SHOULD use the notification address to send all driver notifications to the
device.

If within \field{struct virtio_admin_cmd_legacy_notify_info_result} returned by
VIRTIO_ADMIN_CMD_LEGACY_NOTIFY_INFO, the \field{flags} value
for a specific \field{struct virtio_pci_legacy_notify_info} entry is 0x0, the
driver MUST ignore this entry and all the following \field{entries}.
Additionally, for all other entries, the driver MUST validate that
\begin{itemize}
\item the \field{flags} is either 0x1 or 0x2
\item the \field{bar} corresponds to a valid BAR of either the owner or the
member device, depending on the \field{flags}
\item the \field{offset} is 2-byte aligned and corresponds to an address
within the BAR specified by the \field{bar}
on \field{flags}
\end{itemize}, any entry which does not meet these constraints MUST be ignored
by the driver.
