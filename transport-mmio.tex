\section{Virtio Over MMIO}\label{sec:Virtio Transport Options / Virtio Over MMIO}

Virtual environments without PCI support (a common situation in
embedded devices models) might use simple memory mapped device
(``virtio-mmio'') instead of the PCI device.

The memory mapped virtio device behaviour is based on the PCI
device specification. Therefore most operations including device
initialization, queues configuration and buffer transfers are
nearly identical. Existing differences are described in the
following sections.

\subsection{MMIO Device Discovery}\label{sec:Virtio Transport Options / Virtio Over MMIO / MMIO Device Discovery}

Unlike PCI, MMIO provides no generic device discovery mechanism.  For each
device, the guest OS will need to know the location of the registers
and interrupt(s) used.  The suggested binding for systems using
flattened device trees is shown in this example:

\begin{lstlisting}
// EXAMPLE: virtio_block device taking 512 bytes at 0x1e000, interrupt 42.
virtio_block@1e000 {
        compatible = "virtio,mmio";
        reg = <0x1e000 0x200>;
        interrupts = <42>;
}
\end{lstlisting}

\subsection{MMIO Device Register Layout}\label{sec:Virtio Transport Options / Virtio Over MMIO / MMIO Device Register Layout}

MMIO virtio devices provide a set of memory mapped control
registers followed by a device-specific configuration space,
described in the table~\ref{tab:Virtio Transport Options / Virtio Over MMIO / MMIO Device Register Layout}.

All register values are organized as Little Endian.

\newcommand{\mmioreg}[5]{% Name Function Offset Direction Description
  {\field{#1}} \newline #3 \newline #4 & {\bf#2} \newline #5 \\
}

\newcommand{\mmiodreg}[7]{% NameHigh NameLow Function OffsetHigh OffsetLow Direction Description
  {\field{#1}} \newline #4 \newline {\field{#2}} \newline #5 \newline #6 & {\bf#3} \newline #7 \\
}

\begin{longtable}{p{0.2\textwidth}p{0.7\textwidth}}
  \caption {MMIO Device Register Layout}
  \label{tab:Virtio Transport Options / Virtio Over MMIO / MMIO Device Register Layout} \\
  \hline
  \mmioreg{Name}{Function}{Offset from base}{Direction}{Description}
  \hline
  \hline
  \endfirsthead
  \hline
  \mmioreg{Name}{Function}{Offset from the base}{Direction}{Description}
  \hline
  \hline
  \endhead
  \endfoot
  \endlastfoot
  \mmioreg{MagicValue}{Magic value}{0x000}{R}{%
    0x74726976
    (a Little Endian equivalent of the ``virt'' string).
  }
  \hline
  \mmioreg{Version}{Device version number}{0x004}{R}{%
    0x2.
    \begin{note}
      Legacy devices (see \ref{sec:Virtio Transport Options / Virtio Over MMIO / Legacy interface}~\nameref{sec:Virtio Transport Options / Virtio Over MMIO / Legacy interface}) used 0x1.
    \end{note}
  }
  \hline
  \mmioreg{DeviceID}{Virtio Subsystem Device ID}{0x008}{R}{%
    See \ref{sec:Device Types}~\nameref{sec:Device Types} for possible values.
    Value zero (0x0) is used to
    define a system memory map with placeholder devices at static,
    well known addresses, assigning functions to them depending
    on user's needs.
  }
  \hline
  \mmioreg{VendorID}{Virtio Subsystem Vendor ID}{0x00c}{R}{}
  \hline
  \mmioreg{DeviceFeatures}{Flags representing features the device supports}{0x010}{R}{%
    Reading from this register returns 32 consecutive flag bits,
    the least significant bit depending on the last value written to
    \field{DeviceFeaturesSel}. Access to this register returns
    bits $\field{DeviceFeaturesSel}*32$ to $(\field{DeviceFeaturesSel}*32)+31$, eg.
    feature bits 0 to 31 if \field{DeviceFeaturesSel} is set to 0 and
    features bits 32 to 63 if \field{DeviceFeaturesSel} is set to 1.
    Also see \ref{sec:Basic Facilities of a Virtio Device / Feature Bits}~\nameref{sec:Basic Facilities of a Virtio Device / Feature Bits}.
  }
  \hline
  \mmioreg{DeviceFeaturesSel}{Device (host) features word selection.}{0x014}{W}{%
    Writing to this register selects a set of 32 device feature bits
    accessible by reading from \field{DeviceFeatures}.
  }
  \hline
  \mmioreg{DriverFeatures}{Flags representing device features understood and activated by the driver}{0x020}{W}{%
    Writing to this register sets 32 consecutive flag bits, the least significant
    bit depending on the last value written to \field{DriverFeaturesSel}.
     Access to this register sets bits $\field{DriverFeaturesSel}*32$
    to $(\field{DriverFeaturesSel}*32)+31$, eg. feature bits 0 to 31 if
    \field{DriverFeaturesSel} is set to 0 and features bits 32 to 63 if
    \field{DriverFeaturesSel} is set to 1. Also see \ref{sec:Basic Facilities of a Virtio Device / Feature Bits}~\nameref{sec:Basic Facilities of a Virtio Device / Feature Bits}.
  }
  \hline
  \mmioreg{DriverFeaturesSel}{Activated (guest) features word selection}{0x024}{W}{%
    Writing to this register selects a set of 32 activated feature
    bits accessible by writing to \field{DriverFeatures}.
  }
  \hline
  \mmioreg{QueueSel}{Virtual queue index}{0x030}{W}{%
    Writing to this register selects the virtual queue that the
    following operations on \field{QueueNumMax}, \field{QueueNum}, \field{QueueReady},
    \field{QueueDescLow}, \field{QueueDescHigh}, \field{QueueDriverlLow}, \field{QueueDriverHigh},
    \field{QueueDeviceLow}, \field{QueueDeviceHigh} and \field{QueueReset} apply to. The index
    number of the first queue is zero (0x0).
  }
  \hline
  \mmioreg{QueueNumMax}{Maximum virtual queue size}{0x034}{R}{%
    Reading from the register returns the maximum size (number of
    elements) of the queue the device is ready to process or
    zero (0x0) if the queue is not available. This applies to the
    queue selected by writing to \field{QueueSel}.
  }
  \hline
  \mmioreg{QueueNum}{Virtual queue size}{0x038}{W}{%
    Queue size is the number of elements in the queue.
    Writing to this register notifies the device what size of the
    queue the driver will use. This applies to the queue selected by
    writing to \field{QueueSel}.
  }
  \hline
  \mmioreg{QueueReady}{Virtual queue ready bit}{0x044}{RW}{%
    Writing one (0x1) to this register notifies the device that it can
    execute requests from this virtual queue. Reading from this register
    returns the last value written to it. Both read and write
    accesses apply to the queue selected by writing to \field{QueueSel}.
  }
  \hline
  \mmioreg{QueueNotify}{Queue notifier}{0x050}{W}{%
    Writing a value to this register notifies the device that
    there are new buffers to process in a queue.

    When VIRTIO_F_NOTIFICATION_DATA has not been negotiated,
    the value written is the queue index.

    When VIRTIO_F_NOTIFICATION_DATA has been negotiated,
    the \field{Notification data} value has the following format:

    \lstinputlisting{notifications-le.c}

    See \ref{sec:Basic Facilities of a Virtio Device / Driver notifications}~\nameref{sec:Basic Facilities of a Virtio Device / Driver notifications}
    for the definition of the components.
  }
  \hline
  \mmioreg{InterruptStatus}{Interrupt status}{0x60}{R}{%
    Reading from this register returns a bit mask of events that
    caused the device interrupt to be asserted.
    The following events are possible:
    \begin{description}
      \item[Used Buffer Notification] - bit 0 - the interrupt was asserted
        because the device has used a buffer
        in at least one of the active virtual queues.
      \item [Configuration Change Notification] - bit 1 - the interrupt was
        asserted because the configuration of the device has changed.
    \end{description}
  }
  \hline
  \mmioreg{InterruptACK}{Interrupt acknowledge}{0x064}{W}{%
    Writing a value with bits set as defined in \field{InterruptStatus}
    to this register notifies the device that events causing
    the interrupt have been handled.
  }
  \hline
  \mmioreg{Status}{Device status}{0x070}{RW}{%
    Reading from this register returns the current device status
    flags.
    Writing non-zero values to this register sets the status flags,
    indicating the driver progress. Writing zero (0x0) to this
    register triggers a device reset.
    See also p. \ref{sec:Virtio Transport Options / Virtio Over MMIO / MMIO-specific Initialization And Device Operation / Device Initialization}~\nameref{sec:Virtio Transport Options / Virtio Over MMIO / MMIO-specific Initialization And Device Operation / Device Initialization}.
  }
  \hline
  \mmiodreg{QueueDescLow}{QueueDescHigh}{Virtual queue's Descriptor Area 64 bit long physical address}{0x080}{0x084}{W}{%
    Writing to these two registers (lower 32 bits of the address
    to \field{QueueDescLow}, higher 32 bits to \field{QueueDescHigh}) notifies
    the device about location of the Descriptor Area of the queue
    selected by writing to \field{QueueSel} register.
  }
  \hline
  \mmiodreg{QueueDriverLow}{QueueDriverHigh}{Virtual queue's Driver Area 64 bit long physical address}{0x090}{0x094}{W}{%
    Writing to these two registers (lower 32 bits of the address
    to \field{QueueDriverLow}, higher 32 bits to \field{QueueDriverHigh}) notifies
    the device about location of the Driver Area of the queue
    selected by writing to \field{QueueSel}.
  }
  \hline
  \mmiodreg{QueueDeviceLow}{QueueDeviceHigh}{Virtual queue's Device Area 64 bit long physical address}{0x0a0}{0x0a4}{W}{%
    Writing to these two registers (lower 32 bits of the address
    to \field{QueueDeviceLow}, higher 32 bits to \field{QueueDeviceHigh}) notifies
    the device about location of the Device Area of the queue
    selected by writing to \field{QueueSel}.
  }
  \hline
  \mmioreg{SHMSel}{Shared memory id}{0x0ac}{W}{%
    Writing to this register selects the shared memory region \ref{sec:Basic Facilities of a Virtio Device / Shared Memory Regions}
    following operations on \field{SHMLenLow}, \field{SHMLenHigh},
    \field{SHMBaseLow} and \field{SHMBaseHigh} apply to.
  }
  \hline
  \mmiodreg{SHMLenLow}{SHMLenHigh}{Shared memory region 64 bit long length}{0x0b0}{0x0b4}{R}{%
    These registers return the length of the shared memory
    region in bytes, as defined by the device for the region selected by
    the \field{SHMSel} register.  The lower 32 bits of the length
    are read from \field{SHMLenLow} and the higher 32 bits from
    \field{SHMLenHigh}.  Reading from a non-existent
    region (i.e. where the ID written to \field{SHMSel} is unused)
    results in a length of -1.
  }
  \hline
  \mmiodreg{SHMBaseLow}{SHMBaseHigh}{Shared memory region 64 bit long physical address}{0x0b8}{0x0bc}{R}{%
    The driver reads these registers to discover the base address
    of the region in physical address space.  This address is
    chosen by the device (or other part of the VMM).
    The lower 32 bits of the address are read from \field{SHMBaseLow}
    with the higher 32 bits from \field{SHMBaseHigh}.  Reading
    from a non-existent region (i.e. where the ID written to
    \field{SHMSel} is unused) results in a base address of
    0xffffffffffffffff.
  }
  \hline
  \mmioreg{QueueReset}{Virtual queue reset bit}{0x0c0}{RW}{%
    If VIRTIO_F_RING_RESET has been negotiated, writing one (0x1) to this
    register selectively resets the queue. Both read and write accesses
    apply to the queue selected by writing to \field{QueueSel}.
  }
  \hline
  \mmioreg{ConfigGeneration}{Configuration atomicity value}{0x0fc}{R}{
    Reading from this register returns a value describing a version of the device-specific configuration space (see \field{Config}).
    The driver can then access the configuration space and, when finished, read \field{ConfigGeneration} again.
    If no part of the configuration space has changed between these two \field{ConfigGeneration} reads, the returned values are identical.
    If the values are different, the configuration space accesses were not atomic and the driver has to perform the operations again.
    See also \ref {sec:Basic Facilities of a Virtio Device / Device Configuration Space}.
  }
  \hline
  \mmioreg{Config}{Configuration space}{0x100+}{RW}{
    Device-specific configuration space starts at the offset 0x100
    and is accessed with byte alignment. Its meaning and size
    depend on the device and the driver.
  }
  \hline
\end{longtable}

\devicenormative{\subsubsection}{MMIO Device Register Layout}{Virtio Transport Options / Virtio Over MMIO / MMIO Device Register Layout}

The device MUST return 0x74726976 in \field{MagicValue}.

The device MUST return value 0x2 in \field{Version}.

The device MUST present each event by setting the corresponding bit in \field{InterruptStatus} from the
moment it takes place, until the driver acknowledges the interrupt
by writing a corresponding bit mask to the \field{InterruptACK} register.  Bits which
do not represent events which took place MUST be zero.

Upon reset, the device MUST clear all bits in \field{InterruptStatus} and ready bits in the
\field{QueueReady} register for all queues in the device.

The device MUST change value returned in \field{ConfigGeneration} if there is any risk of a
driver seeing an inconsistent configuration state.

The device MUST NOT access virtual queue contents when \field{QueueReady} is zero (0x0).

If VIRTIO_F_RING_RESET has been negotiated, the device MUST present a 0 in
\field{QueueReset} on reset.

If VIRTIO_F_RING_RESET has been negotiated, The device MUST present a 0 in
\field{QueueReset} after the virtqueue is enabled with \field{QueueReady}.

The device MUST reset the queue when 1 is written to \field{QueueReset}. The
device MUST continue to present 1 in \field{QueueReset} as long as the queue reset
is ongoing. The device MUST present 0 in both \field{QueueReset} and \field{QueueReady}
when queue reset has completed.
(see \ref{sec:Basic Facilities of a Virtio Device / Virtqueues / Virtqueue Reset}).

\drivernormative{\subsubsection}{MMIO Device Register Layout}{Virtio Transport Options / Virtio Over MMIO / MMIO Device Register Layout}
The driver MUST NOT access memory locations not described in the
table \ref{tab:Virtio Transport Options / Virtio Over MMIO / MMIO Device Register Layout}
(or, in case of the configuration space, described in the device specification),
MUST NOT write to the read-only registers (direction R) and
MUST NOT read from the write-only registers (direction W).

The driver MUST only use 32 bit wide and aligned reads and writes to access the control registers
described in table \ref{tab:Virtio Transport Options / Virtio Over MMIO / MMIO Device Register Layout}.
For the device-specific configuration space, the driver MUST use 8 bit wide accesses for
8 bit wide fields, 16 bit wide and aligned accesses for 16 bit wide fields and 32 bit wide and
aligned accesses for 32 and 64 bit wide fields.

The driver MUST ignore a device with \field{MagicValue} which is not 0x74726976,
although it MAY report an error.

The driver MUST ignore a device with \field{Version} which is not 0x2,
although it MAY report an error.

The driver MUST ignore a device with \field{DeviceID} 0x0,
but MUST NOT report any error.

Before reading from \field{DeviceFeatures}, the driver MUST write a value to \field{DeviceFeaturesSel}.

Before writing to the \field{DriverFeatures} register, the driver MUST write a value to the \field{DriverFeaturesSel} register.

The driver MUST write a value to \field{QueueNum} which is less than
or equal to the value presented by the device in \field{QueueNumMax}.

When \field{QueueReady} is not zero, the driver MUST NOT access
\field{QueueNum}, \field{QueueDescLow}, \field{QueueDescHigh},
\field{QueueDriverLow}, \field{QueueDriverHigh}, \field{QueueDeviceLow}, \field{QueueDeviceHigh}.

To stop using the queue the driver MUST write zero (0x0) to this
\field{QueueReady} and MUST read the value back to ensure
synchronization.

The driver MUST ignore undefined bits in \field{InterruptStatus}.

The driver MUST write a value with a bit mask describing events it handled into \field{InterruptACK} when
it finishes handling an interrupt and MUST NOT set any of the undefined bits in the value.

If VIRTIO_F_RING_RESET has been negotiated, after the driver writes 1 to
\field{QueueReset} to reset the queue, the driver MUST NOT consider queue
reset to be complete until it reads back 0 in \field{QueueReset}. The driver
MAY re-enable the queue by writing 1 to \field{QueueReady} after ensuring
that other virtqueue fields have been set up correctly. The driver MAY set
driver-writeable queue configuration values to different values than those
that were used before the queue reset.
(see \ref{sec:Basic Facilities of a Virtio Device / Virtqueues / Virtqueue Reset}).

\subsection{MMIO-specific Initialization And Device Operation}\label{sec:Virtio Transport Options / Virtio Over MMIO / MMIO-specific Initialization And Device Operation}

\subsubsection{Device Initialization}\label{sec:Virtio Transport Options / Virtio Over MMIO / MMIO-specific Initialization And Device Operation / Device Initialization}

\drivernormative{\paragraph}{Device Initialization}{Virtio Transport Options / Virtio Over MMIO / MMIO-specific Initialization And Device Operation / Device Initialization}

The driver MUST start the device initialization by reading and
checking values from \field{MagicValue} and \field{Version}.
If both values are valid, it MUST read \field{DeviceID}
and if its value is zero (0x0) MUST abort initialization and
MUST NOT access any other register.

Drivers not expecting shared memory MUST NOT use the shared
memory registers.

Further initialization MUST follow the procedure described in
\ref{sec:General Initialization And Device Operation / Device Initialization}~\nameref{sec:General Initialization And Device Operation / Device Initialization}.

\subsubsection{Virtqueue Configuration}\label{sec:Virtio Transport Options / Virtio Over MMIO / MMIO-specific Initialization And Device Operation / Virtqueue Configuration}

The driver will typically initialize the virtual queue in the following way:

\begin{enumerate}
\item Select the queue writing its index (first queue is 0) to
   \field{QueueSel}.

\item Check if the queue is not already in use: read \field{QueueReady},
   and expect a returned value of zero (0x0).

\item Read maximum queue size (number of elements) from
   \field{QueueNumMax}. If the returned value is zero (0x0) the
   queue is not available.

\item Allocate and zero the queue memory, making sure the memory
   is physically contiguous.

\item Notify the device about the queue size by writing the size to
   \field{QueueNum}.

\item Write physical addresses of the queue's Descriptor Area,
   Driver Area and Device Area to (respectively) the
   \field{QueueDescLow}/\field{QueueDescHigh},
   \field{QueueDriverLow}/\field{QueueDriverHigh} and
   \field{QueueDeviceLow}/\field{QueueDeviceHigh} register pairs.

\item Write 0x1 to \field{QueueReady}.
\end{enumerate}

\subsubsection{Available Buffer Notifications}\label{sec:Virtio Transport Options / Virtio Over MMIO / MMIO-specific Initialization And Device Operation / Available Buffer Notifications}

When VIRTIO_F_NOTIFICATION_DATA has not been negotiated,
the driver sends an available buffer notification to the device by writing
the 16-bit virtqueue index
of the queue to be notified to \field{QueueNotify}.

When VIRTIO_F_NOTIFICATION_DATA has been negotiated,
the driver sends an available buffer notification to the device by writing
the following 32-bit value to \field{QueueNotify}:
\lstinputlisting{notifications-le.c}

See \ref{sec:Basic Facilities of a Virtio Device / Driver notifications}~\nameref{sec:Basic Facilities of a Virtio Device / Driver notifications}
for the definition of the components.

\subsubsection{Notifications From The Device}\label{sec:Virtio Transport Options / Virtio Over MMIO / MMIO-specific Initialization And Device Operation / Notifications From The Device}

The memory mapped virtio device is using a single, dedicated
interrupt signal, which is asserted when at least one of the
bits described in the description of \field{InterruptStatus}
is set. This is how the device sends a used buffer notification
or a configuration change notification to the device.

\drivernormative{\paragraph}{Notifications From The Device}{Virtio Transport Options / Virtio Over MMIO / MMIO-specific Initialization And Device Operation / Notifications From The Device}
After receiving an interrupt, the driver MUST read
\field{InterruptStatus} to check what caused the interrupt (see the
register description).  The used buffer notification bit being set
SHOULD be interpreted as a used buffer notification for each active
virtqueue.  After the interrupt is handled, the driver MUST acknowledge
it by writing a bit mask corresponding to the handled events to the
InterruptACK register.

\subsection{Legacy interface}\label{sec:Virtio Transport Options / Virtio Over MMIO / Legacy interface}

The legacy MMIO transport used page-based addressing, resulting
in a slightly different control register layout, the device
initialization and the virtual queue configuration procedure.

Table \ref{tab:Virtio Transport Options / Virtio Over MMIO / MMIO Device Legacy Register Layout}
presents control registers layout, omitting
descriptions of registers which did not change their function
nor behaviour:

\begin{longtable}{p{0.2\textwidth}p{0.7\textwidth}}
  \caption {MMIO Device Legacy Register Layout}
  \label{tab:Virtio Transport Options / Virtio Over MMIO / MMIO Device Legacy Register Layout} \\
  \hline
  \mmioreg{Name}{Function}{Offset from base}{Direction}{Description}
  \hline
  \hline
  \endfirsthead
  \hline
  \mmioreg{Name}{Function}{Offset from the base}{Direction}{Description}
  \hline
  \hline
  \endhead
  \endfoot
  \endlastfoot
  \mmioreg{MagicValue}{Magic value}{0x000}{R}{}
  \hline
  \mmioreg{Version}{Device version number}{0x004}{R}{Legacy device returns value 0x1.}
  \hline
  \mmioreg{DeviceID}{Virtio Subsystem Device ID}{0x008}{R}{}
  \hline
  \mmioreg{VendorID}{Virtio Subsystem Vendor ID}{0x00c}{R}{}
  \hline
  \mmioreg{HostFeatures}{Flags representing features the device supports}{0x010}{R}{}
  \hline
  \mmioreg{HostFeaturesSel}{Device (host) features word selection.}{0x014}{W}{}
  \hline
  \mmioreg{GuestFeatures}{Flags representing device features understood and activated by the driver}{0x020}{W}{}
  \hline
  \mmioreg{GuestFeaturesSel}{Activated (guest) features word selection}{0x024}{W}{}
  \hline
  \mmioreg{GuestPageSize}{Guest page size}{0x028}{W}{%
    The driver writes the guest page size in bytes to the
    register during initialization, before any queues are used.
    This value should be a power of 2 and is used by the device to
    calculate the Guest address of the first queue page
    (see QueuePFN).
  }
  \hline
  \mmioreg{QueueSel}{Virtual queue index}{0x030}{W}{%
    Writing to this register selects the virtual queue that the
    following operations on the \field{QueueNumMax}, \field{QueueNum}, \field{QueueAlign}
    and \field{QueuePFN} registers apply to. The index
    number of the first queue is zero (0x0).
.
  }
  \hline
  \mmioreg{QueueNumMax}{Maximum virtual queue size}{0x034}{R}{%
    Reading from the register returns the maximum size of the queue
    the device is ready to process or zero (0x0) if the queue is not
    available. This applies to the queue selected by writing to
    \field{QueueSel} and is allowed only when \field{QueuePFN} is set to zero
    (0x0), so when the queue is not actively used.
  }
  \hline
  \mmioreg{QueueNum}{Virtual queue size}{0x038}{W}{%
    Queue size is the number of elements in the queue.
    Writing to this register notifies the device what size of the
    queue the driver will use. This applies to the queue selected by
    writing to \field{QueueSel}.
  }
  \hline
  \mmioreg{QueueAlign}{Used Ring alignment in the virtual queue}{0x03c}{W}{%
    Writing to this register notifies the device about alignment
    boundary of the Used Ring in bytes. This value should be a power
    of 2 and applies to the queue selected by writing to \field{QueueSel}.
  }
  \hline
  \mmioreg{QueuePFN}{Guest physical page number of the virtual queue}{0x040}{RW}{%
    Writing to this register notifies the device about location of the
    virtual queue in the Guest's physical address space. This value
    is the index number of a page starting with the queue
    Descriptor Table. Value zero (0x0) means physical address zero
    (0x00000000) and is illegal. When the driver stops using the
    queue it writes zero (0x0) to this register.
    Reading from this register returns the currently used page
    number of the queue, therefore a value other than zero (0x0)
    means that the queue is in use.
    Both read and write accesses apply to the queue selected by
    writing to \field{QueueSel}.
  }
  \hline
  \mmioreg{QueueNotify}{Queue notifier}{0x050}{W}{}
  \hline
  \mmioreg{InterruptStatus}{Interrupt status}{0x60}{R}{}
  \hline
  \mmioreg{InterruptACK}{Interrupt acknowledge}{0x064}{W}{}
  \hline
  \mmioreg{Status}{Device status}{0x070}{RW}{%
    Reading from this register returns the current device status
    flags.
    Writing non-zero values to this register sets the status flags,
    indicating the OS/driver progress. Writing zero (0x0) to this
    register triggers a device reset. The device
    sets \field{QueuePFN} to zero (0x0) for all queues in the device.
    Also see \ref{sec:General Initialization And Device Operation / Device Initialization}~\nameref{sec:General Initialization And Device Operation / Device Initialization}.
  }
  \hline
  \mmioreg{Config}{Configuration space}{0x100+}{RW}{}
  \hline
\end{longtable}

The virtual queue page size is defined by writing to \field{GuestPageSize},
as written by the guest. The driver does this before the
virtual queues are configured.

The virtual queue layout follows
p. \ref{sec:Basic Facilities of a Virtio Device / Virtqueues / Legacy Interfaces: A Note on Virtqueue Layout}~\nameref{sec:Basic Facilities of a Virtio Device / Virtqueues / Legacy Interfaces: A Note on Virtqueue Layout},
with the alignment defined in \field{QueueAlign}.

The virtual queue is configured as follows:
\begin{enumerate}
\item Select the queue writing its index (first queue is 0) to
   \field{QueueSel}.

\item Check if the queue is not already in use: read \field{QueuePFN},
   expecting a returned value of zero (0x0).

\item Read maximum queue size (number of elements) from
   \field{QueueNumMax}. If the returned value is zero (0x0) the
   queue is not available.

\item Allocate and zero the queue pages in contiguous virtual
   memory, aligning the Used Ring to an optimal boundary (usually
   page size). The driver should choose a queue size smaller than or
   equal to \field{QueueNumMax}.

\item Notify the device about the queue size by writing the size to
   \field{QueueNum}.

\item Notify the device about the used alignment by writing its value
   in bytes to \field{QueueAlign}.

\item Write the physical number of the first page of the queue to
   the \field{QueuePFN} register.
\end{enumerate}

Notification mechanisms did not change.

\subsection{Features reserved for future use}\label{sec:Virtio Transport Options / Virtio Over MMIO / Features reserved for future use}

Devices and drivers utilizing Virtio Over MMIO
do not support the following features:
\begin{itemize}

\item VIRTIO_F_ADMIN_VQ

\end{itemize}

These features are reserved for future use.
