\section{PMEM Device}\label{sec:Device Types / PMEM Device}

The virtio pmem device is a persistent memory (NVDIMM) device
that provides a virtio based asynchronous flush mechanism. This avoids
the need for a separate page cache in the guest and keeps the page cache
only in the host. Under memory pressure, the host makes use of
efficient memory reclaim decisions for page cache pages of all the
guests. This helps to reduce the memory footprint and fits more guests
in the host system.

The virtio pmem device provides access to byte-addressable persistent
memory. The persistent memory is a directly accessible range of system memory.
Data written to this memory is made persistent by separately sending a
flush command. Writes that have been flushed are preserved across device
reset and power failure.

\subsection{Device ID}\label{sec:Device Types / PMEM Device / Device ID}
  27

\subsection{Virtqueues}\label{sec:Device Types / PMEM Device / Virtqueues}
\begin{description}
\item[0] req_vq
\end{description}

\subsection{Feature bits}\label{sec:Device Types / PMEM Device / Feature bits}

There are currently no feature bits defined for this device.

\subsection{Device configuration layout}\label{sec:Device Types / PMEM Device / Device configuration layout}

\begin{lstlisting}
struct virtio_pmem_config {
	le64 start;
	le64 size;
};
\end{lstlisting}

\begin{description}
\item[\field{start}] contains the physical address of the first byte of the persistent memory region.

\item[\field{size}] contains the length of this address range.
\end{description}

\begin{enumerate}
\item Driver vpmem start is read from \field{start}.
\item Driver vpmem end is read from \field{size}.
\end{enumerate}

\subsection{Driver Initialization}\label{sec:Device Types / PMEM Driver / Driver Initialization}

The driver determines the start address and size of the persistent memory region in preparation for reading or writing data.

The driver initializes req_vq in preparation for making flush requests.

\subsection{Driver Operations}\label{sec:Device Types / PMEM Driver / Driver Operation / Request Queues}

Requests have the following format:

\begin{lstlisting}
struct virtio_pmem_req {
        le32 type;
};
\end{lstlisting}

\field{type} is the request command type.

Possible request types are:

\begin{lstlisting}
#define VIRTIO_PMEM_REQ_TYPE_FLUSH      0
\end{lstlisting}

\subsection{Device Operations}\label{sec:Device Types / PMEM Driver / Device Operation}
\devicenormative{\subsubsection}{Device Operation: Virtqueue flush}{Device Types / PMEM Device / Device Operation / Virtqueue flush}

The device MUST ensure that all writes completed before a flush request persist across device reset and power failure before completing the flush request.

\subsubsection{Device Operations}\label{sec:Device Types / PMEM Driver / Device Operation / Virtqueue return}
\begin{lstlisting}
struct virtio_pmem_resp {
        le32 ret;
};
\end{lstlisting}

\field{ret} is the value which the device returns after command completion.

\devicenormative{\subsubsection}{Device Operation: Virtqueue return}{Device Types / PMEM Device / Device Operation / Virtqueue return}

The device MUST return "0" for success and "-1" for failure.

\subsection{Possible security implications}\label{sec:Device Types / PMEM Device / Possible Security Implications}

There could be potential security implications depending on how
memory mapped backing device is used. By default device emulation
is done with SHARED memory mapping. There is a contract between driver
and device to access shared memory region for read or write operations.

If a malicious driver or device maps the same memory region, the attacker
can make use of known side channel attacks to predict the current state of data.
If both attacker and victim somehow execute same shared code after a flush
or evict operation, with difference in execution timing attacker could infer
another device's data.

\subsection{Countermeasures}\label{sec:Device Types / PMEM Device / Possible Security Implications / Countermeasures}

\subsubsection{With SHARED mapping}\label{sec:Device Types / PMEM Device / Possible Security Implications / Countermeasures / SHARED}

If a device's backing region is shared between multiple devices, this may act
as a metric for side channel attacks. As a counter measure every device
should have its own (not shared with another driver) SHARED backing memory.

\subsubsection{With PRIVATE mapping}\label{sec:Device Types / PMEM Device / Possible Security Implications / Countermeasures / PRIVATE}
There maybe be chances of side channels attack with PRIVATE
memory mapping similar to SHARED with read-only shared mappings.
PRIVATE is not used for virtio pmem making this usecase
irrelevant.

\subsubsection{Workload specific mapping}\label{sec:Device Types / PMEM Device / Possible Security Implications / Countermeasures / Workload}
When using SHARED mappings with a workload that is a single application
inside the driver where the risk in sharing data is very low or
nonexisting, the device sharing the same backing region with a SHARED
mapping can be used as a valid configuration.

\subsubsection{Prevent cache eviction}\label{sec:Device Types / PMEM Device / Possible Security Implications / Countermeasures / Cache eviction}
Don't allow device shared region eviction from driver filesystem trim or discard
like commands with virtio pmem. This rules out any possibility of evict-reload
cache side channel attacks if backing region is shared (SHARED)
between mutliple devices. Though if we use per device backing file with
shared mapping this countermeasure is not required.
